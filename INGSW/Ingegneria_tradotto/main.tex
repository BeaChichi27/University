% !TeX root = main.tex
% !TeX program = pdflatex
% !TeX options = -synctex=1
\documentclass[a4paper]{report}
\usepackage{pdfpages}
\newcounter{insertpages}
\usepackage{amsmath}
\usepackage[utf8]{inputenc}
\usepackage{booktabs}
\usepackage{graphicx}
\usepackage[italian]{babel}
\usepackage[hidelinks]{hyperref}
\usepackage{listings}
\usepackage{xcolor}
\usepackage{ulem}
\usepackage{enumitem}
\usepackage{algorithm}
\usepackage{placeins}
\usepackage{algpseudocode}
\usepackage{caption}
\usepackage{array}
\usepackage{accsupp}
\usepackage[margin=50pt]{geometry}
\usepackage{pdfpages}

\title{Ingegneria del Software}
\author{Beatrice Chiara Morgillo}

\begin{document}

\maketitle
\tableofcontents
\chapter{Lezione 2: Processi e qualità del software}
\section{Processi software}
Un processo software è un insieme di attività correllate che conducono alla produzione di un sistema software. Non c'è un processo universale che funziona sempre, ma ce ne sono di molteplici e tutti includono, in qualche modo, delle attività fondamentali:
\begin{enumerate}
    \item \textbf{Specificazione software}: i vincoli funzionali e operativi vengono definiti.
    \item \textbf{Implementazione software}: il software incontra i requisiti che devono essere prodotti.
    \item \textbf{Validazione software}: il software dev'essere validato per verificare che segua i requisiti.
    \item \textbf{Evolutzione del software}: il software deve evolvere per incontrare i cambiamenti necessari.
\end{enumerate}

\subsection{I modelli di processo software}

Un \textbf{modello di processo software} o \textbf{ciclo di vita di un sistema software} (sigla in inglese: SDLC) è una rappresentazione semplificata di un processo software.

Un modello di processo software può interessarsi si una prospettiva particolare e ci sono dei modelli di processo molto generali, come il modello a cascata.

\subsection{Il modello a cascata}
I processi software consisteno di un numero di passaggi sequenziali, in un processo plan-driven, il risultato di ogni fase è un documento che viene approvato e la fase successiva non può iniziare se la precedente non è completa. I passi da seguire sono:
\begin{enumerate}
    \item \textbf{Ingegnerizzazione dei requisiti}.
    \item \textbf{Design del sistema}.
    \item \textbf{Design dell'UI e del software}. 
    \item \textbf{Implementazione}.
    \item \textbf{Testing}.
    \item \textbf{Operazioni e manutenzione}.
\end{enumerate}

Questo modello rigido, che segue i un approccio che va in base alle previsioni ha senso per l'ingegnerizzazione dell'hardware, dove sono considerati anche gli alti costi di produzione, invece per lo sviluppo software questi passaggi possono sovrapporsi e dare informazioni a vicenda. Durante il design del sistema vengono identificati i problemi coi loro requisiti, durante l'implementazione vengono trovati i problemi col design del software e i requisiti possono cambiare.

\section{Ingegnerizzazione dei requisiti}
L'obiettivo è capire cosa dovrebbe fare il software (e non come dev'essere implementato). Viene fatta un'attenta analisi di cos'ha bisogno l'utente e del problema di dominio.
\newline
Vengono inclusi i clienti, gli utenti finali e gli ingegneri del software, in modo da avere come output principale un \textbf{documento che specifichi i requisiti software}.

\section{Desing del sistema e del software/UI}
L'obiettivo è avere un design adeguato della struttura del software e si sviluppa su due livelli diversi:
\begin{itemize}
    \item \textbf{Design di sistema}: l'architettura generale del sistema.
    \begin{itemize}
        \item Decomposizione in moduli e componenti.
        \item Allocazione di funzionalità per i moduli e moduli per le componenti hardware.
        \item Relazioni e collaborazioni tra i moduli definiti.
    \end{itemize}
    \item \textbf{Design della UI e del software}: dettagli su come implementare ogni modulo.
    \begin{itemize}
        \item include design di un'architettura software di basso livello.
        \item include una prototipazione dell'UI.
    \end{itemize}
\end{itemize}
Il risultato è un insieme di specificazioni di design, spesso formalizzate usando linguaggi di design come UML.
\section{Implementazione}
Il risultato è "tradurre" le specificazioni di design in un linguaggio/tecnologia di programmazione scelta. Non è una traduzione qualunque, ma una di \textbf{alta qualità} e il codice risultante dovrebbe essere \textbf{"clean"}.
\section{Verificazione e validazione (V\&V)}
Si usa per vedere se le implementazioni soddisfano appieno le necessità dell'utente.
\begin{itemize}
    \item \textbf{Verificazione}: sono domande per quanto riguarda la conformità del sistema con le sue specifiche.
    \item \textbf{Validazione}: sono domande per verificare se il sistema incontra le aspettative del cliente.
\end{itemize}
\section{Operazioni e manutenzione}
L'operazione si occupa di distribuire il sistema e renderlo installabile per il cliente, mettendolo effettivamente in uso. La manutenzione stabilisce che il software potrà cambiare in un certo punto, infatti le necessità del cliente possono cambiare, il contesto d'uso potrebbe e i bug che sono sfuggiti alla V\&V potrebbero emergere.
\section{Qualità del software}
Non c'è un principio unico di qualità, ma ci sono approcci e visioni diverse, lo standard che definisce la qualità del modelli software è dato dall'\href{https://www.iso.org/standard/78175.html}{ISO/IEC 25002}.
Nello standard, il cocnetto di qualità del software è modellizzato tramite:
\begin{itemize}
    \item \textbf{Modello della qualità del prodotto}, composto da 9 caratteristiche relative alla qualità delle proprietà del prodotto. Le caratteristiche e le sottocaratteristiche fanno da riferimento del modello per la qualità dei prodotti da specificare, misurare e valutare.
    \item \textbf{Modello della qualità-in-uso}, composto da 3 caratteristiche che influenzano gli stakeholder quando i prodotti o i sistemi sono usati in uno specifico contesto d'uso.
\end{itemize}

\section{Il modello ISO/IEC 25002}
\subsection{Qualità del software: idoneità}
L'\textbf{idoneità funzionale} è un grado per cui una componente o un sistema dà funzioni che incontrano le necessità confermate e implicate quando vengono usate sotto specifiche condizioni.
\newline
Queste caratteristiche sono composte da 3 sottocaratteristiche:
\begin{itemize}
    \item \textbf{Completezza funzionale}: un grado per cui le funzioni date coprono tutte le task specifiche e gli obiettivi dell'utente.
    \item \textbf{Correttezza funzionale}: grado per cui il prodotto provvede risultati accurati quando usato degli utenti.
    \item \textbf{Appropriatezza funzionale}: grado per cui le funzioni facilitano il completamento di task e obiettivi specifici.
\end{itemize}

\subsection{Affidabilità}
L'\textbf{affidabilità} valuta come il sistema performa sotto condizioni specifiche per un periodo di tempo specifico.
Questa caratteristica è composta dalle seguenti sottocaratteristiche:
\begin{itemize}
    \item \textbf{Impeccabilità}: grado per cui un sistema performa specifiche funzioni senza problemi per una normale operazione.
    \item \textbf{Disponibilità}: grado per cui un sistema è operativo e accessibile quando viene richiesto il suo utilizzo.
    \item \textbf{Tolleranza ai problemi}: grado per cui il sistema opera come deve nonostante la presenza di problemi lato hardware o software.
    \item \textbf{Recuperabilità}.
\end{itemize}

\subsection{Efficienza}
L'\textbf{efficienza} rappresenta il grado con cui un prodotto performa le sue funzioni entro limiti specifici delle risorse, ed è efficiente nell'uso delle risorse.
\newline
Questa caratteristica è composta dalle seguenti sottocaratteristiche:
\begin{itemize}
    \item \textbf{Comportamento nel tempo}: il grado per cui il tempo di risposta e i tassi di rendimento di un prodotto o sistema incontrano i requisiti.
    \item \textbf{Utilizzo delle risorse}: il grado per cui la quantità e i tipi delle risorse usate da un prodotto o sistema incontrano i requisiti.
    \item \textbf{Capacità}: grado per cui i limiti massimi di un prodotto o parametro di sistema incontrano i requisiti.
\end{itemize}
\subsection{Usabilità}
Rappresenta il grado con cui s'interagisce con un prodotto o sistema da parte dell'utente.
\newline
Ha le seguenti sottocaratteristiche:
\begin{itemize}
    \item \textbf{Riconoscibilità}.
    \item \textbf{Apprendibilità}.
    \item \textbf{Operatività}.
    \item \textbf{Protezione dell'utente dagli errori}.
    \item \textbf{Inclusività}.
\end{itemize}

\subsection{Sicurezza}
È il grado per cui un sistema si difende dagli attacchi maliziosi e protegge le informazioni, rinforzando i dati tramite adeguati meccanismi di autorizzazione.
Include le seguenti sottocaratteristiche:
\begin{itemize}
    \item \textbf{Confidenzialità}: grado per cui un sistema assicura che i dati sono accessibili solo a chi ne è autorizzato all'accesso.
    \item \textbf{Integrità}: grado per cui un sistema assicura che il suo stato e i suoi dati sono protetti da modificazioni o rimozioni non autorizzate.
    \item \textbf{Non-ripudio}: grado per cui le azioni o eventi vengono dimostrati di prendere luogo in modo che gli eventi o le azioni possano non essere ripudiate dopo.
    \item \textbf{Responsabilità}: grado per cui le azioni di un'entità possono essere tracciate unicamente da quell'entità.
    \item \textbf{Autenticità}: grado per cui l'identità di un soggetto o risorsa puà essere mostrata a chi l'ha rivendicata.
\end{itemize}

\subsection{Compatibilità}
La \textbf{compatibilità} rappresenta il grado in cui un sistema può scambiare informazioni con altri prodotti, sistemi o componenti e/o svolgere le proprie funzioni richieste condividendo lo stesso ambiente e risorse comuni con altri sistemi. Questa caratteristica è composta dalle seguenti sotto-caratteristiche:
\begin{itemize}
    \item \textbf{Coesistenza} - Grado in cui un prodotto può svolgere le proprie funzioni richieste in modo efficiente mentre condivide un ambiente e risorse comuni con altri prodotti, senza impatti negativi su nessun altro prodotto.
    \item \textbf{Interoperabilità} - Grado in cui un sistema, prodotto o componente può scambiare informazioni con altri prodotti e utilizzare reciprocamente le informazioni che sono state scambiate.
\end{itemize}
\subsection{Manutenibilità}
La \textbf{manutenibilità} rappresenta il grado di efficacia ed efficienza con cui un prodotto o sistema può essere modificato per migliorarlo, correggerlo o adattarlo ai cambiamenti dell'ambiente e ai requisiti. Questa caratteristica è composta dalle seguenti sotto-caratteristiche:
\begin{itemize}
    \item \textbf{Modularità} - Grado in cui un software è composto da componenti discreti, in modo che una modifica a un componente abbia un impatto minimo sugli altri.
    \item \textbf{Riutilizzabilità} - Grado in cui un software o un modulo può essere utilizzato come risorsa in più di un sistema.
    \item \textbf{Modificabilità} - Grado in cui un prodotto o sistema può essere modificato in modo efficace ed efficiente senza introdurre difetti o degradare la qualità.
    \item \textbf{Testabilità} - Grado in cui possono essere stabiliti criteri di test per un sistema e possono essere eseguiti test per determinare se tali criteri sono stati soddisfatti.
\end{itemize}

\subsection{Flessibilità}
La \textbf{flessibilità} è il grado in cui un prodotto può essere adattato ai cambiamenti nei suoi requisiti, nei contesti di utilizzo o nell'ambiente operativo. Questa caratteristica è composta dalle seguenti sotto-caratteristiche:
\begin{itemize}
    \item \textbf{Adattabilità} - Grado in cui un sistema può essere adattato in modo efficace ed efficiente a diversi hardware, software o altri ambienti operativi o di utilizzo.
    \item \textbf{Scalabilità} - Grado in cui un sistema può gestire carichi di lavoro in crescita o in diminuzione o adattare la sua capacità per gestire la variabilità.
    \item \textbf{Installabilità} - Grado di efficacia ed efficienza con cui un prodotto o sistema può essere installato e/o disinstallato con successo.
    \item \textbf{Sostituibilità} - Grado in cui un prodotto può sostituire un altro prodotto software specificato per lo stesso scopo nello stesso ambiente.
\end{itemize}
\subsection{Sicurezza}
La \textbf{sicurezza} rappresenta il grado in cui un prodotto evita uno stato in cui la vita umana, la salute, la proprietà o l'ambiente sono messi in pericolo. Questa caratteristica include, tra l'altro, le seguenti sotto-caratteristiche:
\begin{itemize}
    \item \textbf{Sicurezza in caso di guasto} - Grado in cui un prodotto può automaticamente collocarsi in una modalità operativa sicura, o tornare a una condizione sicura in caso di guasto.
    \item \textbf{Identificazione del rischio} - Grado in cui un prodotto può identificare un corso di eventi o operazioni che possono portare a un rischio inaccettabile.
    \item \textbf{Avviso di pericolo} - Grado in cui un sistema fornisce avvisi di rischi inaccettabili per le operazioni o i controlli interni in modo che possano reagire in tempo sufficiente.
\end{itemize}
\section{Modello della qualità-in-uso}
Modello di qualità in uso, è composto da 3 caratteristiche (ulteriormente suddivise in sotto-caratteristiche) che possono influenzare gli stakeholder quando i prodotti o i sistemi vengono utilizzati in un contesto d'uso specifico. Misura il grado in cui un prodotto o un sistema può essere utilizzato da utenti specifici per soddisfare le loro esigenze al fine di raggiungere obiettivi specifici con efficacia, efficienza, assenza di rischi e soddisfazione in contesti d'uso specifici.
\subsection{Usabilità}
L'\textbf{usabilità} misura l'estensione in cui gli utenti possono raggiungere i loro obiettivi in modo efficiente e soddisfacente utilizzando il sistema. Questa caratteristica è composta dalle seguenti sotto-caratteristiche:
\begin{itemize}
    \item \textbf{Efficacia} - Quanto bene gli utenti possono completare i loro compiti previsti utilizzando il sistema.
    \item \textbf{Efficienza} - Le risorse (ad es., tempo, sforzo) necessarie per raggiungere i compiti.
    \item \textbf{Soddisfazione} - Il comfort dell'utente e l'esperienza positiva durante l'utilizzo del sistema.
\end{itemize}
\subsection{Sicurezza}
La \textbf{sicurezza} valuta la capacità del sistema di prevenire danni o pericoli per le persone, l'ambiente e gli interessi commerciali. Questa caratteristica è composta dalle seguenti sotto-caratteristiche: 
\begin{itemize}
    \item \textbf{Danno Commerciale}: Valuta quanto bene il sistema previene perdite finanziarie o danni all'attività.
    \item \textbf{Salute e Sicurezza dell'Operatore}: Quanto bene il sistema protegge gli utenti dai rischi per la salute o dai pericoli per la sicurezza durante il suo utilizzo.
    \item \textbf{Salute e Sicurezza Pubblica}: Previene rischi o danni per il pubblico generale attraverso l'uso o il funzionamento del sistema. \textbf{Danno Ambientale}: Il sistema dovrebbe evitare o ridurre al minimo gli impatti negativi sull'ambiente.
\end{itemize}
\subsection{Flessibilità}
La \textbf{flessibilità} si riferisce alla capacità del sistema di adattarsi e operare efficacemente in contesti o ambienti diversi. Questa caratteristica è composta dalle seguenti sotto-caratteristiche:
\begin{itemize}
    \item \textbf{Conformità al contesto}: La capacità del sistema di adattarsi ai requisiti e ai vincoli specifici di diversi contesti.
    \item \textbf{Estensibilità del contesto}: Il potenziale del sistema di espandersi o adattarsi a ambienti nuovi o in evoluzione senza modifiche significative.
    \item \textbf{Accessibilità}: Cattura l'efficacia con cui il sistema può essere utilizzato da tutte le persone, comprese quelle con disabilità, in diversi ambienti.
\end{itemize}
\chapter{Lezione 3 - Requisiti d'ingegneria: elicitazione e analisi}
\section{Il ciclo di vita del software}
\subsection{Requisiti software}
I requisiti sono descrizioni di quello che il sistema dovrebbe fare:
\begin{itemize}
    \item I \textbf{servizi} che il sistema dovrebbe dare agli utenti.
    \item I \textbf{vincoli operativi}.
\end{itemize}
Il termine \textbf{«requisito»} è usato in modo incoerente nell'industria del software. In alcuni casi, un requisito è una descrizione astratta e ad alto livello di un servizio che il sistema dovrebbe fornire o di un vincolo sul sistema. 
\begin{itemize}
    \item Questi sono chiamati \textbf{Requisiti Utente} (l'attenzione è sulla prospettiva degli utenti finali)
\end{itemize}
In altri, è una definizione più dettagliata e formale di una funzione del sistema 
\begin{itemize}
    \item Questi sono chiamati \textbf{Requisiti di Sistema} (l'attenzione è sul sistema da costruire)
\end{itemize}
Tale ambiguità è inevitabile, poiché i requisiti possono avere una doppia funzione: 
\begin{itemize}
    \item I \textbf{requisiti degli utenti} possono essere la base per un'offerta per un contratto
    \begin{itemize}
        \item Il cliente può definire requisiti generali degli utenti
        \item Diversi appaltatori possono proporre modi diversi per soddisfare i requisiti degli utenti
    \end{itemize}
    \item I \textbf{requisiti di sistema} possono essere la base per il contratto stesso
    \begin{itemize}
        \item Una volta assegnato un contratto per lo sviluppo del software, l'appaltatore formula un insieme più dettagliato di requisiti di sistema
        \item In modo che il cliente comprenda esattamente cosa farà il software e possa validare la proposta
        \item Una volta accettati e convalidati, i requisiti di sistema possono essere inseriti nel contratto finale e sono vincolanti!
    \end{itemize}
\end{itemize}
L'ingegnerizzazione dei requisiti (RE) è una sottoarea dell'Ingegneria del software che si occupa del processo di definire i requisiti di per un quel-che-sarà-software. L'obiettivo è fornire agli ingegneri del software dei metodi, tecniche e strumenti per capire e documentare quel che un sistema software deve fare.
\subsection{Requisiti funzionale e non-funzionale}
I requisiti sono spesso classificati come funzionali o non funzionali:
\begin{itemize}
    \item \textbf{Requisiti Funzionali}:
    \begin{itemize}
        \item Servizi che il sistema dovrebbe fornire
        \item Come il sistema dovrebbe reagire a determinati input o comportarsi in una data situazione
    \end{itemize}
    \item \textbf{Requisiti Non-Funzionali}:
    \begin{itemize}
        \item Vincoli sui servizi o funzioni offerti dal sistema
        \item Includono vincoli temporali, vincoli di processo o vincoli imposti da standard
        \item Spesso si applicano all'intero sistema piuttosto che a singole caratteristiche o servizi
    \end{itemize}
\end{itemize}
In pratica, la distinzione tra questi due tipi di requisiti non è netta. Considera quanto segue: \textit{solo gli utenti autorizzati dovrebbero poter accedere al sistema}.
\newline
Sembra un requisito non funzionale, tuttavia, quando sviluppiamo in maggiore dettaglio, genera requisiti aggiuntivi che sono chiaramente funzionali, ad esempio, gli utenti devono essere in grado di effettuare il login e autenticarsi.
\newline
\textbf{I requisiti non sono indipendenti e un requisito spesso genera o vincola altri requisiti}
\subsubsection{Requisiti Funzionali}
Quando espressi come \textbf{requisiti funzionali degli utenti}, possono essere scritti in linguaggio naturale, in modo che possano essere compresi da persone non tecniche (ad es.: utenti e manager). 
\newline
Quando espressi come \textbf{requisiti funzionali del sistema}, dovrebbero essere dettagliati, descrivere accuratamente gli input e gli output del sistema e le eccezioni, in modo che gli ingegneri del software sappiano esattamente cosa implementare.
\subsubsection{Requisiti non-funzionali}
I requisiti non-funzionali sono non direttamente collegati ai servizi specifici offerti dal sistema. Tipicamente specifica o vincola le caratteristiche del sistema come un insieme:
\begin{itemize}
    \item I vincoli su come dovrebbero essere implementati.
    \item Specifica altre proprietà.
\end{itemize}
Spesso i requisiti non-funzionale sono molto più critici di quelli funzionali.
\begin{itemize}
    \item Gli utenti possono trovare una strada attorno all'implementazione sub-ottimale di una richiesta funzionale.
    \item ...ma fallendo a incontrare i requisiti non-funzionali che possono essere indice di un sistema instabile.
\end{itemize}
\subsubsection{Tipi di requisiti non-funzionali}
I requisiti non funzionali possono derivare da: 
\begin{itemize}
    \item Caratteristiche richieste del prodotto (\textbf{Requisiti di Prodotto}) 
    \begin{itemize}
        \item La velocità del sistema, la quantità di memoria richiesta, requisiti di usabilità, tassi di fallimento accettabili, ...
    \end{itemize}
    \item Le organizzazioni del cliente e degli sviluppatori (\textbf{Requisiti Organizzativi}) 
    \begin{itemize}
        \item Processi operativi che descrivono come il sistema sarà utilizzato, linguaggi di programmazione richiesti, requisiti che specificano l'ambiente operativo, ...
    \end{itemize}
    \item Fonti esterne (\textbf{Requisiti Esterni}) 
    \begin{itemize}
        \item Cosa deve essere fatto affinché il sistema venga approvato da un regolatore di terze parti, requisiti legali, requisiti etici per garantire che il sistema sia accettabile per i suoi utenti e per il pubblico in generale.
    \end{itemize}
\end{itemize}
\section{Proprietà dei buoni requisiti}
I requisiti dovrebbero essere:
\begin{itemize}
    \item \textbf{Chiari e facili da comprendere} (soprattutto i requisiti dell'utente)
    \item \textbf{Univoci }(l'ambiguità porta a controversie con i clienti)
    \item \textbf{ Completi }
    \item \textbf{Coerenti} (cioè, non dovrebbero confliggere tra loro)
    \item \textbf{Verificabili} (la mancanza di verificabilità porta a controversie con i clienti). Dato un requisito, dovrebbe essere possibile determinare in modo univoco se il sistema soddisfa quel requisito.
\end{itemize}
\subsection{Requisiti di testabilità}
Dovrebbe essere possibile determinare in modo univoco se il sistema soddisfa un requisito. Altrimenti, gli sviluppatori potrebbero sostenere che il requisito è soddisfatto, mentre il cliente potrebbe non essere affatto d'accordo! I Requisiti Funzionali del Sistema dovrebbero essere il più dettagliati possibile e non lasciare spazio all'interpretazione. I Requisiti Non Funzionali del Sistema dovrebbero includere indicatori quantitativi ogni volta che è possibile.
\subsection{Requisiti non-funzionali testabili}
\textbf{Requisiti non molto testabili}:
\begin{itemize}
    \item[a] Il sistema dovrebbe essere affidabile
    \item[b] Il sistema dovrebbe essere semplice da usare
    \item[c] Il sistema dovrebbe essere veloce e reattivo 
\end{itemize}
\textbf{Requisiti più testabili}
\begin{itemize}
    \item[a] Il sistema dovrebbe avere un tempo di attività al 99.9\% al mese.
    \item[b] Gli utenti dovrebbero essere capaci di usare tutte le funzioni di sistema dopo al più due ore di allenamento e il numero medio di errori lato utente dovrebbero non superare i 2 all'ora per l'utilizzo del sistema.
    \item[c] Il tempo di risposta medio del sistema non dovrebbe superare i 100 millisecondi.
\end{itemize}
\section{Il processo di ingegnerizzazione del requisito}
I 3 passaggi chiave sono:
\begin{itemize}
    \item \textbf{Elicitazione e analisi dei requisiti}: scopre i requisiti interagendo con gli stackholder.
    \item \textbf{Specificazione del requisito}: converte i requisiti in una forma standardizzata.
    \item \textbf{Validazione del requisito}: controlla che i requisiti definiscano il sistema che il cliente vuole,
\end{itemize}
\section{Il processo RE}
In pratica, il processo RE non è lineare, le sue attività sono attualmente interlivellate in un processo iterativo con livelli diversi di granularità, passando per i requisiti: \textbf{a livello di business}, \textbf{dell'utente} e \textbf{di sistema}.
\subsection{Elicitazione e analisi dei requisiti}
L'obiettivo di quest'attività è di trovare quel che i cliente vogliono e necessitano ed è considerata la parte più critica, vista la dovuta collaborazione con gli stakeholders.
\newline
Gli stakeholder non sanno cosa vogliono dal software, tranne che in termini molto generali. Non sanno cosa sia fattibile e cosa non lo sia, e potrebbero fare richieste irrealistiche. Gli stakeholder sono esperti nel loro dominio. Esprimono naturalmente i requisiti nei propri termini e gergo, con una conoscenza implicita del loro lavoro. Potrebbero dare per scontati alcuni dettagli, quando in realtà non lo sono. Diversi stakeholder, con requisiti diversi, possono esprimere i loro requisiti in modi diversi. Gli ingegneri dei requisiti devono lavorare attorno a comunanze e conflitti. Fattori politici possono influenzare il processo di elicitation. I manager possono richiedere requisiti specifici perché questi consentirebbero loro di aumentare la loro influenza nell'organizzazione. Gli stakeholder hanno opinioni diverse (e talvolta conflittuali) sull'importanza e la priorità dei requisiti. Se alcuni stakeholder sentono che le loro opinioni non sono state adeguatamente considerate, potrebbero tentare deliberatamente di minare il processo di ingegneria dei requisiti. I requisiti cambieranno durante il processo di elicitation. Nuovi requisiti possono emergere da nuovi stakeholder che non erano stati inizialmente considerati.
\section{Processo dell'elicitazione e analisi dei requisiti}
Ci sono de passaggi da seguire:
\begin{enumerate}
    \item \textbf{Scoprire e capire}: interagire con gli stakeholder per scoprire i requisiti.
    \item \textbf{Classificazione e organizzazione}: i requisiti correlati sono raggruppati e organizzati.
    \item \textbf{Prioritizzazione e negoziazione}: risolve i confliti dei requisiti che sorgono dai conflitti delle necessità dei vari stakeholder.
    \item \textbf{Documentazione}: tiene traccia dei requisiti scoperti per la prossima iterazione del processo di elicitazione.
\end{enumerate}
\section{Importanza della Conoscenza degli Utenti}
La comprensione degli utenti è fondamentale per:
\begin{itemize}
    \item Scoprire i requisiti
    \item Progettare interfacce utente efficaci
\end{itemize}

\subsection{Personas}
\begin{itemize}
    \item \textbf{Definizione}: Archetipi ipotetici che rappresentano utenti reali
    \item \textbf{Scopo}: Promuovere l'empatia e comprendere meglio gli utenti
    \item \textbf{Caratteristiche}:
    \begin{itemize}
        \item Personalizzazione (nome, età, biografia)
        \item Informazioni lavorative
        \item Formazione scolastica
        \item Obiettivi
        \item Punti critici/Frustrazioni
    \end{itemize}
    \item Non sono persone reali ma vengono definiti con rigore
    \item Emergono dal processo di elicitatione dei requisiti
\end{itemize}

\subsection{Casi d'Uso delle Personas}
Utili quando:
\begin{itemize}
    \item Sviluppiamo software per il pubblico generico
    \item Gli end-user includono il pubblico generico
    \item Non abbiamo stakeholder specifici da intervistare
\end{itemize}

\subsection{Storie Utente}
\begin{itemize}
    \item Descrizioni narrative di come il sistema può essere utilizzato
    \item Presentano:
    \begin{itemize}
        \item Azioni degli utenti
        \item Informazioni utilizzate
        \item Output richiesti
    \end{itemize}
    \item Efficaci per comunicare obiettivi generali
    \item Gli stakeholder spesso le descrivono meglio dei requisiti tecnici
\end{itemize}

\section{Mockup a bassa fedeltà}
Questi mockup (o wireframe) sono bozze semplificate delle interfacce del sistema lato utente. Sono modi veloci e affetti dal costo per visualizzare la struttura e lo scorrimento base del sistema. L'interesse principale è la funzionalità e il flusso, non l'estetica.
\newline
I suoi benefici sono:
\begin{itemize}
    \item \textbf{Comprensione facilitata}: aiuta gli stakeholder a capire le funzionalità base del sistema. Dando un'idea chiara dell'idea iniziale.
    \item \textbf{Aiuta a scoprire i nuovi requisiti}: è più semplice ragionare su interfacce concrete che di proposte astratte.
    \item \textbf{Feedback interessannte e interattivo}: promuove l'attenzione da parte degli stakeholder e permette una rapida iterazione e feedback senza un design delineato.
    \item \textbf{Fondamenta per lo sviluppo}: provedde una solida fondamenta per spostarsi verso design ad alta fedeltà e a un'eventuale sviluppo.
\end{itemize}
\chapter{Lezione 4: Diagrammi Use Case}
\section{Specificazione dei requisiti}
\subsection{Requisiti e design}

In linea di principio, i requisiti dovrebbero dichiarare \textit{cosa} il sistema dovrebbe fare e il design dovrebbe descrivere \textit{come} lo fa.
In pratica, requisiti e design sono inseparabili:
\begin{itemize}
    \item I requisiti possono essere strutturati e organizzati sulla base di un'architettura di sistema di alto livello.
    \item Il sistema può inter-operare con altri sistemi che generano requisiti di design.
    \item L'uso di un'architettura specifica per soddisfare requisiti non funzionali può essere esso stesso un requisito di dominio.
\end{itemize}

\subsection{Specifica dei requisiti}

Il processo di scrittura dei requisiti utente e di sistema in un documento di specifica dei requisiti.
\begin{itemize}
    \item I \textbf{requisiti utente} devono essere comprensibili da utenti finali e clienti che non hanno una formazione tecnica.
    \item I \textbf{requisiti di sistema} sono requisiti più dettagliati e possono includere informazioni più tecniche.
    \item I requisiti possono far parte di un contratto per lo sviluppo del sistema. È quindi importante che questi siano il più completi e dettagliati possibile.
\end{itemize}

Sono possibili diversi approcci:
\begin{itemize}
    \item \textbf{Linguaggio Naturale:} esprimere i requisiti come frasi numerate in linguaggio naturale. Ogni frase dovrebbe esprimere un singolo requisito.
    \item \textbf{Linguaggio Naturale Strutturato:} utilizzare un modulo o un template standardizzato.
    \item \textbf{Notazioni e Modelli Semi-Formali:} Diagrammi UML dei Casi d'Uso (Use Case Diagrams) e altri modelli di dominio, tipicamente integrati da annotazioni in linguaggio naturale.
    \item \textbf{Specifica Formale:} Queste notazioni si basano su concetti matematici come macchine a stati finite e infinite, logiche temporali, ecc.
\end{itemize}

Diversi approcci sono utilizzati in diversi domini:
\begin{itemize}
    \item Nell'ingegneria di \textbf{sistemi critici per la sicurezza} (safety-critical systems), è comune utilizzare specifiche formali e linguaggio naturale strutturato.
    \item Nell'ingegneria di un'applicazione che listi le cose da fare (to-do list app), si potrebbe usare il linguaggio naturale non strutturato per esprimere i requisiti.
    \item Nell'ingegneria di un sistema informativo di medie-grandi dimensioni, sfruttare notazioni e modelli semi-formali potrebbe essere un buon compromesso.
\end{itemize}

\section{Specificazioni del linguaggio naturale (NL)}
I requisiti sono scritti come frasi in linguaggio naturale, eventualmente integrate da diagrammi e tabelle.
Questo approccio è utilizzato per scrivere i requisiti perché è espressivo, intuitivo e universale.
\begin{itemize}
    \item Ciò significa che i requisiti possono essere compresi da utenti e clienti.
\end{itemize}

Può esprimere sia requisiti funzionali che non funzionali.
Definire un formato ``standard'' e utilizzarlo per tutti i requisiti.
Utilizzare il linguaggio in modo coerente. Usare \textit{shall} per i requisiti obbligatori, \textit{should} per i requisiti desiderabili.
Utilizzare la formattazione del testo per identificare le parti chiave del requisito.
Evitare l'uso di gergo tecnico informatico.
Includere una spiegazione (razionale) del motivo per cui un requisito è necessario.
\newline
Ci sono dei problemi usando il linguaggio naturale:
\begin{itemize}
    \item \textbf{Mancanza di chiarezza:} È difficile essere precisi senza rendere il documento difficile da leggere.
    \item \textbf{Confusione tra i requisiti:} I requisiti funzionali e non funzionali tendono a essere mescolati.
    \item \textbf{Agglomerazione dei requisiti:} Diversi requisiti distinti possono essere espressi insieme in un'unica affermazione.
\end{itemize}
\section{Diagrammi Use Case}
Gli Use Case (Casi d'Uso) sono un modo per descrivere le interazioni tra utenti e un sistema utilizzando un modello grafico e testo in linguaggio naturale strutturato.
Sono una parte fondamentale del Linguaggio di Modellazione Unificato (UML) e possono essere utilizzati per rappresentare l'insieme dei requisiti funzionali di un sistema.

I casi d'uso identificano:
\begin{itemize}
    \item \textbf{Attori:} Categorie di utenti (non necessariamente umani) del sistema.
    \item \textbf{Casi d'Uso:} Tipi di interazioni (o funzionalità) offerte dal sistema.
\end{itemize}

Informazioni aggiuntive sulle interazioni possono essere fornite come descrizioni testuali (strutturate) o per mezzo di uno o più modelli semi-formali (ad es.: Diagrammi di Sequenza UML o Diagrammi degli Stati).
I casi d'uso hanno le seguenti caratteristiche:
\begin{itemize}
    \item Sono un modo per supportare la comunicazione con il cliente per definire le funzionalità del sistema. Dovrebbero essere il più semplici possibile.
    \item Non definiscono \textit{come} il sistema è implementato, ma \textit{cosa} il sistema dovrebbe fare dal punto di vista degli utenti (il sistema è una scatola nera).
    \item I casi d'uso sono spesso documentati utilizzando un Diagramma dei Casi d'Uso (UCD) di alto livello.
\end{itemize}

\subsection{Attori}
Gli attori sono rappresentati utilizzando figure stilizzate (stick figures).
Rappresentano entità esterne che interagiscono con il \textbf{Sistema in Sviluppo} (SUD, System Under Development).
\begin{itemize}
    \item Classi di utenti (umani).
    \item Altri sistemi.
    \item L'ambiente fisico.
\end{itemize}
Ogni attore ha un nome univoco.
Gli attori sono più granulari (coarse-grained) delle Personas: un singolo attore può essere associato a multiple Personas.
\subsection{Euristica per identificare gli attori}
I casi d'uso sono rappresentati come ellissi denominate.
Corrispondono a funzionalità offerte dal sistema, fornendo un qualche beneficio o utilità agli attori.
I casi d'uso modellano i requisiti funzionali.
Un caso d'uso astrae molti possibili scenari (sequenze di azioni) per una determinata funzionalità.
\begin{itemize}
    \item Uno scenario può essere visto come un'istanza di un caso d'uso.
    \item Un caso d'uso rappresenta una classe di scenari che mirano a utilizzare la stessa funzionalità.
\end{itemize}

Per identificare gli attori, ci si può chiedere:
\begin{itemize}
    \item Quali gruppi di utenti sono supportati dal Sistema in Sviluppo (SUD) nel loro lavoro?
    \item Quali gruppi di utenti eseguono le principali funzionalità offerte dal SUD?
    \item Quali gruppi di utenti svolgono le funzioni secondarie del SUD, come l'amministrazione?
    \item Il SUD interagirà con sistemi o software esterni? Ogni sistema o software esterno con cui il SUD interagisce sarà un attore.
\end{itemize}
Gli attori non corrispondono a una singola entità, ma rappresentano piuttosto una classe di utenti che può avere lo stesso ruolo: un utente può ricoprire diversi ruoli nello stesso sistema.
\subsection{Associazioni}
Oltre ad attori e casi d'uso, i diagrammi dei casi d'uso includono
diversi tipi di relazioni tra di essi.

Un'associazione tra un attore e un caso d'uso indica che l'attore
può eseguire il caso d'uso. Graficamente, è rappresentata come una linea che collega un attore a un caso d'uso.
\begin{figure}[htbp!]
    \centering
    \includegraphics[width=0.5\linewidth]{immagini/Lezione4/1.png}
    \caption{Associazioni}
\end{figure}
\subsection{Attori secondari}
Un caso d'uso può essere associato a più attori.
La semantica è che più attori devono collaborare in qualche modo
per eseguire quel caso d'uso.

I diagrammi dei casi d'uso UML non includono meccanismi per specificare
come diversi attori sono coinvolti in un caso d'uso.

Le modalità di interazione e le diverse responsabilità possono essere
specificate con descrizioni aggiuntive.
\begin{figure}[htbp!]
    \centering
    \includegraphics[width=0.5\linewidth]{immagini/Lezione4/2.png}
    \caption{Cardinalità tra più attori}
\end{figure}
\subsection{Generalizzazioni degli attori}
La generalizzazione tra attori può essere applicata quando un attore è un sotto-tipo di un altro attore.
Stesso concetto e notazione come nei Diagrammi delle Classi UML. Rappresentata graficamente come una freccia con una testa vuota.
L'attore specializzato può eseguire qualsiasi caso d'uso che il genitore può eseguire.
La generalizzazione tra casi d'uso è destinata ad essere utilizzata quando un caso d'uso è una specializzazione di un altro. A differenza delle dipendenze \textless\textless extend\textgreater\textgreater{}, non ci sono punti precisi in cui i casi d'uso specializzati deviano dal genitore. Le specializzazioni possono essere molto diverse rispetto ai casi d'uso parent. La notazione è la notazione UML standard per la specializzazione.
\begin{figure}[htbp!]
    \centering
    \includegraphics[width=0.5\linewidth]{immagini/Lezione4/3.png}
    \caption{Generalizzazione}
\end{figure}
\newpage
\subsection{La relazione \textless\textless include\textgreater\textgreater{}}
La relazione \textless\textless include\textgreater\textgreater{} è destinata ad essere utilizzata quando ci sono parti comuni del comportamento di due o più Casi d'Uso.
Questa parte comune viene quindi estratta in un Caso d'Uso separato, da includere in tutti i Casi d'Uso base che hanno questa parte in comune.
Poiché l'uso principale della relazione \textless\textless include\textgreater\textgreater{} è per il riutilizzo di parti comuni, ciò che rimane in un Caso d'Uso base di solito non è completo di per sé ma dipende dalle parti incluse per essere significativo.
Questo si riflette nella direzione della relazione, che indica che il Caso d'Uso base dipende dall'aggiunta ma non viceversa.
Questa relazione può essere utile per:
\begin{itemize}
    \item Scomporre un'interazione complessa in interazioni più piccole e gestibili.
    \item Fattorizzare sequenze comuni di passaggi tra diversi casi d'uso.
\end{itemize}
\begin{figure}[htbp!]
    \centering
    \includegraphics[width=0.5\linewidth]{immagini/Lezione4/4.png}
    \caption{include}
\end{figure}

\subsection{La relazione \textless\textless extend\textgreater\textgreater{}}
La relazione \textless\textless extend\textgreater\textgreater{} è destinata ad essere utilizzata quando c'è un comportamento aggiuntivo che può essere aggiunto, possibilmente in modo condizionale, al comportamento definito in uno o più Casi d'Uso.
Il Caso d'Uso esteso è definito indipendentemente dal Caso d'Uso che estende ed è significativo indipendentemente da esso.
D'altra parte, il Caso d'Uso che estende tipicamente definisce un comportamento che potrebbe non essere necessariamente significativo da solo.
\begin{figure}[htbp!]
    \centering
    \includegraphics[width=0.5\linewidth]{immagini/Lezione4/5.png}
    \caption{extend}
\end{figure}
\newpage
Esistono anche dei punti d'estensione per cui si stabilisce nel diagramma Use Case dove può essere inserito il comportamento di un'extend, possono essere utili quando lo Use Case può essere esteso in più punti.
\begin{figure}[htbp!]
    \centering
    \includegraphics[width=0.5\linewidth]{immagini/Lezione4/6.png}
    \caption{Extension point}
\end{figure}
\section{Errori da principianti}
I casi d'uso dovrebbero fornire qualche beneficio all'attore, aiutare l'attore a completare il suo lavoro o raggiungere qualche obiettivo.

\begin{itemize}
    \item Tipicamente, i nomi dei Casi d'Uso dovrebbero includere un verbo.
    \item Tipicamente, i nomi degli Attori dovrebbero essere sostantivi.
\end{itemize}

Se due attori sono associati allo stesso caso d'uso (con una cardinalità diversa da zero), significa che i due attori sono coinvolti (e necessitano di collaborare) in ogni istanza (scenario) di quel caso d'uso.\\
\textbf{Non} significa che entrambi gli attori possono eseguire quel caso d'uso in modo indipendente!

\subsection*{Attenzione alle generalizzazioni improprie}
\begin{itemize}
    \item Ogni attore dovrebbe avere i propri casi d'uso.
    \item Gli attori specializzati possono già eseguire tutti i casi d'uso dei loro antenati.
    \item Se gli attori specializzati non hanno alcuni casi d'uso propri, la generalizzazione potrebbe essere inutile, oppure potrebbero mancare alcuni casi d'uso.
\end{itemize}

\subsection*{Note importanti}
\begin{itemize}
    \item La relazione \textless\textless include\textgreater\textgreater{} non è un buon modo per rappresentare relazioni temporali.
    \item I diagrammi dei casi d'uso non dovrebbero diventare troppo complessi e confusi.
    \item Utilizzare le relazioni tra casi d'uso e le generalizzazioni tra attori con moderazione.
    \item La modellazione dovrebbe essere a un sufficiente grado di astrazione.
    \item Bisogna essere ordinati (cercare di evitare linee che si intersecano, ecc.).
    \item Un diagramma complesso è indice di una cattiva analisi.
\end{itemize}
\section{Esercizi}
\includepdf[pages=-]{esercizi/Lezione4.pdf}
\chapter{Lezione 5: Use Case completi}
\section{Specificazioni degli Use Case}
Il Diagramma dei Casi d'Uso (UCD) fornisce una panoramica di alto livello dei requisiti funzionali del sistema. Non è sufficientemente dettagliato per stabilire i requisiti di sistema.
Per ogni Casi d'Uso (UC) nell'UCD è necessaria una specifica dettagliata.
L'obiettivo è specificare ogni aspetto e dettaglio dell'interazione, dal punto di vista dell'Attore.
Ogni possibile scenario e variazione dovrebbe essere descritto.

\subsection{Descrizione testuale di un Use Case}
Una descrizione di un caso d'uso generalmente include:
\begin{enumerate}
    \item Una descrizione di ciò che il sistema e gli utenti si aspettano quando il caso d'uso inizia.
    \item Una descrizione del flusso normale degli eventi nel Casi d'Uso (scenario principale).
    \item Una descrizione di ciò che può causare errori e come i problemi risultanti possono essere gestiti.
    \item Una descrizione dello stato del sistema dopo che il Casi d'Uso è completato.
\end{enumerate}

\subsection{Formati di Use Case}
I casi d'uso possono essere scritti in diversi formati e livelli di formalità:
\begin{itemize}
    \item \textbf{Breve:} Riepilogo conciso di un paragrafo, solitamente dello scenario di successo principale.
    \item \textbf{Informale:} Formato a paragrafi informali. Paragrafi multipli che coprono vari scenari.
    \item \textbf{Descrizione Completa (Fully-dressed):} Tutti i passaggi e le variazioni sono scritti in dettaglio e ci sono sezioni di supporto, come precondizioni e garanzie di successo.
\end{itemize}

Le descrizioni Brevi e Informali possono essere utilizzate nelle fasi iniziali della specifica dei requisiti, per avere una rapida idea del soggetto e dell'ambito.
Le descrizioni Complete possono essere sviluppate successivamente, per servire come base per un contratto e specificare in maggiore dettaglio il comportamento del sistema da sviluppare.
\newpage
\subsection{Descrizioni degli Use Case completi}
Sono stati proposti diversi formati per le descrizioni complete dei casi d'uso.
\begin{figure}[htbp!]
    \centering
    \includegraphics[width=0.5\linewidth]{immagini/Lezione5/1.png}
    \caption{Template di Cockburn 1}
\end{figure}
\begin{figure}[htbp!]
    \centering
    \includegraphics[width=0.5\linewidth]{immagini/Lezione5/2.png}
    \caption{Template di Cockburn 2}
\end{figure}
\subsection{Scenari principali ed estensioni}
Lo scenario principale è la sequenza di azioni che si verifica quando tutto nel caso d'uso procede senza intoppi come previsto.

Tuttavia, possono esserci diversi modi per eseguire un caso d'uso:
\begin{itemize}
    \item Gli utenti possono autenticarsi utilizzando il PIN o uno scanner per impronte digitali.
    \item Potrebbe verificarsi un errore in qualche punto.
\end{itemize}

Quando si definisce il comportamento funzionale del sistema, è importante descrivere anche queste sequenze alternative di azioni che possono verificarsi durante l'esecuzione di un caso d'uso.
\begin{itemize}
    \item Ciò viene fatto utilizzando le \textbf{Estensioni}.
    \item Tipicamente, c'è molto più testo nelle Estensioni che nello Scenario Principale.
\end{itemize}
\includepdf[pages=-, addtotoc={1,section,1,Esempio, L1:1}]{esempi/esempioLezione5.pdf}
\section{Validazione dei requisiti}
La validazione punta a dimostrare che i requisiti definiscono il sistema che il cliente desidera veramente.
I costi degli errori nei requisiti sono elevati, quindi la validazione è molto importante.
Correggere un errore nei requisiti dopo la consegna può costare fino a 100 volte il costo della correzione di un errore di implementazione.

\subsection{Controllo dei requisiti}
Ci sono delle domande da fare per il controllo dei requisiti:
\begin{itemize}
    \item \textbf{Validità:} Il sistema fornisce le funzioni che supportano al meglio le esigenze del cliente?
    \item \textbf{Consistenza:} Ci sono conflitti tra i requisiti?
    \item \textbf{Completezza:} Tutte le funzioni richieste dal cliente sono incluse?
    \item \textbf{Realismo:} I requisiti possono essere implementati considerando il budget e la tecnologia disponibili?
    \item \textbf{Verificabilità:} I requisiti possono essere verificati?
\end{itemize}

\subsection{Tecniche di validazione dei requisiti}
Esistono delle tecniche per validare i requisiti:
\begin{itemize}
    \item \textbf{Revisioni dei requisiti:} Analisi manuale sistematica dei requisiti.
    \item \textbf{Prototipazione:} Utilizzo di un modello eseguibile semplificato del sistema per verificare i requisiti. Oppure prototipazione visiva (ad esempio, utilizzando mockup/wireframe).
    \item \textbf{Generazione di test case:} Sviluppo di test per i requisiti per verificarne la testabilità.
\end{itemize}

Dovrebbero essere effettuate revisioni regolari durante la formulazione della definizione dei requisiti.
Sia il personale del cliente che quello dell'appaltatore dovrebbero essere coinvolti nelle revisioni.
Le revisioni possono essere formali (con documenti completati) o informali. Una buona comunicazione tra sviluppatori, clienti e utenti può risolvere i problemi in una fase iniziale.
\chapter{Lezione 6 e 7: Statecharts}
\section{Statechart}
Conosciuti anche come Macchine a Stati (Comportamentali) UML.
Ampiamente utilizzati per modellare gli aspetti dinamici dei sistemi (specialmente quelli reattivi).
Sistemi che reagiscono a eventi (esterni o interni).
Gli Statechart sono ampiamente utilizzati nell'industria, e non solo per la modellazione.

\subsection{Modellazione con stati e transizioni}
Gli stati rappresentano situazioni in cui vale una condizione invariante.
\begin{itemize}
    \item \textbf{Condizioni statiche:} il sistema è in attesa che qualcosa accada.
    \item \textbf{Condizioni dinamiche:} il sistema sta eseguendo un compito specifico.
\end{itemize}
Le transizioni rappresentano possibili cambiamenti di stato.

\subsection{Regioni, vertici e transizioni}
Uno Statechart UML contiene una regione di primo livello (top-level).
Una regione contiene vertici e transizioni.
I vertici rappresentano gli stati.
Le transizioni sono rappresentate come archi orientati tra due vertici.
Esistono diversi tipi di vertici, con semantiche differenti.

\subsection{Stati (semplici)}
Rappresentano stati del sistema non strutturati.
Raffigurati come un rettangolo con angoli arrotondati.
Un compartimento del nome contiene il nome (opzionale) dello stato, come stringa.

\begin{figure}[htbp!]
    \centering
    \includegraphics[width=0.125\linewidth]{immagini/Lezione6/1.png}
    \caption{Uno stato semplice}
\end{figure}

\subsection{Pseudostati iniziali e stati finali}
Gli pseudostati iniziali sono utilizzati per segnare lo stato predefinito (iniziale).
Una regione può contenere al massimo uno pseudostato iniziale.
Gli stati finali modellano una situazione in cui il calcolo è completato (cioè, il sistema non elaborerà ulteriori eventi).
\begin{figure}[htbp!]
    \centering
    \includegraphics[width=0.125\linewidth]{immagini/Lezione6/2.png}
    \caption{Pseudostato e stato iniziale}
\end{figure}

\subsection{Sintassi delle transizioni}
Le transizioni indicano cambiamenti di stato.
Possono essere decorate con un'etichetta della forma:
\begin{center}
    \textit{trigger [guardia] / azioni}
\end{center}
\begin{itemize}
    \item \textit{trigger} (nell'immagine event() è un trigger) è un elenco di eventi che possono indurre un cambiamento di stato.
    \item \textit{guardia} è una condizione booleana.
    \item \textit{azioni} è un elenco di operazioni da eseguire quando la transizione si attiva.
\end{itemize}
Tutte le parti sopra indicate dell'etichetta sono opzionali.
Sono possibili auto-transizioni (self-transitions).

\begin{figure}[htbp!]
    \centering
    \includegraphics[width=0.25\linewidth]{immagini/Lezione6/3.png}
    \caption{trigger [guardia] / azioni}
\end{figure}
\subsection{Semantica delle transizioni}
Affinché una transizione possa essere attivata:
\begin{itemize}
    \item Devono essere generati eventi corrispondenti a tutti i trigger;
    \item La condizione nella guardia deve essere valutata come VERA.
\end{itemize}
Una transizione spontanea è una transizione senza trigger e senza guardia.
Dopo che una transizione viene attivata, la sua lista associata di azioni viene eseguita.
Se più transizioni sono attivabili, solo una di esse viene effettivamente attivata (in modo non deterministico).
Seguono degli esempi.
\includepdf[pages=-]{esempi/esempio6.pdf}

\subsection{Attività interne degli stati}
Gli stati possono (opzionalmente) contenere una lista di attività interne.
Ogni attività è caratterizzata da un'etichetta che indica quando l'attività deve essere invocata.
Etichette riservate:
\begin{itemize}
    \item \textbf{entry /} attività eseguita all'ingresso dello stato
    \item \textbf{do /} attività eseguita finché il sistema si trova nello stato (dopo il completamento delle attività di entry)
    \item \textbf{exit /} attività eseguita all'uscita dallo stato
\end{itemize}
\begin{figure}[htbp!]
    \centering
    \includegraphics[width=0.25\linewidth]{immagini/Lezione6/4.png}
    \caption{Attività interne}
\end{figure}

\subsection{Stati composti}

Uno stato può contenere:
\begin{itemize}
    \item un compartimento per il nome
    \item un compartimento per le attività interne
    \item una o più regioni interne!
\end{itemize}
Uno stato con regioni interne è uno stato composto.
Gli stati in una regione interna sono chiamati sottostati.
\begin{figure}[htbp!]
    \centering
    \includegraphics[width=0.25\linewidth]{immagini/Lezione6/5.png}
    \caption{Stato composto}
\end{figure}
Permettono ai modellatori di definire una struttura gerarchica.
La regione interna dettaglia il comportamento dello stato a cui appartiene.
Forniscono un modo elegante e conciso per modellare comportamenti complessi (e nascondere la complessità quando non è necessaria).
\begin{figure}[htbp!]
    \centering
    \includegraphics[width=0.25\linewidth]{immagini/Lezione6/6.png}
    \caption{Stato composto 2}
    \label{fig:placeholder}
\end{figure}
\newpage
\subsection{Regioni parallele}

Gli stati composti possono contenere più regioni, rappresentando comportamenti che possono verificarsi in parallelo.
Quando si esce da uno stato composto, tutte le sue regioni vengono terminate.
\begin{figure}[htbp!]
    \centering
    \includegraphics[width=0.5\linewidth]{immagini/Lezione6/7.png}
    \caption{Regioni parallele}
\end{figure}
\subsection{Pseudostati di Shallow History}

Raffigurati come una H (circondata da un cerchio).
Rappresenta lo stato più recentemente attivo di uno stato composto, ma non i sottostati di quello stato!
Solo negli stati composti, e solo uno per regione.
\begin{figure}[htbp!]
    \centering
    \includegraphics[width=0.5\linewidth]{immagini/Lezione6/8.png}
    \caption{Shallow history}
\end{figure}

\subsection{Pseudostati di Deep History}

Raffigurati come una H* (sempre circondata da un cerchio).
Stessa funzione di quelli di shallow history, ma ripristina l'intera configurazione della regione (sottostati dei sottostati inclusi!).
\begin{figure}[htbp!]
    \centering
    \includegraphics[width=0.5\linewidth]{immagini/Lezione6/9.png}
    \caption{Deep History}
\end{figure}
\newpage
\subsection{Pseudostati Fork e Join}
\begin{itemize}
    \item I Fork dividono le transizioni in ingresso in più transizioni che entrano in vertici in regioni ortogonali.
    \item I Join uniscono le transizioni in uscita da vertici in regioni ortogonali in un'unica transizione.
\end{itemize}
\begin{figure}[htbp!]
    \centering
    \includegraphics[width=0.5\linewidth]{immagini/Lezione6/10.png}
    \caption{Fork e Join}
\end{figure}

\subsection{Consigli e trucchi per gli Statechart}

Ogni stato dovrebbe tipicamente avere almeno una transizione in entrata e una in uscita.
I diagrammi sono tipicamente letti dall'alto a sinistra verso il basso a destra, quindi posizionare gli pseudostati iniziali/finali di conseguenza!
Se più stati hanno una condizione di ingresso e/o uscita comune, considerare l'uso di stati composti.
Assicurarsi di non modellare il non-determinismo, a meno che non sia ciò di cui si ha veramente bisogno!
Al massimo uno stato può essere attivo in una regione in qualsiasi momento!

\section{Statecharts nel mondo reale}
\subsection{Model-driven Development}

Passo successivo nella tendenza all'aumento dell'astrazione
Standard de-facto in molti domini del software embedded (es: automotive)
Grazie a strumenti come Simulink, è possibile simulare modelli Statechart, generare automaticamente codice e test, e molto altro (es: metodi formali!)

\noindent\textbf{Vantaggi}
\begin{itemize}
    \item In alcuni domini, tipicamente più conveniente, più veloce e porta a una qualità superiore
    \item Modelli comprensibili da esperti di dominio
    \item I modelli sono documentazione!
    \item Minore dipendenza dalla tecnologia
    \item Minore dipendenza dal personale
\end{itemize}

\noindent\textbf{Svantaggi}
\begin{itemize}
    \item Gli strumenti sono costosi
    \item Non abbastanza flessibili per alcune applications
    \item La generazione di codice è tipicamente supportata per un numero limitato di piattaforme
\end{itemize}

\subsection{Gestione degli stati dell'interfaccia utente con Statecharts}

Gli Statechart possono anche essere utilizzati per «guidare» la logica della GUI
Gli Statechart sono più facili da comprendere (rispetto al codice!)
Il comportamento è disaccoppiato dai componenti GUI
Separare il QUANDO (codificato nello Statechart) dal COSA (cosa dovrebbe accadere, codificato nel componente UI)
Gli Statechart scalano bene con la crescita della complessità

\includepdf[pages=-, addtotoc={1,section,1,Esempio, L1:2}]{esempi/esempio6-1.pdf}
\includepdf[pages=-, addtotoc={1,section,1,Esercizi, L1:3}]{esercizi/lezione6.pdf}
\chapter{Lezione 8: Usabilità e Progettazione Centrata sull'uso umano}

\section{L'ascesa dell'usabilità}
Oggi dobbiamo preoccuparci di costruire sistemi con una buona usabilità
Usabilità: una misura qualitativa della facilità ed efficienza con cui
un essere umano può utilizzare le funzioni e le caratteristiche offerte dal sistema.
Indipendentemente dal tipo di software che stiamo costruendo, una buona usabilità può
fare la differenza tra successo e fallimento, e persino tra vita e morte!

\subsection{Usabilità - App per il grande pubblico}

Se stiamo costruendo una nuova app di social media, un sito web di e-commerce,
o un'altra app per smartphone per fare qualcosa:
\begin{itemize}
    \item Facilità di apprendimento, bassi tassi di errore e soddisfazione soggettiva sono fondamentali
    \item L'uso è discrezionale e (generalmente) la concorrenza è agguerrita
    \item Se gli utenti non riescono a ottenere risultati rapidamente e senza sforzo, rinunceranno e proveranno un fornitore concorrente
\end{itemize}

\subsection{Usabilità - Software professionale}

Per software utilizzato in ambiente professionale
(bancario, assicurativo, gestione della produzione,
prenotazioni, fatturazione utilities, ...)
\begin{itemize}
    \item Il tempo di formazione è un costo, la facilità d'uso è importante
    \item L'internazionalizzazione può essere necessaria
    \item La velocità di esecuzione è importante e l'affaticamento, lo stress e l'esaurimento degli operatori sono preoccupazioni
    \item Ridurre del 10\% il tempo medio di transazione potrebbe significare il 10\% in meno di operatori, il 10\% in meno di postazioni di lavoro, ...
\end{itemize}

\subsection{Usabilità - Sistemi critici per la vita}

Per sistemi critici per la vita come quelli che
controllano il traffico aereo, i reattori nucleari, le utilities energetiche,
i servizi di emergenza, le operazioni militari e le cure cliniche)
\begin{itemize}
    \item I costi elevati dovuti ai lunghi tempi di formazione sono previsti, ma dovrebbero garantire prestazioni rapide e senza errori anche quando gli utenti sono sotto pressione
    \item Errori o ritardi nell'esecuzione possono causare danni gravi!
\end{itemize}

\subsection{Interazione Uomo-macchina (HCI)}
L'HCI è la disciplina che studia come gli esseri umani interagiscono con i computer,
e come progettare interfacce utente efficaci e usabili.

\subsection{Usabilità vs User-friendliness}
Quando i fornitori di computer e software iniziarono a vedere gli utenti come più di
un inconveniente, iniziarono a descrivere i loro sistemi come user friendly
Non è un termine molto buono da usare
\begin{itemize}
    \item \textbf{Inutilmente antropomorfico}: gli utenti hanno bisogno di sistemi che li aiutino a svolgere il loro lavoro e non intralcino. Non hanno bisogno che i sistemi siano amichevoli con loro!
    \item \textbf{Implica che i bisogni degli utenti possano essere descritti lungo una singola dimensione} da sistemi che sono più o meno amichevoli
    \item In pratica, sistemi che sono amichevoli per un utente possono risultare tediosi per altri
\end{itemize}

\section{Definizione di usabilità}
L'usabilità è definita mediante 5 attributi di qualità:
\begin{itemize}
    \item \textbf{Apprendibilità}: Quanto è facile per gli utenti portare a termine compiti basici la prima volta che incontrano il design?
    \item \textbf{Efficienza}: Dopo che gli utenti hanno appreso il design, quanto rapidamente possono eseguire i compiti?
    \item \textbf{Memorabilità}: Quando gli utenti tornano al design dopo un periodo di inutilizzo, quanto facilmente possono riacquistare la padronanza?
    \item \textbf{Errori}: Quanti errori fanno gli utenti, quanto sono gravi questi errori e quanto facilmente possono riprendersi dagli errori?
    \item \textbf{Soddisfazione}: Quanto è piacevole usare il design?
\end{itemize}

\subsection{Apprendibilità}

Probabilmente l'attributo di usabilità più fondamentale
La maggior parte dei sistemi deve essere facile da apprendere
Per alcuni sistemi specializzati, è accettabile che siano difficili da apprendere ma altamente efficienti per utenti esperti
I cosiddetti sistemi "walk-up-and-use" (ad es.: sistemi informativi museali) sono pensati per essere utilizzati una sola volta
Questi sistemi richiedono essenzialmente un tempo di apprendimento zero: gli utenti dovrebbero avere successo la prima volta che li usano!

\subsection{Efficienza d'uso}
L'efficienza si riferisce al livello di performance stabile dell'utente esperto quando la curva di apprendimento si appiattisce
Potrebbero volerci mesi o anni per raggiungere quella fase!
Per misurare l'efficienza:
\begin{itemize}
    \item Decidere una definizione di competenza esperta
    \item Ottenere un campione rappresentativo di utenti con quel livello di competenza
    \item Misurare il tempo che impiegano per completare alcuni compiti tipici
\end{itemize}

\subsection{Memorabilità}
Gli utenti occasionali sono una terza categoria di utenti oltre ai principianti e agli esperti
Gli utenti occasionali utilizzano il sistema in modo intermittente
A differenza dei principianti, non hanno bisogno di apprenderlo da zero, ma hanno bisogno di ricordare come usarlo in base al loro apprendimento precedente
L'uso occasionale tipicamente avviene per:
\begin{itemize}
    \item Software che non fanno parte del lavoro principale di un utente
    \item Software che sono intrinsecamente utilizzati a lunghi intervalli (ad es.: per redigere rapporti annuali)
    \item Software che sono utilizzati solo in circostanze eccezionali
\end{itemize}
Interfacce memorabili sono utili anche per utenti che tornano a utilizzare il sistema dopo essere stati in vacanza, o hanno temporaneamente smesso di usarlo
I miglioramenti nell'apprendibilità rendono un'interfaccia anche facile da ricordare
In principio, tuttavia, l'usabilità del ritorno a un sistema è diversa dall'affrontarlo per la prima volta

\subsection{Errori}
Un errore è qualsiasi azione che non raggiunge l'obiettivo desiderato
Il tasso di errore può essere misurato contando il numero di errori commessi dagli utenti durante l'esecuzione di un compito (come parte di un esperimento per misurare anche altri attributi di usabilità)
\begin{itemize}
    \item Il semplice conteggio degli errori potrebbe essere fuorviante: alcuni errori vengono corretti immediatamente dagli utenti e hanno il solo effetto di rallentarli (riducendo in qualche modo la velocità di transazione)
    \item Altri errori sono più catastrofici per natura: l'utente non se ne accorge, portando a un prodotto di lavoro difettoso; potrebbe essere impossibile riprendersi dall'errore
\end{itemize}
Gli utenti dovrebbero commettere il minor numero possibile di errori quando utilizzano un software
E almeno, dovrebbero commettere pochissimi errori catastrofici, se non nessuno!

\subsection{Soddisfazione soggettiva}
Questo attributo si riferisce a quanto sia piacevole usare il sistema
È particolarmente importante per i sistemi utilizzati su base discrezionale: videogiochi, pittura creativa, ...
Per alcuni di questi sistemi, il loro valore di intrattenimento è più importante della velocità con cui le cose vengono fatte, poiché si potrebbe voler passare un tempo più lungo divertendosi

\subsection{Compromessi nell'usabilità}
Non è sempre possibile massimizzare tutti gli attributi di usabilità simultaneamente
\begin{itemize}
    \item Potrebbero essere necessari compromessi: per evitare errori catastrofici, potremmo progettare un'interfaccia utente meno efficiente, che pone domande extra per assicurarsi che l'utente voglia realmente eseguire una certa azione
    \item In alcuni casi, potremmo ottenere una situazione win-win: l'apprendibilità e l'efficienza d'uso per gli esperti non sono necessariamente in conflitto. Potremmo essere in grado di ottenere il meglio di entrambe le curve di apprendimento, ad esempio includendo acceleratori (es.: scorciatoie da tastiera o hotkeys) nella nostra UI
\end{itemize}

\subsection{Ingegneria dell'usabilità}
Molti progetti di sviluppo software falliscono nel raggiungere i loro obiettivi
Molti di questi fallimenti sono dovuti a scarse comunicazioni tra sviluppatori e clienti e tra sviluppatori e utenti
Ciò si traduce in interfacce utente che costringono gli utenti ad adattare e cambiare il loro comportamento piuttosto che soddisfare le esigenze degli utenti

\section{Progettazione Centrata sull'Uomo (HCD)}
Non è un'attività una tantum in cui l'interfaccia utente viene sistemata prima del rilascio
Un insieme di attività che idealmente si svolgono durante l'intero Ciclo di Vita del Software
\href{https://www.iso.org/standard/77520.html}{ISO 9241-210} definisce la Progettazione Centrata sull'Uomo (HCD)
\begin{itemize}
    \item Gli sviluppatori devono mantenere una prospettiva centrata sull'uomo
    \item Gli utenti devono svolgere un ruolo centrale durante l'intero ciclo di vita
    \item Complementare alle metodologie di progettazione esistenti
    \item Fornisce una prospettiva centrata sull'uomo che può essere integrata in diversi processi di progettazione e sviluppo
\end{itemize}

\subsection{Principi HCD}
La progettazione si basa sulla comprensione esplicita di utenti, compiti e ambienti
Gli utenti sono coinvolti il più possibile nella progettazione e nello sviluppo
La progettazione è guidata e raffinata dalla valutazione centrata sull'utente
Il processo può essere iterativo, se necessario
La progettazione affronta l'intera esperienza utente
Il team di progettazione dovrebbe includere competenze e prospettive multidisciplinari

\subsection{HCD: Comprendere il Contesto e gli Utenti}
Contesto: quali sono i tipi di utilizzo del sistema?
\begin{itemize}
    \item Sistema critico per la vita?
    \item Industriale? Commerciale? Militare? Scientifico?
    \item Intrattenimento?
\end{itemize}
In quale mercato compete il sistema?
\begin{itemize}
    \item Progetto di sviluppo software personalizzato?
    \item Sistema per aziende?
    \item App per il grande pubblico?
\end{itemize}

Utenti: conosci i tuoi utenti (come abbiamo fatto nell'Ingegneria dei Requisiti)!
Personas e scenari (mantenere sempre presenti i bisogni degli utenti)
Ma è necessario considerare anche:
\begin{itemize}
    \item Attributi fisici (età, genere, dimensioni, portata, angoli visivi, ecc.)
    \item Abilità percettive (udito, vista, sensibilità al calore...)
    \item Abilità cognitive (capacità di memoria, livello di lettura, formazione musicale, matematica...)
    \item Posti di lavoro fisici (altezza tavolo, livelli sonori, illuminazione, versione software...)
    \item Tratti di personalità e sociali (preferenze, antipatie, pazienza...)
    \item Diversità culturale e internazionale (lingue, flusso delle finestre di dialogo, simboli...)
    \item Popolazioni speciali, (dis)abilità
\end{itemize}

\subsection{HCD: Progettare per Soddisfare i Requisiti}
Una progettazione appropriata del sistema si basa su una chiara comprensione del contesto e degli utenti!
La produzione di soluzioni di progettazione dovrebbe includere le seguenti sotto-attività:
\begin{itemize}
    \item[a)] progettazione di compiti utente, interazione utente-sistema e interfaccia utente per soddisfare i requisiti utente, tenendo in considerazione l'intera esperienza utente;
    \item[b)] rendere le soluzioni di progettazione più concrete (ad es.: prototipi o mock-up);
    \item[c)] migliorare le soluzioni di progettazione basandosi su valutazioni centrate sull'utente e feedback;
    \item[d)] comunicare le soluzioni di progettazione a coloro che sono responsabili della loro implementazione.
\end{itemize}

\subsection{HCD: Progettare per Soddisfare i Requisiti}
Progettare l'interazione utente-sistema implica decidere come gli utenti svolgeranno i compiti con il sistema piuttosto che descrivere come appare il sistema.
La progettazione dell'interazione dovrebbe includere:
\begin{itemize}
    \item prendere decisioni di alto livello (ad es. concept di progettazione iniziale, risultati essenziali);
    \item identificare compiti e sotto-compiti;
    \item allocare compiti e sotto-compiti all'utente e ad altre parti del sistema;
    \item Es.: il sistema tiene traccia dell'ID di login e ricorda agli utenti, ma gli utenti ricordano la password;
    \item identificare gli oggetti di interazione necessari per il completamento dei compiti;
    \item progettare la sequenza e la tempistica (dinamica) dell'interazione;
    \item progettare l'interfaccia utente per consentire un accesso efficiente agli oggetti di interazione.
\end{itemize}

\subsection{HCI: Valutare la Progettazione}
La valutazione centrata sull'utente (cioè, la valutazione basata sulla prospettiva dell'utente) è un'attività richiesta nell'HCD:
\begin{itemize}
    \item Test basati sull'utente (ad es.: coinvolgendo utenti reali o rappresentativi)
    \item Approccio basato sull'ispezione (verifica di linee guida o requisiti)
\end{itemize}

\subsection{Progettazione Centrata sull'Uomo e SDLC}
L'HCD non richiede alcun processo di progettazione particolare
È complementare alle metodologie di sviluppo esistenti
Ciascuna attività può essere integrata (in misura minore o maggiore) in qualsiasi fase dello sviluppo di un sistema
Ad esempio, l'HCD potrebbe essere applicata nella fase di Ingegneria dei Requisiti in un modello di processo a cascata
\includepdf[pages=-, addtotoc={1,subsection,1,Ancora sul processo di design dell'UI, L1:4,
2, section, 1, Design come scelta, L1:5,
4, subsection, 1, Importanza della critica e dei feedback, L1:6
}]{esempi/lezione8.pdf}
\includepdf[pages=-, addtotoc={
1, chapter, 1, Lezione 9: Software Design: System Design, L2:1,
1, section, 1, Progettazione di un sistema, L2:2,
2, subsection, 1, Scopo del System Design, L2:3,
5, section, 1, System Design, L2:4,
6, subsection, 1, Attività di system design, L2:5,
8, subsection, 1, Identificare gli obiettivi qualitativi del sistema, L2:6,
9, subsection, 1, Criteri di design, L2:7,
10, subsection, 1, Compromessi di design, L2:8,
14, section, 1, Concetti di base di buon design, L2:9,
15, subsection, 1, Anticipare, L2:10,
16, subsection, 1, Decomposizione, L2:11,
18, subsection, 1, Modellazione di sottosistemi, L2:12,
19, subsection, 1, Servizi e interfacce di sottosistemi, L2:13,
20, subsection, 1, Esempio, L2:14,
22, subsection, 1, Osservazioni, L2:15,
23, subsection, 1, Indipendenza funzionale, L2:16
}]{capitoli/lezione9.pdf}
\includepdf[pages=-, addtotoc={
1, chapter, 1, Lezione 11: Regole di buon Software Design - La coesione, L3:1,
1, section, 1, Coesione, L3:2,
2, subsection, 1, Granularità della coesione, L3:3,
3, subsection, 1, Single Responsibility Principle, L3:4,
4, subsection, 1, Esempio, L3:5,
8, subsection, 1, Benefici della coesione, L3:6,
11, section, 1, Accoppiamento, L3:7,
12, subsection, 1, Basso accoppiamento, L3:8,
14, subsection, 1, Tipi di accoppiamento in O-O, L3:9,
15, subsection, 1, Esempio, L3:10,
16, subsection, 1, Spiegazione dell'esempio, L3:11,
17, section, 1, Pensare alle interfacce, L3:12,
18, subsection, 1, Esempio di paperboy, L3:13,
20, subsection, 1, Problemi col codice, L3:14,
22, subsection, 1, Migliorie al codice originale, L3:15,
24, subsection, 1, Perché è meglio?, L3:16}]{capitoli/lezione11.pdf}
\includepdf[page=1, addtotoc={
1, subsection, 1, Quali sono gli svantaggi?, L3:17
}]{capitoli/lezione12.pdf}
\includepdf[pages={2-10}, addtotoc={
2, chapter, 1, Lezione 12: Regole di buon Software Design - La legge di Demetra, L4:1,
2, section, 1, La legge di Demetra, L4:2,
4, subsection, 1, Violazioni alla legge, L4:3,
5, subsection, 1, Eliminare il codice di navigazione, L4:4,
6, subsection, 1, Design guidato dalla responsabilità, L4:5,
7, section, 1, Le carte CRC, L4:6,
9, subsection, 1, Conseguenze, L4:7,
10, subsection, 1, Localizzare le modifiche, L4:8
}]{capitoli/lezione12.pdf}
\chapter{Lezione 10: Il design delle cose di tutti i giorni e la natura delle interazioni giornaliere}

\section{Le interazioni}
Alcune interazioni sono più fluide di altre.
Spesso, le difficoltà nell'uso di un sistema non derivano
da complessità profonde e sottili.
Falliamo nell'usare molti oggetti quotidiani!
Quindi, cosa rende difficile un'interazione?
Per rispondere, dobbiamo capire cosa accade
quando qualcuno fa (o prova a fare) qualcosa.
Don Norman ha formulato una teoria per spiegare le
fasi dell'azione.

\subsection{La Struttura dell'Azione: Il Ciclo di Azione}

Per realizzare qualcosa, è
necessario partire da un'idea di
ciò che si vuole ottenere
(obiettivo).
Poi, bisogna agire sul mondo
esterno, muoversi e interagire
con qualcuno o qualcosa (esecuzione).
Infine, verifichiamo se il nostro
obiettivo è stato effettivamente raggiunto
(valutazione).

\subsection{Fasi di Esecuzione}

Per portare alle azioni, gli obiettivi devono
essere trasformati in una dichiarazione specifica
di ciò che deve essere fatto
(intenzione).
Le intenzioni devono essere tradotte in
una sequenza di azioni da eseguire
per soddisfare l'intenzione.
La sequenza di azioni deve essere
fisicamente eseguita, cioè
realizzata nel mondo.
La valutazione inizia con la nostra
percezione del mondo.
La percezione deve essere
interpretata secondo le nostre
aspettative.
Quindi, l'interpretazione viene
valutata rispetto alle nostre
intenzioni e al nostro obiettivo.

\subsection{Sette Fasi dell'Azione: Esempio}

Stiamo leggendo un libro e la stanza sta diventando troppo buia per leggere.
\begin{itemize}
\item Stabilire l'obiettivo: Aumentare la luce nella stanza
\item Formare l'intenzione: Accendere la lampada
\item Specificare la sequenza di azioni: Camminare verso la lampada, raggiungere l'interruttore, azionare l'interruttore
\item Eseguire la sequenza di azioni: [camminare, raggiungere, azionare]
\item Percepire lo stato del sistema: [sentire il suono ``click'', vedere la luce dalla lampada]
\item Interpretare lo stato del sistema: L'interruttore ha cambiato posizione. La lampada emette luce. La lampada sembra funzionare
\item Valutare lo stato del sistema rispetto agli obiettivi e alle intenzioni: Il livello di luce è aumentato [obiettivo soddisfatto]
\end{itemize}

\section{Le Sette Fasi dell'Azione di Don Norman}
Le sette fasi dell'azione sono un modello approssimativo.
Il processo può iniziare da qualsiasi fase.
\begin{itemize}
\item Le persone non sempre si comportano come organismi logici
\item Gli obiettivi possono essere confusi, mal definiti e vaghi
\item Possiamo rispondere agli eventi del mondo (comportamento guidato dagli eventi)
\item Alcune azioni sono opportunistiche piuttosto che pianificate. Le eseguiamo se si presenta l'opportunità
\end{itemize}

\subsection{Sette Fasi dell'Azione: Alternative}
Notare che un determinato obiettivo può essere soddisfatto utilizzando diverse intenzioni e
diverse sequenze di azioni!
Se qualcuno entrasse nella stanza e passasse vicino alla lampada, potremmo modificare
la nostra intenzione da premere l'interruttore a chiedere all'altra persona di farlo
per noi.

Le difficoltà spesso risiedono interamente nel
derivare le relazioni tra
intenzioni e interpretazioni mentali
e azioni e stati fisici.
Norman identifica due "golfi" (divari) che
separano gli stati mentali da quelli fisici
\begin{itemize}
\item Golfo dell'Esecuzione e Golfo della Valutazione
\item Ogni golfo riflette un aspetto della
distanza tra la rappresentazione mentale
dell'utente e i componenti fisici e gli stati
dell'ambiente
\end{itemize}

\section{Principi di Design per aiutare a colmare i golfi}
\subsection{Affordances}
Le affordances forniscono indizi forti sul
funzionamento delle cose.
I pulsanti sono fatti per essere premuti. Le manopole sono fatte
per essere girate. Le fessure sono fatte per inserirvi oggetti.
Le palle sono fatte per essere lanciate o fatte rimbalzare.
Quando si sfruttano le affordances, gli
utenti sanno cosa fare semplicemente guardando: non
serve alcuna immagine, etichetta o istruzione. Le cose complesse potrebbero richiedere spiegazioni, ma
le cose semplici non dovrebbero!
Le false affordances sembrano offrire una particolare capacità, ma in realtà
ne offrono una diversa (o nessuna)!
Nelle UI: ad esempio: se qualcosa sembra un pulsante ma non è cliccabile.
Le affordances nascoste si verificano quando gli indizi che indicano la funzione di un elemento
non sono evidenti e potrebbero non essere visualizzati fino a quando l'azione
non viene intrapresa.

\subsection{Vincoli}
Il modo più semplice per assicurarsi che qualcosa sia facile da usare, con pochi errori,
è rendere impossibile fare diversamente limitando le scelte dell'utente.

\subsection{Feedback}
Ogni azione con effetti collaterali rilevanti
dovrebbe essere esplicitamente confermata dal
sistema.
Il feedback dovrebbe essere immediato e
informativo. Preferibilmente non distrattivo e
discreto.

\subsection{Coerenza}
Le interfacce dovrebbero essere coerenti in modi significativi
\begin{itemize}
\item All'interno dell'applicazione stessa (coerenza interna)
\item Con altre applicazioni esterne (coerenza esterna)
\end{itemize}
La coerenza aiuta gli utenti a colmare i golfi della valutazione e dell'esecuzione.
Ad esempio: se tutte le azioni sono confermate tramite un messaggio toast nell'angolo in alto a destra
(come nell'app Lista delle cose da fare che abbiamo visto alcune diapositive fa), è più facile per gli utenti
capire in quale stato si trova il sistema. Se i messaggi di conferma fossero diversi per
ogni azione...

\subsection{Metafore}
Possono essere utili nelle UI per suggerire un modello mentale esistente e sfruttare
conoscenze specifiche che gli utenti già possiedono in domini diversi
\begin{itemize}
\item Carrozze senza cavalli, telefoni senza fili...
\item Metafora del desktop: Non è un tentativo di simulare una scrivania reale, ma mira a sfruttare la conoscenza che gli utenti hanno di file, documenti, cartelle, cestini, ...
\end{itemize}

\subsection{Mappature}
Una mappatura è una corrispondenza tra un'interfaccia e l'azione corrispondente nel
mondo reale.
Mappature efficaci (naturali) possono minimizzare
i passaggi cognitivi per trasformare un'azione in
effetto, o accelerare il processo di
trasformazione della percezione in
comprensione.
Le mappature naturali possono anche ridurre il carico
sulla memoria.

\section{Linee Guida di Norman per colmare i golfi}
\begin{itemize}
\item Visibilità. Guardando, l'utente può comprendere lo stato del dispositivo e le
alternative per l'azione.
\item Un buon modello concettuale. Il designer fornisce un buon modello
concettuale per l'utente, con coerenza nella presentazione delle operazioni
e dei risultati e un'immagine del sistema coerente e consistente.
\item Buone mappature. È possibile determinare le relazioni tra
azioni e risultati, tra i controlli e i loro effetti, e
tra lo stato del sistema e ciò che è visibile.
\item Feedback. L'utente riceve un feedback completo e continuo sui
risultati delle azioni.
\end{itemize}

\subsection{Caratteristiche di un Buon Design}
Ha affordances (rende le operazioni visibili)
Offre mappature ovvie (rende evidente la relazione tra l'azione
effettiva del dispositivo e l'azione dell'utente)
Fornisce feedback sull'azione dell'utente
Fornisce un buon modello mentale del comportamento sottostante del sistema
\begin{itemize}
\item Un modello mentale è la rappresentazione interna che un utente ha di come un sistema o
un'interfaccia funziona.
\item Un buon modello mentale consente agli utenti di prevedere come si comporterà l'interfaccia
e li aiuta a interagire efficacemente con essa.
\end{itemize}
Fornisce vincoli (per prevenire errori)
\chapter{Lezione 13: Modelli e teorie in HCI}
\subsection{Cos'è un modello?}
Un modello è una semplificazione (astrazione) della realtà
\begin{itemize}
\item Una mappatura perfetta della realtà non è un modello (e non è utile!)
\item Precisione vs generalità
\end{itemize}
«Tutti i modelli sono sbagliati, ma alcuni sono utili»
I modelli ci permettono di:
\begin{itemize}
\item Rappresentare e ragionare su (aspetti di) fenomeni di interesse
\item Anticipare (prevedere) risultati
\end{itemize}

\section{Modelli e teorie HCI}
Nell'HCI, i modelli e le teorie mirano a spiegare come gli esseri umani interagiscono con
i computer.
I modelli e le teorie HCI possono essere classificati come:
\begin{itemize}
\item \textbf{Descrittivi}: mirano a sviluppare una terminologia coerente e tassonomie utili
\item \textbf{Esplicativi}: descrivono sequenze di eventi, possibilmente con relazioni causali. Es.: le sette fasi dell'azione di Norman
\item \textbf{Predittivi}: mirano a consentire il confronto di alternative di design basate su
previsioni numeriche di velocità o errori
\item \textbf{Prescrittivi}: offrono linee guida ai designer per prendere decisioni
\end{itemize}

\subsection{Il modello a tre stati dell'input grafico}
Proposto da Will Buxton nel 1990 [1]
Descrive l'input grafico con dispositivi di puntamento
Diverse tecnologie (mouse, trackpad, tablet con stilo, ecc...)
\begin{figure}[htbp]
\centering
\includegraphics[width=0.5\textwidth]{immagini/lezione13/1.png}
\caption{Descrizione di Buxton}
\end{figure}

\begin{figure}[htbp]
\centering
\includegraphics[width=0.5\textwidth]{immagini/lezione13/2.png}
\caption{Operazioni di trascinamento con un mouse (a sinistra) e con un touchpad lift-and-tap (a destra) (tratto da [1])}
\end{figure}

\subsection{Il modello di Guiard dell'abilità bimanuale}
Molte interazioni sono asimmetriche rispetto alla mano sinistra/destra
Il modello di Guiard descrive ruoli e azioni delle mani preferita/non preferita
Non preferita:
\begin{itemize}
\item Guida la mano preferita
\item Definisce il quadro di riferimento spaziale per la mano preferita
\item Esegue movimenti grossolani
\end{itemize}
Preferita:
\begin{itemize}
\item Segue la mano non preferita
\item Lavora all'interno del quadro di riferimento stabilito dalla mano non preferita
\item Esegue movimenti fini
\end{itemize}

\begin{figure}[htbp]
\centering
\includegraphics[width=0.5\textwidth]{immagini/lezione13/3.png}
\caption{Interazione a due mani. (Schizzo di Shawn Zhang)}
\end{figure}

L'artista acquisisce il modello con la mano sinistra (la mano non preferita guida).
Il modello viene manipolato sullo spazio di lavoro (movimento grossolano, definisce il quadro di riferimento).
Lo stilo viene impugnato nella mano destra (la mano preferita segue) e portato in prossimità del modello (lavora all'interno del quadro di riferimento stabilito dalla mano non preferita).
Ha luogo lo schizzo (la mano preferita esegue movimenti precisi).
\section{Il Modello del Processore Umano (MHP)}
È un modello predittivo a priori, può fornire approssimazioni delle azioni dell'utente prima che utenti reali siano coinvolti nel processo di test (e prima che l'interfaccia utente sia anche solo implementata!)
Un essere umano è modellato da un insieme di memorie e processori che
funzionano secondo un insieme di principi
Modello discreto e sequenziale
Processori percettivi, cognitivi e
motori
Diversi tipi di memoria
Parametri del modello:
\begin{itemize}
\item Tempi di ciclo del processore: $\tau$
\item Tempo di decadimento della memoria: $\delta$
\item Capacità della memoria: $\mu$
\end{itemize}

\begin{figure}[H]
\centering
\includegraphics[width=0.5\textwidth]{immagini/lezione13/4.png}
\caption{Parametri del modello}
\end{figure}

\subsection{MPH: Memorie}
\textbf{Memoria di Lavoro} (WM) è un sottoinsieme di elementi «attivati» (chunk)
dalla \textbf{Memoria a Lungo Termine} (LTM)
\begin{itemize}
\item I chunk possono essere composti da unità più piccole come lettere in una parola
\item Un chunk potrebbe anche consistere in diverse parole, come in una frase ben nota
\end{itemize}
$\mu_{LTM} = \infty$ e $\delta_{LTM} = \infty$
$\mu_{WM} = 7 \pm 2$ chunk
$\delta_{WM} = 7 \pm 2$ s
\begin{itemize}
\item Il tempo di decadimento per WM varia ampiamente in base al numero di chunk memorizzati
\item $\delta_{WM}$ 1 chunk = $73 \pm 53$ s
\item $\delta_{WM}$ 3 chunk = $7 \pm 2$ s
\end{itemize}

\subsection{Percezione di stimoli congruenti e incongruenti}
I tempi di percezione sono influenzati anche dalla natura degli stimoli
Per esempio, c'è un ritardo significativo nel tempo di reazione (effetto Stroop)
tra stimoli congruenti e incongruenti.

\includepdf[pages=-, addtotoc={1, subsection, 1, Perché usare l'MPH?, L5:1, 2, subsection, 1, Esempio, L5:2}]{esempi/lezione13.pdf}

\section{Modello GOMS}
Goals (Obiettivi), Operators (Operatori), Methods (Metodi), Selection rules (Regole di selezione)
Assunzioni:
\begin{itemize}
\item L'interazione con un sistema è risoluzione di problemi
\item Scomporre l'interazione in sottoproblemi
\item Determinare gli obiettivi per «affrontare» il problema
\item Specificare la sequenza di operazioni utilizzate per raggiungere gli obiettivi
\item I valori di tempo possono essere assegnati a ciascuna operazione
\end{itemize}

\begin{enumerate}
\item[a] Goals (Obiettivi): Ciò che l'utente vuole ottenere (es., "avviare la funzione di utilità corretta").
\item[b] Methods (Metodi): Possibili sequenze alternative di operatori utilizzate per raggiungere l'obiettivo.
\item[c] Selection rules (Regole di selezione): Criteri per scegliere tra diversi metodi.
\item[d] Operators (Operatori): Azioni di base eseguite dall'utente (es., "spostare il mouse", "fare clic", "controllare l'impostazione").
\end{enumerate}

\subsection{GOMS: Esempio}
Obiettivo: eliminare una parola in un editor di documenti
Regola di selezione: se il cursore è alla fine della parola da eliminare, usa il Metodo A, altrimenti
usa il Metodo B

\begin{enumerate}
\item Metodo A:
\begin{itemize}
\item Premere il tasto «backspace».
\item Controllare se la parola è stata eliminata e tornare all'operazione precedente se necessario.
\end{itemize}
\item Metodo B:
\begin{itemize}
\item Spostare il cursore del mouse sulla parola.
\item Eseguire un doppio clic.
\item Premere il tasto «backspace».
\end{itemize}
\end{enumerate}

\subsection{Keyboard Level Model (KLM)}
KLM è una delle varianti più semplici di GOMS
Si concentra sul comportamento osservabile: Tasti, movimenti del mouse, ...
Assume prestazioni prive di errori
Operatori comuni e i tempi tipici corrispondenti:
\begin{itemize}
\item K (Keystroke - Pressione tasto): 0,2 secondi (200 ms)
\item P (Pointing with Mouse - Puntamento con il mouse): 1,1 secondi (1100 ms)
\item B (Pressing/holding/releasing mouse button - Premere/tenere/rilasciare il pulsante del mouse): 0,1 secondi (100 ms)
\item H (Homing Hands - Posizionamento delle mani): 0,4 secondi (400 ms)
\item M (Mental Preparation - Preparazione mentale): 1,2 secondi (1200 ms)
\item R (System Response - Risposta del sistema): Variabile; tipicamente intorno a 0,1 secondi (100 ms)
\end{itemize}

\includepdf[pages=-, addtotoc={1, subsection, 1, Esempio KLM: rimozione di un file, L5:3}]{esempi/lezione13-1.pdf}

\section{La Legge di Potenza della Pratica}
Allen Newell (scienziato cognitivo) negli anni '80 analizzò i tempi di reazione
per una varietà di compiti in esperimenti di apprendimento
Notò che le curve di apprendimento ottenute in questi studi hanno una
forma molto simile: quella di una \textbf{legge di potenza}
\begin{itemize}
\item Il tempo necessario per completare un compito dopo n prove ($T_n$) è vicino al
tempo necessario per completare quel compito la prima volta ($T_1$) moltiplicato per $n^{-a}$
$T_n \approx T_1 \cdot n^{-a}$
\item $a$ è un parametro compreso tra 0,2 e 0,6 (generalmente $\sim$0,4)
\end{itemize}

Un utente ha impiegato 5 secondi per eseguire un
determinato compito la prima volta che è stato
esposto alla nuova interfaccia utente che abbiamo sviluppato
Quante ripetizioni sarebbero necessarie a quell'utente
per essere in grado di eseguire il
compito in 2 secondi o meno?
\begin{itemize}
\item Possiamo calcolare una stima con la legge di potenza della pratica: $T_n \approx T_1 \cdot n^{-a}$
\item Risolviamo per $n$, assumendo $a = 0,4$
\item $2 s \leq 5 s \cdot n^{-0,4}$
\item Per $n = 10$, otteniamo che $T_n \approx 1,99 s$
\end{itemize}

\section{Legge di Hick}
La legge di Hick descrive il tempo necessario a una persona per prendere una decisione
tra un insieme di possibili scelte.
La legge di Hick afferma che il tempo $T$ necessario per raggiungere una decisione aumenta
logaritmicamente con il numero di scelte.
Nel caso di alternative ugualmente probabili:
$T = a + b \cdot \log_2(n + 1)$
\begin{itemize}
\item $n$ è il numero di scelte
\item $a$ e $b$ sono parametri che dipendono dalle condizioni del contesto (es.: il modo
in cui le scelte sono presentate, la familiarità dell'utente,...)
\end{itemize}

\subsection{Applicazione della Legge di Hick}
Quale modo è più veloce per selezionare tra 64 opzioni?
\begin{itemize}
\item Menu a un livello 1x64
$T = a + b \cdot \log_2 64 = a + 6b$
\item Menu a due livelli 4x16
$T = a + b \cdot \log_2 4 + a + b \cdot \log_2 16 = 2a + 6b$
\item Menu a due livelli 8x8
$T = 2 \cdot (a + b \cdot \log_2 8) = 2a + 6b$
\item Menu a tre livelli 4x4x4
$T = 3 \cdot (a + b \cdot \log_2 4) = 3a + 6b$
\item Menu a sei livelli 2x2x2x2x2x2
$T = 6 \cdot (a + b \cdot \log_2 2) = 6a + 6b$
\end{itemize}

\section{Legge di Fitts}
Modella il tempo per acquisire obiettivi nei movimenti mirati
\begin{itemize}
\item Raggiungere un controllo in una cabina di pilotaggio
\item Muoversi attraverso un cruscotto
\item Estrarre un elemento difettoso dal nastro trasportatore
\item Fare clic su icone utilizzando un mouse
\end{itemize}

\subsection{Legge di Fitts – Indice di Difficoltà (ID)}
L'indice di difficoltà di un compito di acquisizione di un obiettivo è definito come
$ID = \log_2(A/W + 1)$
\begin{itemize}
\item $A$ è l'Ampiezza del movimento (distanza dall'inizio all'obiettivo)
\item $W$ è la Larghezza dell'obiettivo (variabilità ammissibile)
\end{itemize}

\begin{figure}[htbp!]
\centering
\includegraphics[width=0.5\textwidth]{immagini/lezione13/5.png}
\caption{Ampiezza e Larghezza}
\end{figure}

\subsection{Legge di Fitts – Tempi di Movimento}
I Tempi di Movimento (MT) dipendono dall'Indice di Difficoltà ID
$MT = a+b \cdot ID = a+b \cdot \log_2(A/W + 1)$
I tempi di movimento dipendono anche dal sistema, dispositivo di puntamento, utente...
Possono essere adattati a casi specifici con parametri non negativi $a$ e $b$
È l'equazione di una linea retta ($y = mx + c$), dove $b$ è il gradiente
MT aumenta linearmente con l'ID

\subsection{Legge di Fitts: Applicazioni}
Se dobbiamo ridurre il tempo necessario per eseguire un'azione di ricerca di un obiettivo
O riduciamo l'Ampiezza del movimento (avvicinare l'obiettivo)
O aumentiamo la Larghezza dell'obiettivo
O potremmo lavorare su $a$ e $b$

\begin{figure}[htbp]
\centering
\includegraphics[width=0.5\textwidth]{immagini/lezione13/6.png}
\caption{Variazioni di ID}
\end{figure}

\includepdf[pages=-, addtotoc={6, section, 1, Letture e referenze, L5:4}]{esempi/lezione13-2.pdf}
\chapter{Lezione 14: Bad Smell e refactoring}
\section{Bad Smells}
\subsection{Code Smells e Refactoring}
I "Code smells" (odori del codice) sono indicazioni che ci sono problemi di design nel sistema.
Le tecniche di refactoring correggono i code smells.
La maggior parte delle tecniche di refactoring sono abbastanza semplici, e spesso c'è un
ottimo supporto negli IDE (attualmente).
Sia nel caso dei code smells che delle tecniche di refactoring, ne vengono "scoperti" di nuovi
continuamente, quindi la lista dei nomi è piuttosto lunga: esamineremo un
piccolo sottoinsieme di ciascuno.
Link utili:
\begin{itemize}
\item \href{https://refactoring.com/}{https://refactoring.com/}
\item \href{https://refactoring.guru/}{https://refactoring.guru/}
\end{itemize}

\subsection{Tipi di Bad Smells: Bloaters}
I Bloaters sono codice, metodi e classi che sono
cresciuti a proporzioni così gigantesche da diventare
difficili da gestire.
Di solito, questi odori non emergono immediatamente, ma
si accumulano nel tempo mentre il programma evolve (e
soprattutto quando nessuno si sforza di eliminarli).
Esempi
\begin{itemize}
\item Metodo Lungo (Long Method)
\item Classe Grande (Large Class)
\item Ossessione Primitiva (Primitive Obsession)
\item Lista Parametri Lunga (Long Parameter List)
\item Grumi di Dati (Data Clumps)
\end{itemize}

\subsection{Tipi di Bad Smells: Abusatori della Programmazione Orientata agli Oggetti}
Tutti questi odori sono applicazioni incomplete o errate
dei principi di programmazione orientata agli oggetti.
\begin{itemize}
\item Classi Alternative con Interfacce Diverse (Alternative Classes with Different Interfaces)
\item Eredità Rifiutata (Refused Bequest)
\item Istruzioni Switch (Switch Statements)
\item Campo Temporaneo (Temporary Field)
\end{itemize}

\subsection{Tipi di Bad Smells: Impedimenti al Cambiamento}
Questi odori significano che se è necessario cambiare qualcosa
in un punto del codice, bisogna fare molti
cambiamenti anche in altri punti. Lo sviluppo del programma
diventa molto più complicato e costoso di
conseguenza.
\begin{itemize}
\item Cambiamento Divergente (Divergent Change)
\item Gerarchie di Eredità Parallele (Parallel Inheritance Hierarchies)
\item Chirurgia con Fucile a Pompa (Shotgun Surgery)
\end{itemize}

\subsection{Tipi di Bad Smells: Superflui}
Un superfluo è qualcosa di inutile e non necessario
la cui assenza renderebbe il codice più pulito, più
efficiente e più facile da capire.
\begin{itemize}
\item Commenti (Comments)
\item Codice Duplicato (Duplicate Code)
\item Classe Dati (Data Class)
\item Codice Morto (Dead Code)
\item Classe Pigra (Lazy Class)
\item Generalità Speculativa (Speculative Generality)
\end{itemize}

\subsection{Tipi di Bad Smells: Accoppiatori}
Tutti gli odori in questo gruppo contribuiscono all'eccessivo
accoppiamento tra classi o mostrano cosa succede se
l'accoppiamento è sostituito da un'eccessiva delega.
\begin{itemize}
\item Invidia delle Caratteristiche (Feature Envy)
\item Classe Libreria Incompleta (Incomplete Library Class)
\item Catene di Messaggi (Message Chains)
\item Uomo di Mezzo (Middle Man)
\end{itemize}

\section{Bad Smells e il concetto
di Debito Tecnico}
\subsection{Debito Tecnico}
Si verifica ogni volta che il codice soddisfa i requisiti funzionali ma è subottimale o "veloce
e sporco". Ad esempio:
\begin{itemize}
\item codice "maleodorante"
\item algoritmi inefficienti
\item design sciatto
\end{itemize}
Potrebbe essere corretto durante la code-review, potrebbe generare TODOs o nuovi problemi nel
sistema di tracciamento dei problemi.
Comprendere, misurare e comunicare il debito tecnico è
critico nell'industria del software.

\subsection{Debito Strategico, Intenzionale}
Esempi di debito intenzionale o strategico:
\begin{itemize}
\item Design non modulare
\item Implementazione intenzionalmente troppo semplice/troppo complessa
\item Indifferenza alle prestazioni
\item Mancanza di generalità o estensibilità
\item Mancanza di scalabilità
\end{itemize}

\subsection{Debito Non-Strategico, Non Intenzionale}
Il debito si accumula anche involontariamente, nessun processo di sviluppo è
perfetto.
Esempi di debito non intenzionale o non strategico:
\begin{itemize}
\item Bad Smells
\item Memory leaks
\item Copertura dei test insufficiente
\item Implementazione involontariamente complessa
\item Architettura rigida/fragile
\item Colli di bottiglia nelle prestazioni o nella scalabilità
\item Codice disordinato o difficile da mantenere
\end{itemize}

\subsection{Conseguenze del Debito}
Troppo debito => troppo tempo speso a pagare gli interessi.
Imprevedibilità nella fase di pianificazione del software, aumento del rischio di
investimento.
Rallenta il lavoro futuro.
Più bug, più costoso correggerli.
Sviluppatori frustrati e infelici.

\subsection{Misurare il Debito Tecnico}
Alcuni strumenti
relativi alla qualità,
come SonarQube,
possono fornire una
stima accurata del
debito tecnico
accumulato all'interno
di un repository software.

\section{Correggere i Bad Smells con
il Refactoring}
\subsection{Refactoring}
Metafora di base:
\begin{itemize}
\item Iniziare con una base di codice esistente e migliorarla.
\item Cambiare la struttura interna (dal piccolo al medio) preservando
la semantica complessiva: cioè, riorganizzare i "fattori" ma ottenere lo stesso "prodotto" finale.
\end{itemize}
L'idea è che dovresti migliorare il codice in qualche modo significativo.
Per esempio:
\begin{itemize}
\item Ridurre il codice quasi duplicato
\item Migliorare la coesione, ridurre l'accoppiamento
\item Migliorare la parametrizzazione, la comprensibilità, la manutenibilità, la flessibilità,
l'astrazione, l'efficienza, ecc...
\end{itemize}

\subsection{Il Ciclo di Refactoring}
Schema di base per il refactoring con un programma funzionante:
\begin{enumerate}
\item Scegliere l'odore peggiore
\item Selezionare un refactoring che affronterà l'odore
\item Applicare il refactoring.
\end{enumerate}
Selezionare refactoring per migliorare il codice ad ogni passaggio attraverso il ciclo.
Il comportamento del programma non viene modificato.
Quindi, il programma rimane in uno stato funzionante. Così il ciclo migliora il codice ma mantiene il comportamento.

La parte più difficile del processo: Identificare l'odore!
Quando inizi il refactoring, è meglio iniziare con le cose facili (ad esempio, suddividere grandi routine o rinominare cose per chiarezza).
Questo ti permette di vedere e correggere i problemi rimanenti più facilmente.

\subsection{Refactoring e Test Unitari}
Il refactoring dipende fortemente dall'avere una buona suite di test unitari.
Con i test unitari, possiamo fare refactoring.
Poi eseguire i test automatizzati, per verificare che il comportamento sia effettivamente preservato.
Senza buoni test unitari,
\begin{itemize}
\item gli sviluppatori potrebbero evitare il refactoring
\item A causa della paura di poter rompere qualcosa.
\end{itemize}

\subsection{Perché Dovresti Fare Refactoring?}
La realtà:
\begin{itemize}
\item Estremamente difficile ottenere il design "giusto" la prima volta
\item Difficile comprendere pienamente il dominio del problema
\item Difficile capire i requisiti dell'utente, anche se l'utente li conosce!
\item Difficile sapere come il sistema evolverà in X anni
\item Il design originale è spesso inadeguato
\item Il sistema diventa fragile nel tempo e più difficile da cambiare
\end{itemize}
Il refactoring ti aiuta a
\begin{itemize}
\item Manipolare il codice in un ambiente sicuro (preservando il comportamento)
\item Ricreare una situazione in cui l'evoluzione è possibile
\item Comprendere il codice esistente
\end{itemize}

Il refactoring migliora il design del software
\begin{itemize}
\item Senza refactoring il design del programma si deteriorerà
\item Il codice mal progettato di solito richiede più codice per fare le stesse cose, spesso
perché il codice fa la stessa cosa in luoghi diversi
\end{itemize}
Il refactoring rende il software più facile da capire. Nella maggior parte degli ambienti di sviluppo software, qualcun altro dovrà eventualmente leggere il tuo codice.
Il refactoring ti aiuta a trovare bug.
Il refactoring ti aiuta a programmare più velocemente.

\includepdf[pages=-, addtotoc={1,section,1,Refactoring tipici, L6:1, 7,subsection,1,Referenze,L6:2}]{esempi/lezione14.pdf}

\subsection{Codice Duplicato}
Se vedi la stessa struttura di codice in più di un posto, puoi essere
sicuro che il tuo programma sarà migliore se trovi un modo per unificarli.
Il problema di codice duplicato più semplice è quando hai la stessa
espressione in due metodi della stessa classe.
Esegui Extract Method (Estrai Metodo) e invoca il codice da entrambi i posti.
Un altro problema di duplicazione comune è avere la stessa espressione in
due sottoclassi sorelle.
Esegui Extract Method in entrambe le classi e poi Pull Up Field (Tira Su il Campo).
Se hai codice duplicato in due classi non correlate, considera di usare
Extract Class (Estrai Classe) in una classe e poi usa il nuovo componente nell'altra.

\subsection{Metodo Lungo}
Più lungo è una procedura, più è difficile da capire.
I programmi orientati agli oggetti vivono meglio e più a lungo con metodi brevi.
Quasi sempre, tutto ciò che devi fare per abbreviare un metodo è Extract
Method.
Se provi a usare Extract Method e finisci per passare molti parametri,
spesso puoi creare una nuova Classe che incapsuli i parametri.

\subsection{Classe Grande}
Quando una classe sta cercando di fare troppo, spesso si manifesta con troppe
variabili di istanza. Quando una classe ha troppe variabili di istanza,
il codice duplicato non può essere lontano.
Una classe con troppo codice è anche un terreno fertile per la duplicazione.
In entrambi i casi Extract Class ed Extract Subclass funzioneranno.

\subsection{Lista Parametri Lunga}
Vecchia scuola: passa tutto come parametri. Era meglio dei dati globali.
Con gli oggetti non è necessario passare tutto ciò di cui il metodo ha bisogno,
invece passi abbastanza in modo che il metodo possa accedere a tutto ciò di cui ha
bisogno.
Questo è positivo, perché le lunghe liste di parametri sono difficili da capire,
perché sono incoerenti e difficili da usare, perché le stai
cambiando continuamente quando hai bisogno di più dati.
Usa Replace Parameter with Method (Sostituisci Parametro con Metodo) quando puoi ottenere i dati in un
parametro facendo una richiesta a un oggetto che già conosci.
Usa Introduce Parameter Object (Introduci Oggetto Parametro).

\subsection{Cambiamento Divergente}
Il cambiamento divergente si verifica quando una classe viene comunemente modificata in
modi diversi per ragioni diverse. Se stai facendo questo, stai quasi certamente violando i principi di una chiave astrazione e separazione delle preoccupazioni, e dovresti fare refactoring del tuo codice.
Per sistemare questo, identifica tutto ciò che cambia per una particolare
causa e usa Extract Class per metterli tutti insieme.

\subsection{Chirurgia con Fucile a Pompa}
Questa situazione si verifica quando ogni volta che fai un tipo di cambiamento,
devi fare molti piccoli cambiamenti a molte classi diverse.
Lo stesso tasso di cambiamento in oggetti diversi, particolarmente se sono
scollegati.
Un cambiamento concettualmente semplice che richiede modifiche al codice in
molti posti. Il risultato della Programmazione Copia e Incolla.
Quando i cambiamenti sono sparsi ovunque sono difficili da trovare, ed è
facile perdere un cambiamento importante.
Vuoi usare Move Method (Sposta Metodo) e Move Field (Sposta Campo) per mettere tutti i cambiamenti in una
singola classe.
Se nessuna classe attuale sembra un buon candidato, creane una.

\subsection{Invidia delle Caratteristiche}
"L'invidia delle caratteristiche" è quando un metodo fa un uso intenso di dati e
metodi da un'altra classe.
Usa Move Method per metterlo nella classe più desiderata.
A volte solo una parte del metodo fa un uso intenso delle caratteristiche di
un'altra classe.
Usa Extract Method per estrarre quelle parti che appartengono all'altra classe.

\subsection{Grumi di Dati}
Spesso vedrai gli stessi tre o quattro elementi di dati insieme in molti
luoghi:
\begin{itemize}
\item Campi in un paio di classi
\item Parametri in molte firme di metodi
\end{itemize}
Gruppi di dati che stanno insieme dovrebbero davvero essere trasformati in
un proprio oggetto, Es.: x,y → Point.
Il primo passo è cercare dove i grumi appaiono come campi e usare
Extract Class per trasformare i grumi in un oggetto.
Per i parametri dei metodi usa Introduce Parameter Object o Preserve
Whole Object (Preserva Oggetto Intero) per snellirli.

\subsection{Ossessione Primitiva}
Le persone nuove agli oggetti sono talvolta riluttanti a usare piccoli oggetti per
piccoli compiti, come classi per soldi che combinano numeri e
valuta, intervalli con un limite superiore e inferiore, e stringhe speciali come
numeri di telefono e codici postali.
Molti programmatori sono riluttanti a introdurre classi "piccole" che
rappresentano cose facilmente rappresentate da primitivi—numeri di telefono,
codici postali, importi di denaro, intervalli (variabili con limiti superiori e
inferiori).
Se il tuo primitivo ha bisogno di dati o comportamenti aggiuntivi, considera di trasformarlo
in una classe.
Per esempio, potresti voler formattare il tuo primitivo in un modo speciale, come
(215)898-0587 o 19104-6389.

\subsection{Test di Tipo}
La maggior parte delle volte quando vedi un'istruzione switch su un tipo dovresti
considerare il polimorfismo.
Usa Extract Method per estrarre l'istruzione switch e poi Move
Method per portarla nella classe dove è necessario il polimorfismo.

\subsection{Gerarchie di Eredità Parallele}
È davvero un caso speciale di chirurgia con fucile a pompa.
In questo caso ogni volta che fai una sottoclasse di una classe, devi
fare una sottoclasse di un'altra.
Ci sono due modi per procedere. Per rimuovere il parallelo: fare refactoring di una o entrambe
le gerarchie fino a quando i loro membri sono congruenti, quindi collassarle a coppie.
Per rimuovere la duplicazione tra i paralleli: definire responsabilità distinte
raffinate da ciascuna gerarchia e riposizionare i metodi come
appropriato. -- WardCunningham.
Se usi Move Method e Move Field, la gerarchia sulla classe di riferimento
scompare.

\subsection{Classe Pigra}
Ogni classe che crei costa denaro e tempo da mantenere e
capire.
Una classe che non sta portando il suo peso dovrebbe essere eliminata.
Se hai sottoclassi che non stanno facendo abbastanza prova a usare Collapse
Hierarchy (Collassa Gerarchia) e i componenti quasi inutili dovrebbero essere sottoposti a
Inline Class (Inserisci Classe).

\subsection{Generalità Speculativa}
Ottieni questo odore quando qualcuno dice "Penso che avremo bisogno di fare questo
un giorno" e hai bisogno di ogni tipo di ganci e casi speciali per gestire cose
che non sono richieste.
Questo odore è facilmente rilevabile quando gli unici utenti di una classe o metodo sono
casi di test.
Se hai classi astratte che non stanno facendo abbastanza, usa Collapse
Hierarchy.
La delega non necessaria può essere rimossa con Inline Class.
I metodi con parametri inutilizzati dovrebbero essere soggetti a Remove Parameter
(Rimuovi Parametro).
I metodi nominati con strani nomi astratti dovrebbero essere riparati con Rename
Method (Rinomina Metodo).

\subsection{Campo Temporaneo}
A volte vedrai un oggetto in cui una variabile di istanza è impostata
solo in determinate circostanze.
Questo può rendere il codice difficile da capire perché di solito ci aspettiamo
che un oggetto usi tutte le sue variabili.
Usa Extract Class per creare una casa per queste variabili orfane mettendo
tutto il codice che usa la variabile nel componente.

\subsection{Catene di Messaggi}
Le catene di messaggi si verificano quando vedi un client che chiede a un oggetto un
altro oggetto, che il client poi chiede per un altro oggetto ancora, che
il client poi chiede per un altro oggetto ancora, ecc.: intermediate.getProvider().doSomething()
Altrimenti: 
 ch = vehicle->getChassis();
body = ch->getBody();
shell = body->getShell();
material = shell->getMaterial();
props = material->getProperties();
color = props->getColor();
Navigare in questo modo significa che il client è strutturato sulla struttura
della gerarchia.

\subsection{Middle Man}
Una delle principali caratteristiche degli Oggetti è l'incapsulamento.
L'incapsulamento spesso viene con la delega.
A volte la delega può andare troppo oltre.
Per esempio, se trovi che metà dei metodi sono delegati a un'altra classe,
potrebbe essere il momento di usare Remove Middle Man (Rimuovi Uomo di Mezzo) e parlare con l'oggetto che
realmente sa cosa sta succedendo.
Se solo pochi metodi non stanno facendo molto, usa Inline Method per incorporarli
nel chiamante.
Se c'è un comportamento aggiuntivo, puoi usare Replace Delegation with
Inheritance (Sostituisci Delega con Eredità) per trasformare l'uomo di mezzo in una sottoclasse dell'oggetto reale.

\includepdf[pages=-, addtotoc={1,subsection,1,Esempio di Middle Man, L6:3,2,section,1,Esercizi,L6:4}]{esercizi/esercizi14.pdf}
\chapter{Lezione 15: Teoria dei colori, tipografia e psicologia di Gestalt}

\section{Percezione del colore tramite i coni}
Non tutti i coni sono uguali. Esistono 3 tipi diversi, con fotopigmenti specializzati per percepire il colore: blu, verde e rosso. Ogni tipo di cono è sensibile a una diversa banda dello spettro luminoso. Il rapporto di stimolazione neurale tra i tre tipi ci dà una percezione continua dei colori.

I tipi di coni non sono distribuiti uniformemente al centro della retina: principalmente rossi, pochissimi blu. Sensibilità limitata alle lunghezze d'onda corte, alta sensibilità a quelle lunghe. Pochi coni blu al centro della retina, è più difficile mettere a fuoco piccoli oggetti blu. Con l'età, il cristallino tende ad assorbire più lunghezze d'onda corte, riducendo ulteriormente la sensibilità ai blu.

\section{Modelli di colore}
I modelli di colore sono modelli matematici astratti che descrivono come i colori possono essere rappresentati tramite tuple di numeri. Due schemi di colore ampiamente utilizzati sono:
\begin{itemize}
    \item CMYK
    \item RGB
\end{itemize}

\subsection{Modello di colore CMYK}
Il CMYK è un modello di colore sottrattivo. Utilizza: Ciano, Magenta, Giallo, Key (nero). È generalmente usato per materiali stampati. Utilizza pigmenti d'inchiostro per mostrare il colore. I colori risultano dalla luce riflessa.

\subsection{Modello di colore RGB}
L'RGB è un modello di colore additivo. Utilizza Rosso, Verde, Blu. È pensato per i display dei computer. Utilizza la luce per mostrare il colore. Il colore risulta dalla luce emessa. I display non possono produrre diversi canali di colore nella stessa posizione: la griglia dei pixel è tipicamente divisa in regioni di colore singolo (subpixel), che contribuiscono al colore visualizzato quando osservate a distanza.

\section{Dimensioni percettive del colore}
HSL (Hue, Saturation, Lightness) è una delle rappresentazioni più comuni dei colori nel modello RGB.

\subsection{Hue}
La tonalità (\textit{Hue}) è la lunghezza d'onda dominante in un colore. In HSL, è un valore tra 0 e 360:
\begin{itemize}
    \item 0 (e 360) è rosso
    \item 60 è giallo
    \item 120 è verde
    \item 180 è ciano
    \item 240 è blu
    \item 300 è magenta
\end{itemize}

\subsection{Saturazione}
Indica quanto il colore è «intenso».
\begin{itemize}
    \item Tipicamente è una percentuale tra 0 e 100
    \item 100\% significa che il colore è brillante e puro
    \item 80\% significa che il colore è miscelato con il 20\% di grigio
    \item 0\% significa che si ottiene solo il grigio
\end{itemize}

\subsection{Luminosità}
La luminosità (\textit{Lightness}) indica quanta luce ha il colore.
\begin{itemize}
    \item 0\% è molto scuro (nero)
    \item 50\% è a metà, né troppo scuro né troppo brillante
    \item 100\% è molto brillante (bianco)
\end{itemize}

\subsection{Rappresentazione HSL}
HSL è un modello a coordinate cilindriche. Definendo Hue, Saturation e Lightness, si identifica un punto nello spazio cilindrico.
\begin{figure}[htbp!]
    \centering
    \includegraphics[width=0.5\textwidth]{immagini/lezione15/1.png}
    \caption{Rappresentazione cilindrica di HSL}
\end{figure}

\subsection{Tinte, sfumature e toni}
\begin{itemize}
    \item Una tinta (\textit{tint}) è una miscela di un colore con il bianco, aumentando la luminosità
    \item Una sfumatura (\textit{shade}) è una miscela con il nero, diminuendo la luminosità
    \item Un tono (\textit{tone}) è una miscela con il grigio
\end{itemize}

\section{Uso dei colori nel design delle interfacce}
Il colore è uno strumento potente nell’arsenale di un designer di UI.
\begin{itemize}
    \item Può far risaltare alcuni elementi (vedremo di più su questo alla fine della lezione)
    \item Può trasmettere significato (ma attenzione alle differenze regionali!)
\end{itemize}

Quindi, più colori ci sono, meglio è?
\begin{itemize}
    \item Non proprio. Una UI dovrebbe generalmente includere non più di 6 colori diversi.
    \item Corollario: scegli una palette e sii coerente!
\end{itemize}

Una buona regola pratica è la regola del 60-30-10:
\begin{itemize}
    \item 60\% colore primario
    \item 30\% colore secondario
    \item 10\% colore d’accento (per le parti che vogliamo far risaltare)
\end{itemize}

\subsection{Teoria del colore}
Non tutti i colori «stanno bene» insieme. Possiamo usare le armonie cromatiche per costruire palette (insiemi di colori) per la nostra UI che non siano in contrasto tra loro.
\begin{itemize}
    \item Monocromatica
    \item Analoghi
    \item Complementari
    \item Complementari suddivisi
    \item Triadica
\end{itemize}

\includepdf[pages=-, addtotoc={
    1, subsection, 1, Armonie monocromatiche, L7:1,
    2, subsection, 1, Armonie analoghe, L7:2,
    3, subsection, 1, Armonie complementari, L7:3,
    4, subsection, 1, Armonie complementari suddivise, L7:4,
    5, subsection, 1, Armonie triadiche, L7:5,
    6, subsection, 1, Riferimenti: Armonie cromatiche, L7:6
}]{esempi/lezione15.pdf}

\subsection{Psicologia del colore}
La mente umana può reagire inconsciamente ai colori (psicologia del colore).
\begin{itemize}
    \item Il nero è associato a eleganza, potere e autorità
    \item Il blu è percepito come autorevole, affidabile, degno di fiducia
    \item Il rosso può essere associato a passione, desiderio, amore, energia, pericolo
    \item Il verde può essere associato a natura, freschezza, serenità, salute, denaro
\end{itemize}
Non ci sono molte ricerche che dimostrano l’effetto reale di un colore sulle emozioni.
Inoltre, tieni presente che esistono differenze regionali! In Cina, il rosso è il colore associato al denaro. Negli Stati Uniti, è il verde.

\section{Tipografia}
\subsection{L’ipotesi dei serif}
I caratteri con grazie (\textit{serif}) \textbf{sono più facili da leggere} – e quindi preferibili per lunghi testi – perché le grazie forniscono ancoraggi che guidano l’occhio del lettore.
I font senza grazie (\textit{sans serif}) mancano di questi ancoraggi e sono quindi meno adatti per lunghi testi.
In pratica, le differenze individuali superano qualsiasi effetto della presenza/assenza di grazie, cioè alcune persone leggono più velocemente di altre.

\subsection{Impatto della tipografia nelle UI}
La tipografia può essere un buon modo per veicolare messaggi nelle interfacce.
\begin{itemize}
    \item Leggere non significa solo riconoscere sequenze di lettere
    \item La tipografia può trasmettere messaggi aggiuntivi:
    \begin{itemize}
        \item «Questo è pensato per essere facile da leggere»
        \item «Questo è un messaggio giocoso»
        \item «Questo può essere percepito come codice sorgente»
    \end{itemize}
\end{itemize}
La leggibilità è un aspetto importante dell’usabilità.
\begin{itemize}
    \item Gli utenti devono leggere etichette e informazioni nelle nostre UI
    \item La tipografia gioca un ruolo importante nella facilità di lettura. Anche l’aspetto delle parole può essere importante.
\end{itemize}

\subsection{Effetto di superiorità della parola}
A volte, i lettori riconoscono una parola dalla sua forma, prima ancora di riconoscere le lettere che la compongono.
Questo è chiamato «Effetto di superiorità della parola» in psicologia cognitiva.
Le persone generalmente leggono il testo latino minuscolo più velocemente di quello maiuscolo.
Non usare il maiuscolo per lunghi testi!

\section{Teoria della percezione Gestalt}
\textbf{La teoria della percezione Gestalt} (nota anche come psicologia della Gestalt) si concentra su come la mente umana elabora le informazioni visive.
La psicologia della Gestalt si riferisce all’idea di insieme unificato.
\begin{itemize}
    \item Generalmente percepiamo qualcosa di diverso dalla semplice somma degli elementi che vediamo
    \item Diamo significato alla somma delle parti piuttosto che ai singoli elementi
\end{itemize}

Quindi, cosa vedi nell’immagine qui sotto?
\begin{itemize}
    \item Un triangolo bianco sopra
    \item Un triangolo con bordi neri sotto
    \item Tre cerchi neri parzialmente coperti
\end{itemize}
Perché non vediamo solo un insieme di linee e macchie?
La nostra mente tende a completare oggetti incompleti e a vedere connessioni tra elementi in base alla loro apparenza o posizione relativa.

\begin{figure}[htbp!]
    \centering
    \includegraphics[width=0.5\textwidth]{immagini/lezione15/2.png}
    \caption{Gestalt in azione}
\end{figure}

\subsection{Psicologia della Gestalt}
La psicologia della Gestalt studia queste tendenze della nostra mente e come si manifestano.
Queste tendenze sono talvolta chiamate \textbf{«leggi della percezione»}.
\begin{itemize}
    \item Non sono vere leggi, ma principi o euristiche importanti
    \item Nelle prossime slide vedremo alcuni di questi principi
    \item Capendoli, possiamo usarli per rendere le nostre UI più intuitive
    \item Vogliamo che le nostre UI lavorino con, e non contro, il modo in cui il cervello elabora gli stimoli visivi
\end{itemize}

\section{Principi della Gestalt}
\subsection{Somiglianza}
\begin{figure}[htbp!]
    \centering
    \includegraphics[width=0.5\textwidth]{immagini/lezione15/3.png}
    \caption{Forme righe-colonne}
\end{figure}
Scommetto che hai interpretato l’immagine sopra come quattro colonne e non tre righe.
Gli elementi che condividono una caratteristica visiva sono percepiti come più correlati rispetto agli elementi dissimili.
Le caratteristiche visive possono essere forme, dimensioni, colori, font, movimento, orientamento...

\begin{figure}[H]
    \centering
    \includegraphics[width=0.5\textwidth]{immagini/lezione15/4.png}
    \caption{Aggiungendo un colore...}
\end{figure}

Aggiungendo un’altra caratteristica visiva (colore), la percezione può cambiare.
Probabilmente ora vedi tre righe.
La somiglianza cromatica spesso prevale su altre caratteristiche visive.

\subsection{Somiglianza nelle UI}
La somiglianza può essere usata per raggruppare elementi correlati.
\begin{itemize}
    \item Se vuoi che elementi diversi siano percepiti come raggruppati e correlati, puoi farli condividere una o più caratteristiche visive
    \item Può essere usata per comunicare funzionalità comuni (ad esempio: pensa ai colori per segnalare i link nelle pagine web)
\end{itemize}
La somiglianza può anche essere usata per enfatizzare le differenze.

\includepdf[pages=-, addtotoc={1, subsection, 1, Prossimità, L7:9}]{esempi/lezione15-1.pdf}

\subsection{Prossimità: moduli}
Moduli lunghi con molti campi di input possono sembrare \textbf{opprimenti}.
Raggruppare i campi correlati aiuta gli utenti a comprendere le informazioni da inserire.
Lo spazio è un modo per raggruppare i campi correlati.

\subsection{Prossimità: posizionamento delle etichette nei moduli}
L’approccio più sicuro è posizionare le etichette sopra i campi di input.
Se vuoi moduli più compatti, considera di posizionare le etichette a sinistra dei campi.
\begin{itemize}
    \item Le etichette dovrebbero avere lunghezza simile e non essere troppo distanti dai campi
    \item Le etichette allineate a destra sono note per \href{https://www.nngroup.com/articles/right-justified-navigation-menus/}{ostacolare la scansione}
\end{itemize}
Attenzione a non raggruppare elementi non correlati!
Può nascondere elementi non correlati, rendendoli meno visibili.

\includepdf[pages=-, addtotoc={
    1, subsection, 1, Proximity can backfire: example , L7:10,
    5, subsection, 1, Principle of Connectedness, L7:11,
    12, subsection, 1, Principle of Common Region, L7:12,
    15, subsection, 1, Principle of Common Region: examples, L7:13,
    17, section, 1, Visual Hierarchy in UI Design , L7:14,
    20, subsection, 1, Creating a Visual Hierarchy, L7:15,
    21, subsection, 1, Scala, L7:16,
    22, subsection, 1, Colore/Contrasto, L7:17,
    23, subsection, 1, Raggruppamento, L7:18
}]{esempi/lezione15-2.pdf}
\chapter{Lezione 17: Linee guida e principi nell'Interazione Uomo-Macchina}

\section{Linee guida e principi nell'HCI}
Linee guida e principi per un buon design delle interfacce sono stati sviluppati nel corso degli anni.
Queste linee guida sono applicabili alla maggior parte dei sistemi interattivi.
Derivano dall'esperienza e sono state affinate nel tempo.
Non pretendono di essere complete o universali.
Richiedono validazione e adattamento per domini di design specifici.
Sono comunque utili per studenti e professionisti.

Le linee guida e i principi possono essere utili:
\begin{itemize}
    \item Per guidare la fase di progettazione
    \item Per valutare una UI e trovare problemi di usabilità (vedremo tra qualche lezione)
\end{itemize}
Esistono sovrapposizioni tra i diversi insiemi di linee guida/principi, come temi ricorrenti: prevenire errori, minimizzare il carico di memoria, ...

\section{Le otto regole d'oro di Shneiderman}
\begin{enumerate}
    \item Cerca la coerenza
    \item Punta all'usabilità universale
    \item Offri feedback informativo
    \item Progetta dialoghi che portino a una conclusione
    \item Previeni gli errori
    \item Permetti la facile reversibilità delle azioni
    \item Mantieni l'utente al controllo
    \item Riduci il carico di memoria a breve termine
\end{enumerate}

\subsection{Le dieci euristiche di usabilità Nielsen-Molich}
\begin{enumerate}
    \item Dialogo semplice e naturale
    \item Parla la lingua dell'utente
    \item Minimizza il carico di memoria dell'utente
    \item Coerenza
    \item Feedback
    \item Uscite chiaramente indicate
    \item Scorciatoie
    \item Buoni messaggi di errore
    \item Prevenzione degli errori
    \item Aiuto e documentazione
\end{enumerate}

\subsection{Dialogo semplice e naturale}
La UI dovrebbe essere il più semplice possibile (ma non troppo!).
\begin{itemize}
    \item Meno è meglio: ogni funzionalità/informazione aggiuntiva è qualcosa in più da imparare, fraintendere o cercare
    \item Gli utenti inesperti possono sentirsi sopraffatti da troppe informazioni
    \item Legge di Hick!
\end{itemize}
La UI dovrebbe corrispondere il più naturalmente possibile al compito dell'utente (mapping e metafore!).
\begin{itemize}
    \item La versione digitale di un modulo è organizzata come quella cartacea
    \item Un'app bussola funziona come una bussola reale
\end{itemize}

\section{Parla la lingua dell'utente}
Il dialogo dovrebbe essere espresso con parole, frasi e concetti familiari all'utente, non con termini orientati al sistema (design centrato sull'uomo!).
I dialoghi dovrebbero essere nella lingua madre dell'utente (localizzazione), non solo nel testo ma anche negli elementi non verbali come le icone!
Attenzione alle parole che usi.
\begin{itemize}
    \item Se progetti per il pubblico generale, usa parole che tutti comprendono, con il loro significato standard
    \item Se progetti per un gruppo con una terminologia specifica, usa i termini specializzati
\end{itemize}

\subsection{Minimizza il carico di memoria dell'utente}
I computer ricordano perfettamente, la memoria di lavoro umana molto meno! (Ricordi MHP?)
La UI dovrebbe sollevare l'utente dal peso della memoria il più possibile.
Come?
\begin{itemize}
    \item Il riconoscimento è meglio del richiamo
    \item Quando chiedi input agli utenti, descrivi il formato richiesto e fornisci un esempio. Indica esplicitamente i valori ammessi (se ci sono)
\end{itemize}

\subsection{Minimizza il carico di memoria dell'utente: esempi}
Gli utenti devono inserire il codice a due lettere di una provincia italiana.
\begin{itemize}
    \item Accettare l'input tramite campo di testo richiede agli utenti di ricordare il codice della provincia desiderata
    \begin{itemize}
        \item Qual è il codice per Lecce?
        \item E per Lecco?
    \end{itemize}
    \item Usare una lista a discesa con nomi e codici di tutte le province solleva l'utente dal dover ricordare i codici delle 110 province italiane!
\end{itemize}

\subsection{Coerenza}
Per essere usabile, un sistema dovrebbe mostrare coerenza interna ed esterna.
\textbf{Coerenza interna} (all'interno del prodotto o della famiglia di prodotti):
\begin{itemize}
    \item Sequenze di azioni coerenti dovrebbero essere richieste in situazioni simili: cancellare un cliente e cancellare un fornitore dovrebbero richiedere una sequenza simile
    \item Le stesse informazioni dovrebbero essere presentate nello stesso modo e nella stessa posizione su tutte le schermate
\end{itemize}
\textbf{Coerenza esterna} (con le convenzioni consolidate)

\includepdf[pages=-, addtotoc={
    1, subsection, 1, Coerenza interna: esempi, L8:1,
    5, subsection, 1, Coerenza esterna, L8:2
}]{esempi/lezione17.pdf}

\section{Feedback}
Il sistema dovrebbe informare continuamente l'utente su cosa sta facendo e come interpreta l'input dell'utente.
\begin{itemize}
    \item Non solo quando si verificano errori
    \item Il feedback positivo è importante quanto quello negativo
    \item Quando possibile, dai feedback anche in caso di errori di sistema
    \item Il peggior feedback possibile è nessun feedback!
    \item Il feedback non dovrebbe essere troppo astratto o generico
\end{itemize}

\subsection{Persistenza del feedback}
Tipi diversi di feedback possono richiedere diversi livelli di persistenza.
\begin{itemize}
    \item Alcuni feedback sono rilevanti solo per la durata di un fenomeno o sono semplici conferme di un'operazione
    \item Possono scomparire automaticamente (es: messaggi toast)
    \item Altri (soprattutto avvisi o errori) possono richiedere un riconoscimento esplicito da parte dell'utente
    \item Altri ancora possono richiedere alta persistenza e diventare parte permanente della UI
\end{itemize}

\subsection{Feedback e tempi di risposta del sistema}
Il feedback è cruciale quando i sistemi hanno tempi di risposta lunghi.
\begin{enumerate}
    \item Meno di 0,1 secondi: le reazioni sono percepite come istantanee
    \begin{itemize}
        \item Nessun feedback richiesto, tranne mostrare il risultato o confermare l'esito
    \end{itemize}
    \item Meno di 1 secondo: il flusso di pensiero dell'utente resta ininterrotto
    \begin{itemize}
        \item Nessun feedback speciale richiesto (ma non si ha la sensazione di reazione istantanea)
    \end{itemize}
    \item 10 secondi: limite per mantenere l'attenzione dell'utente
    \begin{itemize}
        \item Il feedback è cruciale per ritardi superiori a 10 secondi
        \item Fornisci una stima di quando il compito sarà completato (gli utenti vorranno fare altro mentre aspettano)
        \item Aggiorna frequentemente l'indicatore di progresso
    \end{itemize}
\end{enumerate}

\subsection{Labor Perception Bias}
Il Labor Perception Bias: le persone si fidano e apprezzano di più ciò che percepiscono come frutto di lavoro.
Tutti odiano aspettare.
Ma se le aspettative sono alte (es: gestione di denaro, backup o migrazione di dati importanti, analisi e report...), possono diventare scettici se il tempo di attesa è troppo breve!
\begin{itemize}
    \item Aggiungere una schermata di lavoro subito dopo un'azione chiave può migliorare l'esperienza utente
    \item Talvolta vengono aggiunti dai designer dei "inganni benevoli" (es: tempi di caricamento finti): \href{https://www.theatlantic.com/technology/archive/2017/02/why-some-apps-use-fake-progress-bars/517233/}{link all'articolo}
\end{itemize}

\subsection{Uscite chiaramente indicate e reversibilità delle azioni}
Gli utenti vogliono sentirsi in controllo dell’interazione.
Gli utenti commetteranno comunque errori durante l’uso del sistema.
Il sistema dovrebbe offrire una via d’uscita semplice dalla maggior parte delle situazioni.
\begin{itemize}
    \item Se il sistema non riesce a completare l’azione entro 10 secondi, l’utente dovrebbe poter interrompere l’operazione e annullare l’azione
    \item In operazioni con effetti collaterali, le uscite possono essere offerte tramite una funzione di «Undo» che riporta il sistema allo stato precedente
\end{itemize}

\subsection{Scorciatoie}
In generale, l’uso di una UI dovrebbe richiedere la conoscenza di poche regole.
Gli utenti esperti dovrebbero poter eseguire rapidamente azioni frequenti tramite scorciatoie e acceleratori.
\begin{itemize}
    \item Tasti funzione o combinazioni che eseguono un intero comando con una pressione
    \item Doppio clic su un oggetto per eseguire l’azione più comune su quell’oggetto
    \item Pulsanti specifici per accedere direttamente a funzioni importanti dove sono più necessarie
    \item Riutilizzo della cronologia delle interazioni (ripetere rapidamente gli stessi comandi)
    \item Fornire valori predefiniti nei moduli, quando possibile
\end{itemize}

\includepdf[pages=-, addtotoc={
    4, subsection, 1, Linee guida sulle scorciatoie, L8:3
}]{esempi/lezione17-1.pdf}

\section{Messaggi di errore}
I messaggi di errore sono fondamentali per l’usabilità.
\begin{itemize}
    \item Rappresentano situazioni in cui gli utenti sono in difficoltà e potrebbero non riuscire a raggiungere i propri obiettivi
    \item Offrono opportunità per aiutare gli utenti a comprendere meglio il sistema. Gli utenti sono generalmente più motivati a prestare attenzione al contenuto dei messaggi di errore
\end{itemize}

\subsection{Buoni messaggi di errore}
Secondo Shneiderman, i messaggi di errore dovrebbero seguire quattro regole:
\begin{enumerate}
    \item Devono essere formulati in linguaggio chiaro ed evitare codici oscuri
    \item Devono essere precisi, non vaghi o generici
    \item Devono aiutare costruttivamente l’utente a risolvere il problema
    \item Devono essere cortesi e non intimidire o colpevolizzare l’utente
\end{enumerate}

\includepdf[pages=-]{esempi/lezione17-2.pdf}

\subsection{Buoni messaggi di errore: sii cortese}
Non intimidire o colpevolizzare l’utente.
\begin{itemize}
    \item Gli utenti già si sentono frustrati quando non riescono a raggiungere i propri obiettivi, non serve peggiorare la situazione!
    \item Evita termini offensivi come: «AZIONE UTENTE ILLEGALE!», «LAVORO ANNULLATO», «PROCESSO TERMINATO», «ERRORE FATALE»
    \item Cerca di formulare il messaggio di errore in modo da suggerire che il problema è del sistema (in fondo, una buona UI avrebbe potuto prevenire quell’errore!)
\end{itemize}

\subsection{Buoni messaggi di errore: livelli multipli}
I messaggi di errore sono utili sia per gli utenti che per il personale tecnico.
\begin{itemize}
    \item Gli utenti normalmente non comprendono dettagli tecnici (es: codici di errore o stack trace), ma il personale tecnico può averne bisogno per la risoluzione dei problemi
    \item I messaggi di errore possono dover contenere entrambi i livelli di informazione
\end{itemize}
Spesso è preferibile separare le viste dei diversi livelli:
\begin{itemize}
    \item Gli utenti normali non sono intimiditi da messaggi strani
    \item Il personale tecnico può accedere alle informazioni di troubleshooting
    \item Le finestre di errore possono includere link a siti di supporto
\end{itemize}

\subsection{Prevenire gli errori}
Ancora meglio che avere buoni messaggi di errore, è evitare gli errori!
Cerca di non mettere gli utenti in situazioni soggette a errore.
\begin{itemize}
    \item Se chiedi all’utente di digitare il nome di una città, c’è il rischio di errori di ortografia
    \item Se l’utente deve inserire un intervallo di date nel futuro, formattato in modo specifico, c’è il rischio che non formatti correttamente la data o inserisca una data non valida
\end{itemize}
Progetta la UI per evitare (o minimizzare) questi errori: non è solo positivo per l’usabilità, ma significa anche meno lavoro per formalizzare i casi d’uso e meno codice per gestire situazioni di errore!

\section{Tipi di errori}
Secondo Don Norman, esistono due categorie di errori:
\begin{enumerate}
    \item \textbf{Slips}: l’utente intende eseguire un’azione, ma ne compie un’altra
    \begin{itemize}
        \item Premere il tasto «Invio» invece di «Backspace»
        \item Cliccare sul pulsante «Minimizza» invece di «Massimizza»
    \end{itemize}
    \item \textbf{Mistakes}: l’utente forma obiettivi non appropriati per il problema/compito corrente
    \begin{itemize}
        \item Il responsabile di un sito e-commerce vuole eliminare tutti gli articoli di una certa categoria. Crede che eliminando la categoria verranno eliminati anche gli articoli associati. In realtà, gli articoli vengono spostati implicitamente nella categoria «Altro».
    \end{itemize}
\end{enumerate}

\subsection{Tipi di errori: Slips}
Se gli utenti formano obiettivi corretti, ma sbagliano l’esecuzione, hanno commesso uno \textit{slip}.
Gli slip derivano tipicamente da comportamenti automatici.
Sono più frequenti nei comportamenti esperti (gli utenti prestano più attenzione quando stanno ancora imparando).
Gli slip sono spesso «errori di cattura»: quando due sequenze di azioni hanno un prefisso comune, e una viene usata molto più spesso dell’altra, gli utenti finiscono inconsciamente per seguire la sequenza più frequente, anche se volevano eseguire quella meno frequente.
Gli slip sono il motivo per cui permettere la facile reversibilità delle azioni è generalmente preferibile rispetto a fare affidamento solo su una finestra di conferma.

\subsection{Tipi di errori: Mistakes}
\begin{itemize}
    \item I mistake sono molto più critici
    \item Spesso derivano dal fatto che l’utente ha formato un modello mentale errato del sistema
    \item Possono essere molto più difficili da rilevare (e quindi più pericolosi!)
    \item Ripensa all’esempio precedente:
    \begin{itemize}
        \item Il responsabile di un sito e-commerce vuole eliminare tutti gli articoli di una certa categoria. Crede che eliminando la categoria verranno eliminati anche gli articoli associati. In realtà, gli articoli vengono spostati implicitamente nella categoria «Altro».
        \item Quando se ne accorgerà?
    \end{itemize}
\end{itemize}

\subsection{Aiuto e documentazione}
Idealmente, un sistema dovrebbe essere così facile da usare da non richiedere aiuto o documentazione aggiuntiva.
Questo obiettivo purtroppo non è sempre raggiungibile. A parte i sistemi veramente «walk-up-and-use», la maggior parte delle UI ha abbastanza funzioni da giustificare un manuale e possibilmente un sistema di aiuto.
Un manuale può essere utile anche agli utenti regolari per acquisire maggiore competenza e aumentare la produttività.
Nota: avere un buon manuale e sistema di aiuto non riduce i requisiti di usabilità! «È tutto spiegato nel manuale!» non è una buona scusa per una UI poco usabile!

\subsection{La verità fondamentale sui manuali utente}
Gli utenti non leggono i manuali utente.
Preferiscono dedicare tempo ad attività che li fanno sentire produttivi.
Tipicamente iniziano a usare il sistema senza aver letto le istruzioni.
Corollario:
\begin{itemize}
    \item Se gli utenti vogliono leggere il manuale, probabilmente sono in difficoltà e hanno bisogno di aiuto immediato
    \item I manuali online con ricerca orientata ai compiti e funzioni di ricerca personalizzata sono particolarmente utili in questi casi
\end{itemize}

\subsection{Puntare all'usabilità universale}
Cerca l’usabilità per tutti! Bisogna considerare:
Differenze tra utenti principianti ed esperti, fasce d’età, disabilità, variazioni internazionali.
Progettare per tutti non significa ottenere un prodotto meno efficace.
Spesso molte categorie di utenti beneficiano di accorgimenti pensati per una categoria specifica.
Pensa alle rampe sui marciapiedi! (\href{https://en.wikipedia.org/wiki/Curb_cut_effect}{curb cut effect})

\subsection{Progettare dialoghi che portino a una conclusione}
Le sequenze di azioni dovrebbero essere organizzate in gruppi con inizio, sviluppo e fine.
Un feedback informativo al termine di un gruppo dovrebbe dare all’utente soddisfazione, senso di sollievo, e indicare che può prepararsi al prossimo gruppo di azioni.
Ad esempio, i siti e-commerce guidano i clienti attraverso una serie di passaggi chiari:
\begin{itemize}
    \item Aggiunta degli articoli al carrello
    \item Specifica del metodo di pagamento, indirizzo di consegna, ecc.
    \item Pagamento
\end{itemize}

\section{Letture e riferimenti}
\begin{enumerate}
    \item Shneiderman, B., \& Plaisant, C. (2010). Designing the user interface: strategies for effective human-computer interaction. Pearson Education.
    \item Molich, R., \& Nielsen, J. (1990). Improving a human-computer dialogue. Communications of the ACM, 33(3), 338-348.\\
    \href{https://dl.acm.org/doi/10.1145/77481.77486}{https://dl.acm.org/doi/10.1145/77481.77486}
    \item Holcomb, R., \& Tharp, A. L. (1991). What users say about software usability. International Journal of Human‐Computer Interaction, 3(1), 49-78.\\
    \href{https://doi.org/10.1080/10447319109525996}{https://doi.org/10.1080/10447319109525996}
    \item Polson, P. G., \& Lewis, C. H. (1990). Theory-based design for easily learned interfaces. Human–Computer Interaction, 5(2-3), 191-220.\\
    \href{https://doi.org/10.1080/07370024.1990.9667154}{https://doi.org/10.1080/07370024.1990.9667154}
    \item Carroll, J. M., \& Rosson, M. B. (1992). Getting around the task-artifact cycle: How to make claims and design by scenario. ACM Transactions on Information Systems (TOIS), 10(2), 181-212.\\
    \href{https://dl.acm.org/doi/abs/10.1145/146802.146834}{https://dl.acm.org/doi/abs/10.1145/146802.146834}
    \item Nielsen, J. (1994, April). Enhancing the explanatory power of usability heuristics. In Proceedings of the SIGCHI conference on Human Factors in Computing Systems (pp. 152-158).\\
    \href{https://dl.acm.org/doi/10.1145/191666.191729}{https://dl.acm.org/doi/10.1145/191666.191729}
\end{enumerate}

\includepdf[pages={11-45}, addtotoc={
11, chapter, 1, Lezione 20: Software Design: Architetture Software, L820,
11, section, 1, Architetture, L821,
13, subsection, 1, Definizione dell'architettura, L822,
14, section, 1, Identificazione di sottosistemi, L823,
15, subsection, 1, Design Principle: Divide and conquer, L824,
16, subsection, 1, Ways of dividing a software system, L825,
18, subsection, 1, Layer, L826,
19, subsection, 1, Macchina Virtuale (Dijkstra), L827,
20, subsection, 1, Architettura Chiusa, L828,
23, subsection, 1, Architettura Aperta, L829,
25, section, 1, Principali Architetture, L830,
26, subsection, 1, Architettura Client-server, L831,
28, subsection, 1, Principali Architetture Software, L832,
29, subsection, 1, Repository Architecture, L833,
31, subsection, 1, Esempio di Repository Architecture, L834,
32, subsection, 1, Vantaggi dell'architettura a repository, L835,
33, subsection, 1, Svantaggi dell'architettura a repository, L836,
34, subsection, 1, Layered Architecture, L837,
35, subsection, 1, Esempio Layered Architecture, L838,
38, subsection, 1, Architetture three-tier - I, L839,
40, subsection, 1, Vantaggi di architetture three-tier, L840,
43, subsection, 1, Vantaggi di architetture three-tier, L841,
44, subsection, 1, Svantaggi di architetture three-tier, L842
}]{esempi/lezione20.pdf}
\includepdf[pages=-, addtotoc={
1, chapter, 1, Lezione 21: MVC, L91,
2, section, 1, Struttura di un'app software, L92,
3, section, 1, Model-View-Controller, L93,
4, subsection, 1, Struttura e responsabilità, L94,
6, subsection, 1, Interazioni fondamentali, L95,
8, subsection, 1, Considerazioni, L96,
9, subsection, 1, MVC e qualità del software, L97,
11, subsection, 1, MVC ed eventi in Java, L98,
13, subsection, 1, Esempio, L99,
14, subsection, 1, Applicazioni e livelli, L910,
19, subsection, 1, MVC per Classi di Dominio e Widget, L911
}]{esempi/lezione21.pdf}
\end{document}
