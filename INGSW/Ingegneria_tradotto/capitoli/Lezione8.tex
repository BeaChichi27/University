\chapter{Lezione 8: Usabilità e Progettazione Centrata sull'uso umano}

\section{L'ascesa dell'usabilità}
Oggi dobbiamo preoccuparci di costruire sistemi con una buona usabilità
Usabilità: una misura qualitativa della facilità ed efficienza con cui
un essere umano può utilizzare le funzioni e le caratteristiche offerte dal sistema.
Indipendentemente dal tipo di software che stiamo costruendo, una buona usabilità può
fare la differenza tra successo e fallimento, e persino tra vita e morte!

\subsection{Usabilità - App per il grande pubblico}

Se stiamo costruendo una nuova app di social media, un sito web di e-commerce,
o un'altra app per smartphone per fare qualcosa:
\begin{itemize}
    \item Facilità di apprendimento, bassi tassi di errore e soddisfazione soggettiva sono fondamentali
    \item L'uso è discrezionale e (generalmente) la concorrenza è agguerrita
    \item Se gli utenti non riescono a ottenere risultati rapidamente e senza sforzo, rinunceranno e proveranno un fornitore concorrente
\end{itemize}

\subsection{Usabilità - Software professionale}

Per software utilizzato in ambiente professionale
(bancario, assicurativo, gestione della produzione,
prenotazioni, fatturazione utilities, ...)
\begin{itemize}
    \item Il tempo di formazione è un costo, la facilità d'uso è importante
    \item L'internazionalizzazione può essere necessaria
    \item La velocità di esecuzione è importante e l'affaticamento, lo stress e l'esaurimento degli operatori sono preoccupazioni
    \item Ridurre del 10\% il tempo medio di transazione potrebbe significare il 10\% in meno di operatori, il 10\% in meno di postazioni di lavoro, ...
\end{itemize}

\subsection{Usabilità - Sistemi critici per la vita}

Per sistemi critici per la vita come quelli che
controllano il traffico aereo, i reattori nucleari, le utilities energetiche,
i servizi di emergenza, le operazioni militari e le cure cliniche)
\begin{itemize}
    \item I costi elevati dovuti ai lunghi tempi di formazione sono previsti, ma dovrebbero garantire prestazioni rapide e senza errori anche quando gli utenti sono sotto pressione
    \item Errori o ritardi nell'esecuzione possono causare danni gravi!
\end{itemize}

\subsection{Interazione Uomo-macchina (HCI)}
L'HCI è la disciplina che studia come gli esseri umani interagiscono con i computer,
e come progettare interfacce utente efficaci e usabili.

\subsection{Usabilità vs User-friendliness}
Quando i fornitori di computer e software iniziarono a vedere gli utenti come più di
un inconveniente, iniziarono a descrivere i loro sistemi come user friendly
Non è un termine molto buono da usare
\begin{itemize}
    \item \textbf{Inutilmente antropomorfico}: gli utenti hanno bisogno di sistemi che li aiutino a svolgere il loro lavoro e non intralcino. Non hanno bisogno che i sistemi siano amichevoli con loro!
    \item \textbf{Implica che i bisogni degli utenti possano essere descritti lungo una singola dimensione} da sistemi che sono più o meno amichevoli
    \item In pratica, sistemi che sono amichevoli per un utente possono risultare tediosi per altri
\end{itemize}

\section{Definizione di usabilità}
L'usabilità è definita mediante 5 attributi di qualità:
\begin{itemize}
    \item \textbf{Apprendibilità}: Quanto è facile per gli utenti portare a termine compiti basici la prima volta che incontrano il design?
    \item \textbf{Efficienza}: Dopo che gli utenti hanno appreso il design, quanto rapidamente possono eseguire i compiti?
    \item \textbf{Memorabilità}: Quando gli utenti tornano al design dopo un periodo di inutilizzo, quanto facilmente possono riacquistare la padronanza?
    \item \textbf{Errori}: Quanti errori fanno gli utenti, quanto sono gravi questi errori e quanto facilmente possono riprendersi dagli errori?
    \item \textbf{Soddisfazione}: Quanto è piacevole usare il design?
\end{itemize}

\subsection{Apprendibilità}

Probabilmente l'attributo di usabilità più fondamentale
La maggior parte dei sistemi deve essere facile da apprendere
Per alcuni sistemi specializzati, è accettabile che siano difficili da apprendere ma altamente efficienti per utenti esperti
I cosiddetti sistemi "walk-up-and-use" (ad es.: sistemi informativi museali) sono pensati per essere utilizzati una sola volta
Questi sistemi richiedono essenzialmente un tempo di apprendimento zero: gli utenti dovrebbero avere successo la prima volta che li usano!

\subsection{Efficienza d'uso}
L'efficienza si riferisce al livello di performance stabile dell'utente esperto quando la curva di apprendimento si appiattisce
Potrebbero volerci mesi o anni per raggiungere quella fase!
Per misurare l'efficienza:
\begin{itemize}
    \item Decidere una definizione di competenza esperta
    \item Ottenere un campione rappresentativo di utenti con quel livello di competenza
    \item Misurare il tempo che impiegano per completare alcuni compiti tipici
\end{itemize}

\subsection{Memorabilità}
Gli utenti occasionali sono una terza categoria di utenti oltre ai principianti e agli esperti
Gli utenti occasionali utilizzano il sistema in modo intermittente
A differenza dei principianti, non hanno bisogno di apprenderlo da zero, ma hanno bisogno di ricordare come usarlo in base al loro apprendimento precedente
L'uso occasionale tipicamente avviene per:
\begin{itemize}
    \item Software che non fanno parte del lavoro principale di un utente
    \item Software che sono intrinsecamente utilizzati a lunghi intervalli (ad es.: per redigere rapporti annuali)
    \item Software che sono utilizzati solo in circostanze eccezionali
\end{itemize}
Interfacce memorabili sono utili anche per utenti che tornano a utilizzare il sistema dopo essere stati in vacanza, o hanno temporaneamente smesso di usarlo
I miglioramenti nell'apprendibilità rendono un'interfaccia anche facile da ricordare
In principio, tuttavia, l'usabilità del ritorno a un sistema è diversa dall'affrontarlo per la prima volta

\subsection{Errori}
Un errore è qualsiasi azione che non raggiunge l'obiettivo desiderato
Il tasso di errore può essere misurato contando il numero di errori commessi dagli utenti durante l'esecuzione di un compito (come parte di un esperimento per misurare anche altri attributi di usabilità)
\begin{itemize}
    \item Il semplice conteggio degli errori potrebbe essere fuorviante: alcuni errori vengono corretti immediatamente dagli utenti e hanno il solo effetto di rallentarli (riducendo in qualche modo la velocità di transazione)
    \item Altri errori sono più catastrofici per natura: l'utente non se ne accorge, portando a un prodotto di lavoro difettoso; potrebbe essere impossibile riprendersi dall'errore
\end{itemize}
Gli utenti dovrebbero commettere il minor numero possibile di errori quando utilizzano un software
E almeno, dovrebbero commettere pochissimi errori catastrofici, se non nessuno!

\subsection{Soddisfazione soggettiva}
Questo attributo si riferisce a quanto sia piacevole usare il sistema
È particolarmente importante per i sistemi utilizzati su base discrezionale: videogiochi, pittura creativa, ...
Per alcuni di questi sistemi, il loro valore di intrattenimento è più importante della velocità con cui le cose vengono fatte, poiché si potrebbe voler passare un tempo più lungo divertendosi

\subsection{Compromessi nell'usabilità}
Non è sempre possibile massimizzare tutti gli attributi di usabilità simultaneamente
\begin{itemize}
    \item Potrebbero essere necessari compromessi: per evitare errori catastrofici, potremmo progettare un'interfaccia utente meno efficiente, che pone domande extra per assicurarsi che l'utente voglia realmente eseguire una certa azione
    \item In alcuni casi, potremmo ottenere una situazione win-win: l'apprendibilità e l'efficienza d'uso per gli esperti non sono necessariamente in conflitto. Potremmo essere in grado di ottenere il meglio di entrambe le curve di apprendimento, ad esempio includendo acceleratori (es.: scorciatoie da tastiera o hotkeys) nella nostra UI
\end{itemize}

\subsection{Ingegneria dell'usabilità}
Molti progetti di sviluppo software falliscono nel raggiungere i loro obiettivi
Molti di questi fallimenti sono dovuti a scarse comunicazioni tra sviluppatori e clienti e tra sviluppatori e utenti
Ciò si traduce in interfacce utente che costringono gli utenti ad adattare e cambiare il loro comportamento piuttosto che soddisfare le esigenze degli utenti

\section{Progettazione Centrata sull'Uomo (HCD)}
Non è un'attività una tantum in cui l'interfaccia utente viene sistemata prima del rilascio
Un insieme di attività che idealmente si svolgono durante l'intero Ciclo di Vita del Software
\href{https://www.iso.org/standard/77520.html}{ISO 9241-210} definisce la Progettazione Centrata sull'Uomo (HCD)
\begin{itemize}
    \item Gli sviluppatori devono mantenere una prospettiva centrata sull'uomo
    \item Gli utenti devono svolgere un ruolo centrale durante l'intero ciclo di vita
    \item Complementare alle metodologie di progettazione esistenti
    \item Fornisce una prospettiva centrata sull'uomo che può essere integrata in diversi processi di progettazione e sviluppo
\end{itemize}

\subsection{Principi HCD}
La progettazione si basa sulla comprensione esplicita di utenti, compiti e ambienti
Gli utenti sono coinvolti il più possibile nella progettazione e nello sviluppo
La progettazione è guidata e raffinata dalla valutazione centrata sull'utente
Il processo può essere iterativo, se necessario
La progettazione affronta l'intera esperienza utente
Il team di progettazione dovrebbe includere competenze e prospettive multidisciplinari

\subsection{HCD: Comprendere il Contesto e gli Utenti}
Contesto: quali sono i tipi di utilizzo del sistema?
\begin{itemize}
    \item Sistema critico per la vita?
    \item Industriale? Commerciale? Militare? Scientifico?
    \item Intrattenimento?
\end{itemize}
In quale mercato compete il sistema?
\begin{itemize}
    \item Progetto di sviluppo software personalizzato?
    \item Sistema per aziende?
    \item App per il grande pubblico?
\end{itemize}

Utenti: conosci i tuoi utenti (come abbiamo fatto nell'Ingegneria dei Requisiti)!
Personas e scenari (mantenere sempre presenti i bisogni degli utenti)
Ma è necessario considerare anche:
\begin{itemize}
    \item Attributi fisici (età, genere, dimensioni, portata, angoli visivi, ecc.)
    \item Abilità percettive (udito, vista, sensibilità al calore...)
    \item Abilità cognitive (capacità di memoria, livello di lettura, formazione musicale, matematica...)
    \item Posti di lavoro fisici (altezza tavolo, livelli sonori, illuminazione, versione software...)
    \item Tratti di personalità e sociali (preferenze, antipatie, pazienza...)
    \item Diversità culturale e internazionale (lingue, flusso delle finestre di dialogo, simboli...)
    \item Popolazioni speciali, (dis)abilità
\end{itemize}

\subsection{HCD: Progettare per Soddisfare i Requisiti}
Una progettazione appropriata del sistema si basa su una chiara comprensione del contesto e degli utenti!
La produzione di soluzioni di progettazione dovrebbe includere le seguenti sotto-attività:
\begin{itemize}
    \item[a)] progettazione di compiti utente, interazione utente-sistema e interfaccia utente per soddisfare i requisiti utente, tenendo in considerazione l'intera esperienza utente;
    \item[b)] rendere le soluzioni di progettazione più concrete (ad es.: prototipi o mock-up);
    \item[c)] migliorare le soluzioni di progettazione basandosi su valutazioni centrate sull'utente e feedback;
    \item[d)] comunicare le soluzioni di progettazione a coloro che sono responsabili della loro implementazione.
\end{itemize}

\subsection{HCD: Progettare per Soddisfare i Requisiti}
Progettare l'interazione utente-sistema implica decidere come gli utenti svolgeranno i compiti con il sistema piuttosto che descrivere come appare il sistema.
La progettazione dell'interazione dovrebbe includere:
\begin{itemize}
    \item prendere decisioni di alto livello (ad es. concept di progettazione iniziale, risultati essenziali);
    \item identificare compiti e sotto-compiti;
    \item allocare compiti e sotto-compiti all'utente e ad altre parti del sistema;
    \item Es.: il sistema tiene traccia dell'ID di login e ricorda agli utenti, ma gli utenti ricordano la password;
    \item identificare gli oggetti di interazione necessari per il completamento dei compiti;
    \item progettare la sequenza e la tempistica (dinamica) dell'interazione;
    \item progettare l'interfaccia utente per consentire un accesso efficiente agli oggetti di interazione.
\end{itemize}

\subsection{HCI: Valutare la Progettazione}
La valutazione centrata sull'utente (cioè, la valutazione basata sulla prospettiva dell'utente) è un'attività richiesta nell'HCD:
\begin{itemize}
    \item Test basati sull'utente (ad es.: coinvolgendo utenti reali o rappresentativi)
    \item Approccio basato sull'ispezione (verifica di linee guida o requisiti)
\end{itemize}

\subsection{Progettazione Centrata sull'Uomo e SDLC}
L'HCD non richiede alcun processo di progettazione particolare
È complementare alle metodologie di sviluppo esistenti
Ciascuna attività può essere integrata (in misura minore o maggiore) in qualsiasi fase dello sviluppo di un sistema
Ad esempio, l'HCD potrebbe essere applicata nella fase di Ingegneria dei Requisiti in un modello di processo a cascata
\includepdf[pages=-, addtotoc={1,subsection,1,Ancora sul processo di design dell'UI, L1:4,
2, section, 1, Design come scelta, L1:5,
4, subsection, 1, Importanza della critica e dei feedback, L1:6
}]{esempi/lezione8.pdf}