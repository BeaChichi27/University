\chapter{Lezione 5: Use Case completi}
\section{Specificazioni degli Use Case}
Il Diagramma dei Casi d'Uso (UCD) fornisce una panoramica di alto livello dei requisiti funzionali del sistema. Non è sufficientemente dettagliato per stabilire i requisiti di sistema.
Per ogni Casi d'Uso (UC) nell'UCD è necessaria una specifica dettagliata.
L'obiettivo è specificare ogni aspetto e dettaglio dell'interazione, dal punto di vista dell'Attore.
Ogni possibile scenario e variazione dovrebbe essere descritto.

\subsection{Descrizione testuale di un Use Case}
Una descrizione di un caso d'uso generalmente include:
\begin{enumerate}
    \item Una descrizione di ciò che il sistema e gli utenti si aspettano quando il caso d'uso inizia.
    \item Una descrizione del flusso normale degli eventi nel Casi d'Uso (scenario principale).
    \item Una descrizione di ciò che può causare errori e come i problemi risultanti possono essere gestiti.
    \item Una descrizione dello stato del sistema dopo che il Casi d'Uso è completato.
\end{enumerate}

\subsection{Formati di Use Case}
I casi d'uso possono essere scritti in diversi formati e livelli di formalità:
\begin{itemize}
    \item \textbf{Breve:} Riepilogo conciso di un paragrafo, solitamente dello scenario di successo principale.
    \item \textbf{Informale:} Formato a paragrafi informali. Paragrafi multipli che coprono vari scenari.
    \item \textbf{Descrizione Completa (Fully-dressed):} Tutti i passaggi e le variazioni sono scritti in dettaglio e ci sono sezioni di supporto, come precondizioni e garanzie di successo.
\end{itemize}

Le descrizioni Brevi e Informali possono essere utilizzate nelle fasi iniziali della specifica dei requisiti, per avere una rapida idea del soggetto e dell'ambito.
Le descrizioni Complete possono essere sviluppate successivamente, per servire come base per un contratto e specificare in maggiore dettaglio il comportamento del sistema da sviluppare.
\newpage
\subsection{Descrizioni degli Use Case completi}
Sono stati proposti diversi formati per le descrizioni complete dei casi d'uso.
\begin{figure}[htbp!]
    \centering
    \includegraphics[width=0.5\linewidth]{immagini/Lezione5/1.png}
    \caption{Template di Cockburn 1}
\end{figure}
\begin{figure}[htbp!]
    \centering
    \includegraphics[width=0.5\linewidth]{immagini/Lezione5/2.png}
    \caption{Template di Cockburn 2}
\end{figure}
\subsection{Scenari principali ed estensioni}
Lo scenario principale è la sequenza di azioni che si verifica quando tutto nel caso d'uso procede senza intoppi come previsto.

Tuttavia, possono esserci diversi modi per eseguire un caso d'uso:
\begin{itemize}
    \item Gli utenti possono autenticarsi utilizzando il PIN o uno scanner per impronte digitali.
    \item Potrebbe verificarsi un errore in qualche punto.
\end{itemize}

Quando si definisce il comportamento funzionale del sistema, è importante descrivere anche queste sequenze alternative di azioni che possono verificarsi durante l'esecuzione di un caso d'uso.
\begin{itemize}
    \item Ciò viene fatto utilizzando le \textbf{Estensioni}.
    \item Tipicamente, c'è molto più testo nelle Estensioni che nello Scenario Principale.
\end{itemize}
\includepdf[pages=-, addtotoc={1,section,1,Esempio, L1:1}]{esempi/esempioLezione5.pdf}
\section{Validazione dei requisiti}
La validazione punta a dimostrare che i requisiti definiscono il sistema che il cliente desidera veramente.
I costi degli errori nei requisiti sono elevati, quindi la validazione è molto importante.
Correggere un errore nei requisiti dopo la consegna può costare fino a 100 volte il costo della correzione di un errore di implementazione.

\subsection{Controllo dei requisiti}
Ci sono delle domande da fare per il controllo dei requisiti:
\begin{itemize}
    \item \textbf{Validità:} Il sistema fornisce le funzioni che supportano al meglio le esigenze del cliente?
    \item \textbf{Consistenza:} Ci sono conflitti tra i requisiti?
    \item \textbf{Completezza:} Tutte le funzioni richieste dal cliente sono incluse?
    \item \textbf{Realismo:} I requisiti possono essere implementati considerando il budget e la tecnologia disponibili?
    \item \textbf{Verificabilità:} I requisiti possono essere verificati?
\end{itemize}

\subsection{Tecniche di validazione dei requisiti}
Esistono delle tecniche per validare i requisiti:
\begin{itemize}
    \item \textbf{Revisioni dei requisiti:} Analisi manuale sistematica dei requisiti.
    \item \textbf{Prototipazione:} Utilizzo di un modello eseguibile semplificato del sistema per verificare i requisiti. Oppure prototipazione visiva (ad esempio, utilizzando mockup/wireframe).
    \item \textbf{Generazione di test case:} Sviluppo di test per i requisiti per verificarne la testabilità.
\end{itemize}

Dovrebbero essere effettuate revisioni regolari durante la formulazione della definizione dei requisiti.
Sia il personale del cliente che quello dell'appaltatore dovrebbero essere coinvolti nelle revisioni.
Le revisioni possono essere formali (con documenti completati) o informali. Una buona comunicazione tra sviluppatori, clienti e utenti può risolvere i problemi in una fase iniziale.