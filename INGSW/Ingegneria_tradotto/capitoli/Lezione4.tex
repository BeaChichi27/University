\chapter{Lezione 4: Diagrammi Use Case}
\section{Specificazione dei requisiti}
\subsection{Requisiti e design}

In linea di principio, i requisiti dovrebbero dichiarare \textit{cosa} il sistema dovrebbe fare e il design dovrebbe descrivere \textit{come} lo fa.
In pratica, requisiti e design sono inseparabili:
\begin{itemize}
    \item I requisiti possono essere strutturati e organizzati sulla base di un'architettura di sistema di alto livello.
    \item Il sistema può inter-operare con altri sistemi che generano requisiti di design.
    \item L'uso di un'architettura specifica per soddisfare requisiti non funzionali può essere esso stesso un requisito di dominio.
\end{itemize}

\subsection{Specifica dei requisiti}

Il processo di scrittura dei requisiti utente e di sistema in un documento di specifica dei requisiti.
\begin{itemize}
    \item I \textbf{requisiti utente} devono essere comprensibili da utenti finali e clienti che non hanno una formazione tecnica.
    \item I \textbf{requisiti di sistema} sono requisiti più dettagliati e possono includere informazioni più tecniche.
    \item I requisiti possono far parte di un contratto per lo sviluppo del sistema. È quindi importante che questi siano il più completi e dettagliati possibile.
\end{itemize}

Sono possibili diversi approcci:
\begin{itemize}
    \item \textbf{Linguaggio Naturale:} esprimere i requisiti come frasi numerate in linguaggio naturale. Ogni frase dovrebbe esprimere un singolo requisito.
    \item \textbf{Linguaggio Naturale Strutturato:} utilizzare un modulo o un template standardizzato.
    \item \textbf{Notazioni e Modelli Semi-Formali:} Diagrammi UML dei Casi d'Uso (Use Case Diagrams) e altri modelli di dominio, tipicamente integrati da annotazioni in linguaggio naturale.
    \item \textbf{Specifica Formale:} Queste notazioni si basano su concetti matematici come macchine a stati finite e infinite, logiche temporali, ecc.
\end{itemize}

Diversi approcci sono utilizzati in diversi domini:
\begin{itemize}
    \item Nell'ingegneria di \textbf{sistemi critici per la sicurezza} (safety-critical systems), è comune utilizzare specifiche formali e linguaggio naturale strutturato.
    \item Nell'ingegneria di un'applicazione che listi le cose da fare (to-do list app), si potrebbe usare il linguaggio naturale non strutturato per esprimere i requisiti.
    \item Nell'ingegneria di un sistema informativo di medie-grandi dimensioni, sfruttare notazioni e modelli semi-formali potrebbe essere un buon compromesso.
\end{itemize}

\section{Specificazioni del linguaggio naturale (NL)}
I requisiti sono scritti come frasi in linguaggio naturale, eventualmente integrate da diagrammi e tabelle.
Questo approccio è utilizzato per scrivere i requisiti perché è espressivo, intuitivo e universale.
\begin{itemize}
    \item Ciò significa che i requisiti possono essere compresi da utenti e clienti.
\end{itemize}

Può esprimere sia requisiti funzionali che non funzionali.
Definire un formato ``standard'' e utilizzarlo per tutti i requisiti.
Utilizzare il linguaggio in modo coerente. Usare \textit{shall} per i requisiti obbligatori, \textit{should} per i requisiti desiderabili.
Utilizzare la formattazione del testo per identificare le parti chiave del requisito.
Evitare l'uso di gergo tecnico informatico.
Includere una spiegazione (razionale) del motivo per cui un requisito è necessario.
\newline
Ci sono dei problemi usando il linguaggio naturale:
\begin{itemize}
    \item \textbf{Mancanza di chiarezza:} È difficile essere precisi senza rendere il documento difficile da leggere.
    \item \textbf{Confusione tra i requisiti:} I requisiti funzionali e non funzionali tendono a essere mescolati.
    \item \textbf{Agglomerazione dei requisiti:} Diversi requisiti distinti possono essere espressi insieme in un'unica affermazione.
\end{itemize}
\section{Diagrammi Use Case}
Gli Use Case (Casi d'Uso) sono un modo per descrivere le interazioni tra utenti e un sistema utilizzando un modello grafico e testo in linguaggio naturale strutturato.
Sono una parte fondamentale del Linguaggio di Modellazione Unificato (UML) e possono essere utilizzati per rappresentare l'insieme dei requisiti funzionali di un sistema.

I casi d'uso identificano:
\begin{itemize}
    \item \textbf{Attori:} Categorie di utenti (non necessariamente umani) del sistema.
    \item \textbf{Casi d'Uso:} Tipi di interazioni (o funzionalità) offerte dal sistema.
\end{itemize}

Informazioni aggiuntive sulle interazioni possono essere fornite come descrizioni testuali (strutturate) o per mezzo di uno o più modelli semi-formali (ad es.: Diagrammi di Sequenza UML o Diagrammi degli Stati).
I casi d'uso hanno le seguenti caratteristiche:
\begin{itemize}
    \item Sono un modo per supportare la comunicazione con il cliente per definire le funzionalità del sistema. Dovrebbero essere il più semplici possibile.
    \item Non definiscono \textit{come} il sistema è implementato, ma \textit{cosa} il sistema dovrebbe fare dal punto di vista degli utenti (il sistema è una scatola nera).
    \item I casi d'uso sono spesso documentati utilizzando un Diagramma dei Casi d'Uso (UCD) di alto livello.
\end{itemize}

\subsection{Attori}
Gli attori sono rappresentati utilizzando figure stilizzate (stick figures).
Rappresentano entità esterne che interagiscono con il \textbf{Sistema in Sviluppo} (SUD, System Under Development).
\begin{itemize}
    \item Classi di utenti (umani).
    \item Altri sistemi.
    \item L'ambiente fisico.
\end{itemize}
Ogni attore ha un nome univoco.
Gli attori sono più granulari (coarse-grained) delle Personas: un singolo attore può essere associato a multiple Personas.
\subsection{Euristica per identificare gli attori}
I casi d'uso sono rappresentati come ellissi denominate.
Corrispondono a funzionalità offerte dal sistema, fornendo un qualche beneficio o utilità agli attori.
I casi d'uso modellano i requisiti funzionali.
Un caso d'uso astrae molti possibili scenari (sequenze di azioni) per una determinata funzionalità.
\begin{itemize}
    \item Uno scenario può essere visto come un'istanza di un caso d'uso.
    \item Un caso d'uso rappresenta una classe di scenari che mirano a utilizzare la stessa funzionalità.
\end{itemize}

Per identificare gli attori, ci si può chiedere:
\begin{itemize}
    \item Quali gruppi di utenti sono supportati dal Sistema in Sviluppo (SUD) nel loro lavoro?
    \item Quali gruppi di utenti eseguono le principali funzionalità offerte dal SUD?
    \item Quali gruppi di utenti svolgono le funzioni secondarie del SUD, come l'amministrazione?
    \item Il SUD interagirà con sistemi o software esterni? Ogni sistema o software esterno con cui il SUD interagisce sarà un attore.
\end{itemize}
Gli attori non corrispondono a una singola entità, ma rappresentano piuttosto una classe di utenti che può avere lo stesso ruolo: un utente può ricoprire diversi ruoli nello stesso sistema.
\subsection{Associazioni}
Oltre ad attori e casi d'uso, i diagrammi dei casi d'uso includono
diversi tipi di relazioni tra di essi.

Un'associazione tra un attore e un caso d'uso indica che l'attore
può eseguire il caso d'uso. Graficamente, è rappresentata come una linea che collega un attore a un caso d'uso.
\begin{figure}[htbp!]
    \centering
    \includegraphics[width=0.5\linewidth]{immagini/Lezione4/1.png}
    \caption{Associazioni}
\end{figure}
\subsection{Attori secondari}
Un caso d'uso può essere associato a più attori.
La semantica è che più attori devono collaborare in qualche modo
per eseguire quel caso d'uso.

I diagrammi dei casi d'uso UML non includono meccanismi per specificare
come diversi attori sono coinvolti in un caso d'uso.

Le modalità di interazione e le diverse responsabilità possono essere
specificate con descrizioni aggiuntive.
\begin{figure}[htbp!]
    \centering
    \includegraphics[width=0.5\linewidth]{immagini/Lezione4/2.png}
    \caption{Cardinalità tra più attori}
\end{figure}
\subsection{Generalizzazioni degli attori}
La generalizzazione tra attori può essere applicata quando un attore è un sotto-tipo di un altro attore.
Stesso concetto e notazione come nei Diagrammi delle Classi UML. Rappresentata graficamente come una freccia con una testa vuota.
L'attore specializzato può eseguire qualsiasi caso d'uso che il genitore può eseguire.
La generalizzazione tra casi d'uso è destinata ad essere utilizzata quando un caso d'uso è una specializzazione di un altro. A differenza delle dipendenze \textless\textless extend\textgreater\textgreater{}, non ci sono punti precisi in cui i casi d'uso specializzati deviano dal genitore. Le specializzazioni possono essere molto diverse rispetto ai casi d'uso parent. La notazione è la notazione UML standard per la specializzazione.
\begin{figure}[htbp!]
    \centering
    \includegraphics[width=0.5\linewidth]{immagini/Lezione4/3.png}
    \caption{Generalizzazione}
\end{figure}
\newpage
\subsection{La relazione \textless\textless include\textgreater\textgreater{}}
La relazione \textless\textless include\textgreater\textgreater{} è destinata ad essere utilizzata quando ci sono parti comuni del comportamento di due o più Casi d'Uso.
Questa parte comune viene quindi estratta in un Caso d'Uso separato, da includere in tutti i Casi d'Uso base che hanno questa parte in comune.
Poiché l'uso principale della relazione \textless\textless include\textgreater\textgreater{} è per il riutilizzo di parti comuni, ciò che rimane in un Caso d'Uso base di solito non è completo di per sé ma dipende dalle parti incluse per essere significativo.
Questo si riflette nella direzione della relazione, che indica che il Caso d'Uso base dipende dall'aggiunta ma non viceversa.
Questa relazione può essere utile per:
\begin{itemize}
    \item Scomporre un'interazione complessa in interazioni più piccole e gestibili.
    \item Fattorizzare sequenze comuni di passaggi tra diversi casi d'uso.
\end{itemize}
\begin{figure}[htbp!]
    \centering
    \includegraphics[width=0.5\linewidth]{immagini/Lezione4/4.png}
    \caption{include}
\end{figure}

\subsection{La relazione \textless\textless extend\textgreater\textgreater{}}
La relazione \textless\textless extend\textgreater\textgreater{} è destinata ad essere utilizzata quando c'è un comportamento aggiuntivo che può essere aggiunto, possibilmente in modo condizionale, al comportamento definito in uno o più Casi d'Uso.
Il Caso d'Uso esteso è definito indipendentemente dal Caso d'Uso che estende ed è significativo indipendentemente da esso.
D'altra parte, il Caso d'Uso che estende tipicamente definisce un comportamento che potrebbe non essere necessariamente significativo da solo.
\begin{figure}[htbp!]
    \centering
    \includegraphics[width=0.5\linewidth]{immagini/Lezione4/5.png}
    \caption{extend}
\end{figure}
\newpage
Esistono anche dei punti d'estensione per cui si stabilisce nel diagramma Use Case dove può essere inserito il comportamento di un'extend, possono essere utili quando lo Use Case può essere esteso in più punti.
\begin{figure}[htbp!]
    \centering
    \includegraphics[width=0.5\linewidth]{immagini/Lezione4/6.png}
    \caption{Extension point}
\end{figure}
\section{Errori da principianti}
I casi d'uso dovrebbero fornire qualche beneficio all'attore, aiutare l'attore a completare il suo lavoro o raggiungere qualche obiettivo.

\begin{itemize}
    \item Tipicamente, i nomi dei Casi d'Uso dovrebbero includere un verbo.
    \item Tipicamente, i nomi degli Attori dovrebbero essere sostantivi.
\end{itemize}

Se due attori sono associati allo stesso caso d'uso (con una cardinalità diversa da zero), significa che i due attori sono coinvolti (e necessitano di collaborare) in ogni istanza (scenario) di quel caso d'uso.\\
\textbf{Non} significa che entrambi gli attori possono eseguire quel caso d'uso in modo indipendente!

\subsection*{Attenzione alle generalizzazioni improprie}
\begin{itemize}
    \item Ogni attore dovrebbe avere i propri casi d'uso.
    \item Gli attori specializzati possono già eseguire tutti i casi d'uso dei loro antenati.
    \item Se gli attori specializzati non hanno alcuni casi d'uso propri, la generalizzazione potrebbe essere inutile, oppure potrebbero mancare alcuni casi d'uso.
\end{itemize}

\subsection*{Note importanti}
\begin{itemize}
    \item La relazione \textless\textless include\textgreater\textgreater{} non è un buon modo per rappresentare relazioni temporali.
    \item I diagrammi dei casi d'uso non dovrebbero diventare troppo complessi e confusi.
    \item Utilizzare le relazioni tra casi d'uso e le generalizzazioni tra attori con moderazione.
    \item La modellazione dovrebbe essere a un sufficiente grado di astrazione.
    \item Bisogna essere ordinati (cercare di evitare linee che si intersecano, ecc.).
    \item Un diagramma complesso è indice di una cattiva analisi.
\end{itemize}
\section{Esercizi}
\includepdf[pages=-]{esercizi/Lezione4.pdf}