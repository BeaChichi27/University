\chapter{Lezione 13: Modelli e teorie in HCI}
\subsection{Cos'è un modello?}
Un modello è una semplificazione (astrazione) della realtà
\begin{itemize}
\item Una mappatura perfetta della realtà non è un modello (e non è utile!)
\item Precisione vs generalità
\end{itemize}
«Tutti i modelli sono sbagliati, ma alcuni sono utili»
I modelli ci permettono di:
\begin{itemize}
\item Rappresentare e ragionare su (aspetti di) fenomeni di interesse
\item Anticipare (prevedere) risultati
\end{itemize}

\section{Modelli e teorie HCI}
Nell'HCI, i modelli e le teorie mirano a spiegare come gli esseri umani interagiscono con
i computer.
I modelli e le teorie HCI possono essere classificati come:
\begin{itemize}
\item \textbf{Descrittivi}: mirano a sviluppare una terminologia coerente e tassonomie utili
\item \textbf{Esplicativi}: descrivono sequenze di eventi, possibilmente con relazioni causali. Es.: le sette fasi dell'azione di Norman
\item \textbf{Predittivi}: mirano a consentire il confronto di alternative di design basate su
previsioni numeriche di velocità o errori
\item \textbf{Prescrittivi}: offrono linee guida ai designer per prendere decisioni
\end{itemize}

\subsection{Il modello a tre stati dell'input grafico}
Proposto da Will Buxton nel 1990 [1]
Descrive l'input grafico con dispositivi di puntamento
Diverse tecnologie (mouse, trackpad, tablet con stilo, ecc...)
\begin{figure}[htbp]
\centering
\includegraphics[width=0.5\textwidth]{immagini/lezione13/1.png}
\caption{Descrizione di Buxton}
\end{figure}

\begin{figure}[htbp]
\centering
\includegraphics[width=0.5\textwidth]{immagini/lezione13/2.png}
\caption{Operazioni di trascinamento con un mouse (a sinistra) e con un touchpad lift-and-tap (a destra) (tratto da [1])}
\end{figure}

\subsection{Il modello di Guiard dell'abilità bimanuale}
Molte interazioni sono asimmetriche rispetto alla mano sinistra/destra
Il modello di Guiard descrive ruoli e azioni delle mani preferita/non preferita
Non preferita:
\begin{itemize}
\item Guida la mano preferita
\item Definisce il quadro di riferimento spaziale per la mano preferita
\item Esegue movimenti grossolani
\end{itemize}
Preferita:
\begin{itemize}
\item Segue la mano non preferita
\item Lavora all'interno del quadro di riferimento stabilito dalla mano non preferita
\item Esegue movimenti fini
\end{itemize}

\begin{figure}[htbp]
\centering
\includegraphics[width=0.5\textwidth]{immagini/lezione13/3.png}
\caption{Interazione a due mani. (Schizzo di Shawn Zhang)}
\end{figure}

L'artista acquisisce il modello con la mano sinistra (la mano non preferita guida).
Il modello viene manipolato sullo spazio di lavoro (movimento grossolano, definisce il quadro di riferimento).
Lo stilo viene impugnato nella mano destra (la mano preferita segue) e portato in prossimità del modello (lavora all'interno del quadro di riferimento stabilito dalla mano non preferita).
Ha luogo lo schizzo (la mano preferita esegue movimenti precisi).
\section{Il Modello del Processore Umano (MHP)}
È un modello predittivo a priori, può fornire approssimazioni delle azioni dell'utente prima che utenti reali siano coinvolti nel processo di test (e prima che l'interfaccia utente sia anche solo implementata!)
Un essere umano è modellato da un insieme di memorie e processori che
funzionano secondo un insieme di principi
Modello discreto e sequenziale
Processori percettivi, cognitivi e
motori
Diversi tipi di memoria
Parametri del modello:
\begin{itemize}
\item Tempi di ciclo del processore: $\tau$
\item Tempo di decadimento della memoria: $\delta$
\item Capacità della memoria: $\mu$
\end{itemize}

\begin{figure}[H]
\centering
\includegraphics[width=0.5\textwidth]{immagini/lezione13/4.png}
\caption{Parametri del modello}
\end{figure}

\subsection{MPH: Memorie}
\textbf{Memoria di Lavoro} (WM) è un sottoinsieme di elementi «attivati» (chunk)
dalla \textbf{Memoria a Lungo Termine} (LTM)
\begin{itemize}
\item I chunk possono essere composti da unità più piccole come lettere in una parola
\item Un chunk potrebbe anche consistere in diverse parole, come in una frase ben nota
\end{itemize}
$\mu_{LTM} = \infty$ e $\delta_{LTM} = \infty$
$\mu_{WM} = 7 \pm 2$ chunk
$\delta_{WM} = 7 \pm 2$ s
\begin{itemize}
\item Il tempo di decadimento per WM varia ampiamente in base al numero di chunk memorizzati
\item $\delta_{WM}$ 1 chunk = $73 \pm 53$ s
\item $\delta_{WM}$ 3 chunk = $7 \pm 2$ s
\end{itemize}

\subsection{Percezione di stimoli congruenti e incongruenti}
I tempi di percezione sono influenzati anche dalla natura degli stimoli
Per esempio, c'è un ritardo significativo nel tempo di reazione (effetto Stroop)
tra stimoli congruenti e incongruenti.

\includepdf[pages=-, addtotoc={1, subsection, 1, Perché usare l'MPH?, L5:1, 2, subsection, 1, Esempio, L5:2}]{esempi/lezione13.pdf}

\section{Modello GOMS}
Goals (Obiettivi), Operators (Operatori), Methods (Metodi), Selection rules (Regole di selezione)
Assunzioni:
\begin{itemize}
\item L'interazione con un sistema è risoluzione di problemi
\item Scomporre l'interazione in sottoproblemi
\item Determinare gli obiettivi per «affrontare» il problema
\item Specificare la sequenza di operazioni utilizzate per raggiungere gli obiettivi
\item I valori di tempo possono essere assegnati a ciascuna operazione
\end{itemize}

\begin{enumerate}
\item[a] Goals (Obiettivi): Ciò che l'utente vuole ottenere (es., "avviare la funzione di utilità corretta").
\item[b] Methods (Metodi): Possibili sequenze alternative di operatori utilizzate per raggiungere l'obiettivo.
\item[c] Selection rules (Regole di selezione): Criteri per scegliere tra diversi metodi.
\item[d] Operators (Operatori): Azioni di base eseguite dall'utente (es., "spostare il mouse", "fare clic", "controllare l'impostazione").
\end{enumerate}

\subsection{GOMS: Esempio}
Obiettivo: eliminare una parola in un editor di documenti
Regola di selezione: se il cursore è alla fine della parola da eliminare, usa il Metodo A, altrimenti
usa il Metodo B

\begin{enumerate}
\item Metodo A:
\begin{itemize}
\item Premere il tasto «backspace».
\item Controllare se la parola è stata eliminata e tornare all'operazione precedente se necessario.
\end{itemize}
\item Metodo B:
\begin{itemize}
\item Spostare il cursore del mouse sulla parola.
\item Eseguire un doppio clic.
\item Premere il tasto «backspace».
\end{itemize}
\end{enumerate}

\subsection{Keyboard Level Model (KLM)}
KLM è una delle varianti più semplici di GOMS
Si concentra sul comportamento osservabile: Tasti, movimenti del mouse, ...
Assume prestazioni prive di errori
Operatori comuni e i tempi tipici corrispondenti:
\begin{itemize}
\item K (Keystroke - Pressione tasto): 0,2 secondi (200 ms)
\item P (Pointing with Mouse - Puntamento con il mouse): 1,1 secondi (1100 ms)
\item B (Pressing/holding/releasing mouse button - Premere/tenere/rilasciare il pulsante del mouse): 0,1 secondi (100 ms)
\item H (Homing Hands - Posizionamento delle mani): 0,4 secondi (400 ms)
\item M (Mental Preparation - Preparazione mentale): 1,2 secondi (1200 ms)
\item R (System Response - Risposta del sistema): Variabile; tipicamente intorno a 0,1 secondi (100 ms)
\end{itemize}

\includepdf[pages=-, addtotoc={1, subsection, 1, Esempio KLM: rimozione di un file, L5:3}]{esempi/lezione13-1.pdf}

\section{La Legge di Potenza della Pratica}
Allen Newell (scienziato cognitivo) negli anni '80 analizzò i tempi di reazione
per una varietà di compiti in esperimenti di apprendimento
Notò che le curve di apprendimento ottenute in questi studi hanno una
forma molto simile: quella di una \textbf{legge di potenza}
\begin{itemize}
\item Il tempo necessario per completare un compito dopo n prove ($T_n$) è vicino al
tempo necessario per completare quel compito la prima volta ($T_1$) moltiplicato per $n^{-a}$
$T_n \approx T_1 \cdot n^{-a}$
\item $a$ è un parametro compreso tra 0,2 e 0,6 (generalmente $\sim$0,4)
\end{itemize}

Un utente ha impiegato 5 secondi per eseguire un
determinato compito la prima volta che è stato
esposto alla nuova interfaccia utente che abbiamo sviluppato
Quante ripetizioni sarebbero necessarie a quell'utente
per essere in grado di eseguire il
compito in 2 secondi o meno?
\begin{itemize}
\item Possiamo calcolare una stima con la legge di potenza della pratica: $T_n \approx T_1 \cdot n^{-a}$
\item Risolviamo per $n$, assumendo $a = 0,4$
\item $2 s \leq 5 s \cdot n^{-0,4}$
\item Per $n = 10$, otteniamo che $T_n \approx 1,99 s$
\end{itemize}

\section{Legge di Hick}
La legge di Hick descrive il tempo necessario a una persona per prendere una decisione
tra un insieme di possibili scelte.
La legge di Hick afferma che il tempo $T$ necessario per raggiungere una decisione aumenta
logaritmicamente con il numero di scelte.
Nel caso di alternative ugualmente probabili:
$T = a + b \cdot \log_2(n + 1)$
\begin{itemize}
\item $n$ è il numero di scelte
\item $a$ e $b$ sono parametri che dipendono dalle condizioni del contesto (es.: il modo
in cui le scelte sono presentate, la familiarità dell'utente,...)
\end{itemize}

\subsection{Applicazione della Legge di Hick}
Quale modo è più veloce per selezionare tra 64 opzioni?
\begin{itemize}
\item Menu a un livello 1x64
$T = a + b \cdot \log_2 64 = a + 6b$
\item Menu a due livelli 4x16
$T = a + b \cdot \log_2 4 + a + b \cdot \log_2 16 = 2a + 6b$
\item Menu a due livelli 8x8
$T = 2 \cdot (a + b \cdot \log_2 8) = 2a + 6b$
\item Menu a tre livelli 4x4x4
$T = 3 \cdot (a + b \cdot \log_2 4) = 3a + 6b$
\item Menu a sei livelli 2x2x2x2x2x2
$T = 6 \cdot (a + b \cdot \log_2 2) = 6a + 6b$
\end{itemize}

\section{Legge di Fitts}
Modella il tempo per acquisire obiettivi nei movimenti mirati
\begin{itemize}
\item Raggiungere un controllo in una cabina di pilotaggio
\item Muoversi attraverso un cruscotto
\item Estrarre un elemento difettoso dal nastro trasportatore
\item Fare clic su icone utilizzando un mouse
\end{itemize}

\subsection{Legge di Fitts – Indice di Difficoltà (ID)}
L'indice di difficoltà di un compito di acquisizione di un obiettivo è definito come
$ID = \log_2(A/W + 1)$
\begin{itemize}
\item $A$ è l'Ampiezza del movimento (distanza dall'inizio all'obiettivo)
\item $W$ è la Larghezza dell'obiettivo (variabilità ammissibile)
\end{itemize}

\begin{figure}[htbp!]
\centering
\includegraphics[width=0.5\textwidth]{immagini/lezione13/5.png}
\caption{Ampiezza e Larghezza}
\end{figure}

\subsection{Legge di Fitts – Tempi di Movimento}
I Tempi di Movimento (MT) dipendono dall'Indice di Difficoltà ID
$MT = a+b \cdot ID = a+b \cdot \log_2(A/W + 1)$
I tempi di movimento dipendono anche dal sistema, dispositivo di puntamento, utente...
Possono essere adattati a casi specifici con parametri non negativi $a$ e $b$
È l'equazione di una linea retta ($y = mx + c$), dove $b$ è il gradiente
MT aumenta linearmente con l'ID

\subsection{Legge di Fitts: Applicazioni}
Se dobbiamo ridurre il tempo necessario per eseguire un'azione di ricerca di un obiettivo
O riduciamo l'Ampiezza del movimento (avvicinare l'obiettivo)
O aumentiamo la Larghezza dell'obiettivo
O potremmo lavorare su $a$ e $b$

\begin{figure}[htbp]
\centering
\includegraphics[width=0.5\textwidth]{immagini/lezione13/6.png}
\caption{Variazioni di ID}
\end{figure}

\includepdf[pages=-, addtotoc={6, section, 1, Letture e referenze, L5:4}]{esempi/lezione13-2.pdf}