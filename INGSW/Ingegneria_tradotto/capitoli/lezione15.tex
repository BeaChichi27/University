\chapter{Lezione 15: Teoria dei colori, tipografia e psicologia di Gestalt}

\section{Percezione del colore tramite i coni}
Non tutti i coni sono uguali. Esistono 3 tipi diversi, con fotopigmenti specializzati per percepire il colore: blu, verde e rosso. Ogni tipo di cono è sensibile a una diversa banda dello spettro luminoso. Il rapporto di stimolazione neurale tra i tre tipi ci dà una percezione continua dei colori.

I tipi di coni non sono distribuiti uniformemente al centro della retina: principalmente rossi, pochissimi blu. Sensibilità limitata alle lunghezze d'onda corte, alta sensibilità a quelle lunghe. Pochi coni blu al centro della retina, è più difficile mettere a fuoco piccoli oggetti blu. Con l'età, il cristallino tende ad assorbire più lunghezze d'onda corte, riducendo ulteriormente la sensibilità ai blu.

\section{Modelli di colore}
I modelli di colore sono modelli matematici astratti che descrivono come i colori possono essere rappresentati tramite tuple di numeri. Due schemi di colore ampiamente utilizzati sono:
\begin{itemize}
    \item CMYK
    \item RGB
\end{itemize}

\subsection{Modello di colore CMYK}
Il CMYK è un modello di colore sottrattivo. Utilizza: Ciano, Magenta, Giallo, Key (nero). È generalmente usato per materiali stampati. Utilizza pigmenti d'inchiostro per mostrare il colore. I colori risultano dalla luce riflessa.

\subsection{Modello di colore RGB}
L'RGB è un modello di colore additivo. Utilizza Rosso, Verde, Blu. È pensato per i display dei computer. Utilizza la luce per mostrare il colore. Il colore risulta dalla luce emessa. I display non possono produrre diversi canali di colore nella stessa posizione: la griglia dei pixel è tipicamente divisa in regioni di colore singolo (subpixel), che contribuiscono al colore visualizzato quando osservate a distanza.

\section{Dimensioni percettive del colore}
HSL (Hue, Saturation, Lightness) è una delle rappresentazioni più comuni dei colori nel modello RGB.

\subsection{Hue}
La tonalità (\textit{Hue}) è la lunghezza d'onda dominante in un colore. In HSL, è un valore tra 0 e 360:
\begin{itemize}
    \item 0 (e 360) è rosso
    \item 60 è giallo
    \item 120 è verde
    \item 180 è ciano
    \item 240 è blu
    \item 300 è magenta
\end{itemize}

\subsection{Saturazione}
Indica quanto il colore è «intenso».
\begin{itemize}
    \item Tipicamente è una percentuale tra 0 e 100
    \item 100\% significa che il colore è brillante e puro
    \item 80\% significa che il colore è miscelato con il 20\% di grigio
    \item 0\% significa che si ottiene solo il grigio
\end{itemize}

\subsection{Luminosità}
La luminosità (\textit{Lightness}) indica quanta luce ha il colore.
\begin{itemize}
    \item 0\% è molto scuro (nero)
    \item 50\% è a metà, né troppo scuro né troppo brillante
    \item 100\% è molto brillante (bianco)
\end{itemize}

\subsection{Rappresentazione HSL}
HSL è un modello a coordinate cilindriche. Definendo Hue, Saturation e Lightness, si identifica un punto nello spazio cilindrico.
\begin{figure}[htbp!]
    \centering
    \includegraphics[width=0.5\textwidth]{immagini/lezione15/1.png}
    \caption{Rappresentazione cilindrica di HSL}
\end{figure}

\subsection{Tinte, sfumature e toni}
\begin{itemize}
    \item Una tinta (\textit{tint}) è una miscela di un colore con il bianco, aumentando la luminosità
    \item Una sfumatura (\textit{shade}) è una miscela con il nero, diminuendo la luminosità
    \item Un tono (\textit{tone}) è una miscela con il grigio
\end{itemize}

\section{Uso dei colori nel design delle interfacce}
Il colore è uno strumento potente nell’arsenale di un designer di UI.
\begin{itemize}
    \item Può far risaltare alcuni elementi (vedremo di più su questo alla fine della lezione)
    \item Può trasmettere significato (ma attenzione alle differenze regionali!)
\end{itemize}

Quindi, più colori ci sono, meglio è?
\begin{itemize}
    \item Non proprio. Una UI dovrebbe generalmente includere non più di 6 colori diversi.
    \item Corollario: scegli una palette e sii coerente!
\end{itemize}

Una buona regola pratica è la regola del 60-30-10:
\begin{itemize}
    \item 60\% colore primario
    \item 30\% colore secondario
    \item 10\% colore d’accento (per le parti che vogliamo far risaltare)
\end{itemize}

\subsection{Teoria del colore}
Non tutti i colori «stanno bene» insieme. Possiamo usare le armonie cromatiche per costruire palette (insiemi di colori) per la nostra UI che non siano in contrasto tra loro.
\begin{itemize}
    \item Monocromatica
    \item Analoghi
    \item Complementari
    \item Complementari suddivisi
    \item Triadica
\end{itemize}

\includepdf[pages=-, addtotoc={
    1, subsection, 1, Armonie monocromatiche, L7:1,
    2, subsection, 1, Armonie analoghe, L7:2,
    3, subsection, 1, Armonie complementari, L7:3,
    4, subsection, 1, Armonie complementari suddivise, L7:4,
    5, subsection, 1, Armonie triadiche, L7:5,
    6, subsection, 1, Riferimenti: Armonie cromatiche, L7:6
}]{esempi/lezione15.pdf}

\subsection{Psicologia del colore}
La mente umana può reagire inconsciamente ai colori (psicologia del colore).
\begin{itemize}
    \item Il nero è associato a eleganza, potere e autorità
    \item Il blu è percepito come autorevole, affidabile, degno di fiducia
    \item Il rosso può essere associato a passione, desiderio, amore, energia, pericolo
    \item Il verde può essere associato a natura, freschezza, serenità, salute, denaro
\end{itemize}
Non ci sono molte ricerche che dimostrano l’effetto reale di un colore sulle emozioni.
Inoltre, tieni presente che esistono differenze regionali! In Cina, il rosso è il colore associato al denaro. Negli Stati Uniti, è il verde.

\section{Tipografia}
\subsection{L’ipotesi dei serif}
I caratteri con grazie (\textit{serif}) \textbf{sono più facili da leggere} – e quindi preferibili per lunghi testi – perché le grazie forniscono ancoraggi che guidano l’occhio del lettore.
I font senza grazie (\textit{sans serif}) mancano di questi ancoraggi e sono quindi meno adatti per lunghi testi.
In pratica, le differenze individuali superano qualsiasi effetto della presenza/assenza di grazie, cioè alcune persone leggono più velocemente di altre.

\subsection{Impatto della tipografia nelle UI}
La tipografia può essere un buon modo per veicolare messaggi nelle interfacce.
\begin{itemize}
    \item Leggere non significa solo riconoscere sequenze di lettere
    \item La tipografia può trasmettere messaggi aggiuntivi:
    \begin{itemize}
        \item «Questo è pensato per essere facile da leggere»
        \item «Questo è un messaggio giocoso»
        \item «Questo può essere percepito come codice sorgente»
    \end{itemize}
\end{itemize}
La leggibilità è un aspetto importante dell’usabilità.
\begin{itemize}
    \item Gli utenti devono leggere etichette e informazioni nelle nostre UI
    \item La tipografia gioca un ruolo importante nella facilità di lettura. Anche l’aspetto delle parole può essere importante.
\end{itemize}

\subsection{Effetto di superiorità della parola}
A volte, i lettori riconoscono una parola dalla sua forma, prima ancora di riconoscere le lettere che la compongono.
Questo è chiamato «Effetto di superiorità della parola» in psicologia cognitiva.
Le persone generalmente leggono il testo latino minuscolo più velocemente di quello maiuscolo.
Non usare il maiuscolo per lunghi testi!

\section{Teoria della percezione Gestalt}
\textbf{La teoria della percezione Gestalt} (nota anche come psicologia della Gestalt) si concentra su come la mente umana elabora le informazioni visive.
La psicologia della Gestalt si riferisce all’idea di insieme unificato.
\begin{itemize}
    \item Generalmente percepiamo qualcosa di diverso dalla semplice somma degli elementi che vediamo
    \item Diamo significato alla somma delle parti piuttosto che ai singoli elementi
\end{itemize}

Quindi, cosa vedi nell’immagine qui sotto?
\begin{itemize}
    \item Un triangolo bianco sopra
    \item Un triangolo con bordi neri sotto
    \item Tre cerchi neri parzialmente coperti
\end{itemize}
Perché non vediamo solo un insieme di linee e macchie?
La nostra mente tende a completare oggetti incompleti e a vedere connessioni tra elementi in base alla loro apparenza o posizione relativa.

\begin{figure}[htbp!]
    \centering
    \includegraphics[width=0.5\textwidth]{immagini/lezione15/2.png}
    \caption{Gestalt in azione}
\end{figure}

\subsection{Psicologia della Gestalt}
La psicologia della Gestalt studia queste tendenze della nostra mente e come si manifestano.
Queste tendenze sono talvolta chiamate \textbf{«leggi della percezione»}.
\begin{itemize}
    \item Non sono vere leggi, ma principi o euristiche importanti
    \item Nelle prossime slide vedremo alcuni di questi principi
    \item Capendoli, possiamo usarli per rendere le nostre UI più intuitive
    \item Vogliamo che le nostre UI lavorino con, e non contro, il modo in cui il cervello elabora gli stimoli visivi
\end{itemize}

\section{Principi della Gestalt}
\subsection{Somiglianza}
\begin{figure}[htbp!]
    \centering
    \includegraphics[width=0.5\textwidth]{immagini/lezione15/3.png}
    \caption{Forme righe-colonne}
\end{figure}
Scommetto che hai interpretato l’immagine sopra come quattro colonne e non tre righe.
Gli elementi che condividono una caratteristica visiva sono percepiti come più correlati rispetto agli elementi dissimili.
Le caratteristiche visive possono essere forme, dimensioni, colori, font, movimento, orientamento...

\begin{figure}[H]
    \centering
    \includegraphics[width=0.5\textwidth]{immagini/lezione15/4.png}
    \caption{Aggiungendo un colore...}
\end{figure}

Aggiungendo un’altra caratteristica visiva (colore), la percezione può cambiare.
Probabilmente ora vedi tre righe.
La somiglianza cromatica spesso prevale su altre caratteristiche visive.

\subsection{Somiglianza nelle UI}
La somiglianza può essere usata per raggruppare elementi correlati.
\begin{itemize}
    \item Se vuoi che elementi diversi siano percepiti come raggruppati e correlati, puoi farli condividere una o più caratteristiche visive
    \item Può essere usata per comunicare funzionalità comuni (ad esempio: pensa ai colori per segnalare i link nelle pagine web)
\end{itemize}
La somiglianza può anche essere usata per enfatizzare le differenze.

\includepdf[pages=-, addtotoc={1, subsection, 1, Prossimità, L7:9}]{esempi/lezione15-1.pdf}

\subsection{Prossimità: moduli}
Moduli lunghi con molti campi di input possono sembrare \textbf{opprimenti}.
Raggruppare i campi correlati aiuta gli utenti a comprendere le informazioni da inserire.
Lo spazio è un modo per raggruppare i campi correlati.

\subsection{Prossimità: posizionamento delle etichette nei moduli}
L’approccio più sicuro è posizionare le etichette sopra i campi di input.
Se vuoi moduli più compatti, considera di posizionare le etichette a sinistra dei campi.
\begin{itemize}
    \item Le etichette dovrebbero avere lunghezza simile e non essere troppo distanti dai campi
    \item Le etichette allineate a destra sono note per \href{https://www.nngroup.com/articles/right-justified-navigation-menus/}{ostacolare la scansione}
\end{itemize}
Attenzione a non raggruppare elementi non correlati!
Può nascondere elementi non correlati, rendendoli meno visibili.

\includepdf[pages=-, addtotoc={
    1, subsection, 1, Proximity can backfire: example , L7:10,
    5, subsection, 1, Principle of Connectedness, L7:11,
    12, subsection, 1, Principle of Common Region, L7:12,
    15, subsection, 1, Principle of Common Region: examples, L7:13,
    17, section, 1, Visual Hierarchy in UI Design , L7:14,
    20, subsection, 1, Creating a Visual Hierarchy, L7:15,
    21, subsection, 1, Scala, L7:16,
    22, subsection, 1, Colore/Contrasto, L7:17,
    23, subsection, 1, Raggruppamento, L7:18
}]{esempi/lezione15-2.pdf}