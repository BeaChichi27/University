\chapter{Lezione 10: Il design delle cose di tutti i giorni e la natura delle interazioni giornaliere}

\section{Le interazioni}
Alcune interazioni sono più fluide di altre.
Spesso, le difficoltà nell'uso di un sistema non derivano
da complessità profonde e sottili.
Falliamo nell'usare molti oggetti quotidiani!
Quindi, cosa rende difficile un'interazione?
Per rispondere, dobbiamo capire cosa accade
quando qualcuno fa (o prova a fare) qualcosa.
Don Norman ha formulato una teoria per spiegare le
fasi dell'azione.

\subsection{La Struttura dell'Azione: Il Ciclo di Azione}

Per realizzare qualcosa, è
necessario partire da un'idea di
ciò che si vuole ottenere
(obiettivo).
Poi, bisogna agire sul mondo
esterno, muoversi e interagire
con qualcuno o qualcosa (esecuzione).
Infine, verifichiamo se il nostro
obiettivo è stato effettivamente raggiunto
(valutazione).

\subsection{Fasi di Esecuzione}

Per portare alle azioni, gli obiettivi devono
essere trasformati in una dichiarazione specifica
di ciò che deve essere fatto
(intenzione).
Le intenzioni devono essere tradotte in
una sequenza di azioni da eseguire
per soddisfare l'intenzione.
La sequenza di azioni deve essere
fisicamente eseguita, cioè
realizzata nel mondo.
La valutazione inizia con la nostra
percezione del mondo.
La percezione deve essere
interpretata secondo le nostre
aspettative.
Quindi, l'interpretazione viene
valutata rispetto alle nostre
intenzioni e al nostro obiettivo.

\subsection{Sette Fasi dell'Azione: Esempio}

Stiamo leggendo un libro e la stanza sta diventando troppo buia per leggere.
\begin{itemize}
\item Stabilire l'obiettivo: Aumentare la luce nella stanza
\item Formare l'intenzione: Accendere la lampada
\item Specificare la sequenza di azioni: Camminare verso la lampada, raggiungere l'interruttore, azionare l'interruttore
\item Eseguire la sequenza di azioni: [camminare, raggiungere, azionare]
\item Percepire lo stato del sistema: [sentire il suono ``click'', vedere la luce dalla lampada]
\item Interpretare lo stato del sistema: L'interruttore ha cambiato posizione. La lampada emette luce. La lampada sembra funzionare
\item Valutare lo stato del sistema rispetto agli obiettivi e alle intenzioni: Il livello di luce è aumentato [obiettivo soddisfatto]
\end{itemize}

\section{Le Sette Fasi dell'Azione di Don Norman}
Le sette fasi dell'azione sono un modello approssimativo.
Il processo può iniziare da qualsiasi fase.
\begin{itemize}
\item Le persone non sempre si comportano come organismi logici
\item Gli obiettivi possono essere confusi, mal definiti e vaghi
\item Possiamo rispondere agli eventi del mondo (comportamento guidato dagli eventi)
\item Alcune azioni sono opportunistiche piuttosto che pianificate. Le eseguiamo se si presenta l'opportunità
\end{itemize}

\subsection{Sette Fasi dell'Azione: Alternative}
Notare che un determinato obiettivo può essere soddisfatto utilizzando diverse intenzioni e
diverse sequenze di azioni!
Se qualcuno entrasse nella stanza e passasse vicino alla lampada, potremmo modificare
la nostra intenzione da premere l'interruttore a chiedere all'altra persona di farlo
per noi.

Le difficoltà spesso risiedono interamente nel
derivare le relazioni tra
intenzioni e interpretazioni mentali
e azioni e stati fisici.
Norman identifica due "golfi" (divari) che
separano gli stati mentali da quelli fisici
\begin{itemize}
\item Golfo dell'Esecuzione e Golfo della Valutazione
\item Ogni golfo riflette un aspetto della
distanza tra la rappresentazione mentale
dell'utente e i componenti fisici e gli stati
dell'ambiente
\end{itemize}

\section{Principi di Design per aiutare a colmare i golfi}
\subsection{Affordances}
Le affordances forniscono indizi forti sul
funzionamento delle cose.
I pulsanti sono fatti per essere premuti. Le manopole sono fatte
per essere girate. Le fessure sono fatte per inserirvi oggetti.
Le palle sono fatte per essere lanciate o fatte rimbalzare.
Quando si sfruttano le affordances, gli
utenti sanno cosa fare semplicemente guardando: non
serve alcuna immagine, etichetta o istruzione. Le cose complesse potrebbero richiedere spiegazioni, ma
le cose semplici non dovrebbero!
Le false affordances sembrano offrire una particolare capacità, ma in realtà
ne offrono una diversa (o nessuna)!
Nelle UI: ad esempio: se qualcosa sembra un pulsante ma non è cliccabile.
Le affordances nascoste si verificano quando gli indizi che indicano la funzione di un elemento
non sono evidenti e potrebbero non essere visualizzati fino a quando l'azione
non viene intrapresa.

\subsection{Vincoli}
Il modo più semplice per assicurarsi che qualcosa sia facile da usare, con pochi errori,
è rendere impossibile fare diversamente limitando le scelte dell'utente.

\subsection{Feedback}
Ogni azione con effetti collaterali rilevanti
dovrebbe essere esplicitamente confermata dal
sistema.
Il feedback dovrebbe essere immediato e
informativo. Preferibilmente non distrattivo e
discreto.

\subsection{Coerenza}
Le interfacce dovrebbero essere coerenti in modi significativi
\begin{itemize}
\item All'interno dell'applicazione stessa (coerenza interna)
\item Con altre applicazioni esterne (coerenza esterna)
\end{itemize}
La coerenza aiuta gli utenti a colmare i golfi della valutazione e dell'esecuzione.
Ad esempio: se tutte le azioni sono confermate tramite un messaggio toast nell'angolo in alto a destra
(come nell'app Lista delle cose da fare che abbiamo visto alcune diapositive fa), è più facile per gli utenti
capire in quale stato si trova il sistema. Se i messaggi di conferma fossero diversi per
ogni azione...

\subsection{Metafore}
Possono essere utili nelle UI per suggerire un modello mentale esistente e sfruttare
conoscenze specifiche che gli utenti già possiedono in domini diversi
\begin{itemize}
\item Carrozze senza cavalli, telefoni senza fili...
\item Metafora del desktop: Non è un tentativo di simulare una scrivania reale, ma mira a sfruttare la conoscenza che gli utenti hanno di file, documenti, cartelle, cestini, ...
\end{itemize}

\subsection{Mappature}
Una mappatura è una corrispondenza tra un'interfaccia e l'azione corrispondente nel
mondo reale.
Mappature efficaci (naturali) possono minimizzare
i passaggi cognitivi per trasformare un'azione in
effetto, o accelerare il processo di
trasformazione della percezione in
comprensione.
Le mappature naturali possono anche ridurre il carico
sulla memoria.

\section{Linee Guida di Norman per colmare i golfi}
\begin{itemize}
\item Visibilità. Guardando, l'utente può comprendere lo stato del dispositivo e le
alternative per l'azione.
\item Un buon modello concettuale. Il designer fornisce un buon modello
concettuale per l'utente, con coerenza nella presentazione delle operazioni
e dei risultati e un'immagine del sistema coerente e consistente.
\item Buone mappature. È possibile determinare le relazioni tra
azioni e risultati, tra i controlli e i loro effetti, e
tra lo stato del sistema e ciò che è visibile.
\item Feedback. L'utente riceve un feedback completo e continuo sui
risultati delle azioni.
\end{itemize}

\subsection{Caratteristiche di un Buon Design}
Ha affordances (rende le operazioni visibili)
Offre mappature ovvie (rende evidente la relazione tra l'azione
effettiva del dispositivo e l'azione dell'utente)
Fornisce feedback sull'azione dell'utente
Fornisce un buon modello mentale del comportamento sottostante del sistema
\begin{itemize}
\item Un modello mentale è la rappresentazione interna che un utente ha di come un sistema o
un'interfaccia funziona.
\item Un buon modello mentale consente agli utenti di prevedere come si comporterà l'interfaccia
e li aiuta a interagire efficacemente con essa.
\end{itemize}
Fornisce vincoli (per prevenire errori)