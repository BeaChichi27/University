\chapter{Lezione 3 - Requisiti d'ingegneria: elicitazione e analisi}
\section{Il ciclo di vita del software}
\subsection{Requisiti software}
I requisiti sono descrizioni di quello che il sistema dovrebbe fare:
\begin{itemize}
    \item I \textbf{servizi} che il sistema dovrebbe dare agli utenti.
    \item I \textbf{vincoli operativi}.
\end{itemize}
Il termine \textbf{«requisito»} è usato in modo incoerente nell'industria del software. In alcuni casi, un requisito è una descrizione astratta e ad alto livello di un servizio che il sistema dovrebbe fornire o di un vincolo sul sistema. 
\begin{itemize}
    \item Questi sono chiamati \textbf{Requisiti Utente} (l'attenzione è sulla prospettiva degli utenti finali)
\end{itemize}
In altri, è una definizione più dettagliata e formale di una funzione del sistema 
\begin{itemize}
    \item Questi sono chiamati \textbf{Requisiti di Sistema} (l'attenzione è sul sistema da costruire)
\end{itemize}
Tale ambiguità è inevitabile, poiché i requisiti possono avere una doppia funzione: 
\begin{itemize}
    \item I \textbf{requisiti degli utenti} possono essere la base per un'offerta per un contratto
    \begin{itemize}
        \item Il cliente può definire requisiti generali degli utenti
        \item Diversi appaltatori possono proporre modi diversi per soddisfare i requisiti degli utenti
    \end{itemize}
    \item I \textbf{requisiti di sistema} possono essere la base per il contratto stesso
    \begin{itemize}
        \item Una volta assegnato un contratto per lo sviluppo del software, l'appaltatore formula un insieme più dettagliato di requisiti di sistema
        \item In modo che il cliente comprenda esattamente cosa farà il software e possa validare la proposta
        \item Una volta accettati e convalidati, i requisiti di sistema possono essere inseriti nel contratto finale e sono vincolanti!
    \end{itemize}
\end{itemize}
L'ingegnerizzazione dei requisiti (RE) è una sottoarea dell'Ingegneria del software che si occupa del processo di definire i requisiti di per un quel-che-sarà-software. L'obiettivo è fornire agli ingegneri del software dei metodi, tecniche e strumenti per capire e documentare quel che un sistema software deve fare.
\subsection{Requisiti funzionale e non-funzionale}
I requisiti sono spesso classificati come funzionali o non funzionali:
\begin{itemize}
    \item \textbf{Requisiti Funzionali}:
    \begin{itemize}
        \item Servizi che il sistema dovrebbe fornire
        \item Come il sistema dovrebbe reagire a determinati input o comportarsi in una data situazione
    \end{itemize}
    \item \textbf{Requisiti Non-Funzionali}:
    \begin{itemize}
        \item Vincoli sui servizi o funzioni offerti dal sistema
        \item Includono vincoli temporali, vincoli di processo o vincoli imposti da standard
        \item Spesso si applicano all'intero sistema piuttosto che a singole caratteristiche o servizi
    \end{itemize}
\end{itemize}
In pratica, la distinzione tra questi due tipi di requisiti non è netta. Considera quanto segue: \textit{solo gli utenti autorizzati dovrebbero poter accedere al sistema}.
\newline
Sembra un requisito non funzionale, tuttavia, quando sviluppiamo in maggiore dettaglio, genera requisiti aggiuntivi che sono chiaramente funzionali, ad esempio, gli utenti devono essere in grado di effettuare il login e autenticarsi.
\newline
\textbf{I requisiti non sono indipendenti e un requisito spesso genera o vincola altri requisiti}
\subsubsection{Requisiti Funzionali}
Quando espressi come \textbf{requisiti funzionali degli utenti}, possono essere scritti in linguaggio naturale, in modo che possano essere compresi da persone non tecniche (ad es.: utenti e manager). 
\newline
Quando espressi come \textbf{requisiti funzionali del sistema}, dovrebbero essere dettagliati, descrivere accuratamente gli input e gli output del sistema e le eccezioni, in modo che gli ingegneri del software sappiano esattamente cosa implementare.
\subsubsection{Requisiti non-funzionali}
I requisiti non-funzionali sono non direttamente collegati ai servizi specifici offerti dal sistema. Tipicamente specifica o vincola le caratteristiche del sistema come un insieme:
\begin{itemize}
    \item I vincoli su come dovrebbero essere implementati.
    \item Specifica altre proprietà.
\end{itemize}
Spesso i requisiti non-funzionale sono molto più critici di quelli funzionali.
\begin{itemize}
    \item Gli utenti possono trovare una strada attorno all'implementazione sub-ottimale di una richiesta funzionale.
    \item ...ma fallendo a incontrare i requisiti non-funzionali che possono essere indice di un sistema instabile.
\end{itemize}
\subsubsection{Tipi di requisiti non-funzionali}
I requisiti non funzionali possono derivare da: 
\begin{itemize}
    \item Caratteristiche richieste del prodotto (\textbf{Requisiti di Prodotto}) 
    \begin{itemize}
        \item La velocità del sistema, la quantità di memoria richiesta, requisiti di usabilità, tassi di fallimento accettabili, ...
    \end{itemize}
    \item Le organizzazioni del cliente e degli sviluppatori (\textbf{Requisiti Organizzativi}) 
    \begin{itemize}
        \item Processi operativi che descrivono come il sistema sarà utilizzato, linguaggi di programmazione richiesti, requisiti che specificano l'ambiente operativo, ...
    \end{itemize}
    \item Fonti esterne (\textbf{Requisiti Esterni}) 
    \begin{itemize}
        \item Cosa deve essere fatto affinché il sistema venga approvato da un regolatore di terze parti, requisiti legali, requisiti etici per garantire che il sistema sia accettabile per i suoi utenti e per il pubblico in generale.
    \end{itemize}
\end{itemize}
\section{Proprietà dei buoni requisiti}
I requisiti dovrebbero essere:
\begin{itemize}
    \item \textbf{Chiari e facili da comprendere} (soprattutto i requisiti dell'utente)
    \item \textbf{Univoci }(l'ambiguità porta a controversie con i clienti)
    \item \textbf{ Completi }
    \item \textbf{Coerenti} (cioè, non dovrebbero confliggere tra loro)
    \item \textbf{Verificabili} (la mancanza di verificabilità porta a controversie con i clienti). Dato un requisito, dovrebbe essere possibile determinare in modo univoco se il sistema soddisfa quel requisito.
\end{itemize}
\subsection{Requisiti di testabilità}
Dovrebbe essere possibile determinare in modo univoco se il sistema soddisfa un requisito. Altrimenti, gli sviluppatori potrebbero sostenere che il requisito è soddisfatto, mentre il cliente potrebbe non essere affatto d'accordo! I Requisiti Funzionali del Sistema dovrebbero essere il più dettagliati possibile e non lasciare spazio all'interpretazione. I Requisiti Non Funzionali del Sistema dovrebbero includere indicatori quantitativi ogni volta che è possibile.
\subsection{Requisiti non-funzionali testabili}
\textbf{Requisiti non molto testabili}:
\begin{itemize}
    \item[a] Il sistema dovrebbe essere affidabile
    \item[b] Il sistema dovrebbe essere semplice da usare
    \item[c] Il sistema dovrebbe essere veloce e reattivo 
\end{itemize}
\textbf{Requisiti più testabili}
\begin{itemize}
    \item[a] Il sistema dovrebbe avere un tempo di attività al 99.9\% al mese.
    \item[b] Gli utenti dovrebbero essere capaci di usare tutte le funzioni di sistema dopo al più due ore di allenamento e il numero medio di errori lato utente dovrebbero non superare i 2 all'ora per l'utilizzo del sistema.
    \item[c] Il tempo di risposta medio del sistema non dovrebbe superare i 100 millisecondi.
\end{itemize}
\section{Il processo di ingegnerizzazione del requisito}
I 3 passaggi chiave sono:
\begin{itemize}
    \item \textbf{Elicitazione e analisi dei requisiti}: scopre i requisiti interagendo con gli stackholder.
    \item \textbf{Specificazione del requisito}: converte i requisiti in una forma standardizzata.
    \item \textbf{Validazione del requisito}: controlla che i requisiti definiscano il sistema che il cliente vuole,
\end{itemize}
\section{Il processo RE}
In pratica, il processo RE non è lineare, le sue attività sono attualmente interlivellate in un processo iterativo con livelli diversi di granularità, passando per i requisiti: \textbf{a livello di business}, \textbf{dell'utente} e \textbf{di sistema}.
\subsection{Elicitazione e analisi dei requisiti}
L'obiettivo di quest'attività è di trovare quel che i cliente vogliono e necessitano ed è considerata la parte più critica, vista la dovuta collaborazione con gli stakeholders.
\newline
Gli stakeholder non sanno cosa vogliono dal software, tranne che in termini molto generali. Non sanno cosa sia fattibile e cosa non lo sia, e potrebbero fare richieste irrealistiche. Gli stakeholder sono esperti nel loro dominio. Esprimono naturalmente i requisiti nei propri termini e gergo, con una conoscenza implicita del loro lavoro. Potrebbero dare per scontati alcuni dettagli, quando in realtà non lo sono. Diversi stakeholder, con requisiti diversi, possono esprimere i loro requisiti in modi diversi. Gli ingegneri dei requisiti devono lavorare attorno a comunanze e conflitti. Fattori politici possono influenzare il processo di elicitation. I manager possono richiedere requisiti specifici perché questi consentirebbero loro di aumentare la loro influenza nell'organizzazione. Gli stakeholder hanno opinioni diverse (e talvolta conflittuali) sull'importanza e la priorità dei requisiti. Se alcuni stakeholder sentono che le loro opinioni non sono state adeguatamente considerate, potrebbero tentare deliberatamente di minare il processo di ingegneria dei requisiti. I requisiti cambieranno durante il processo di elicitation. Nuovi requisiti possono emergere da nuovi stakeholder che non erano stati inizialmente considerati.
\section{Processo dell'elicitazione e analisi dei requisiti}
Ci sono de passaggi da seguire:
\begin{enumerate}
    \item \textbf{Scoprire e capire}: interagire con gli stakeholder per scoprire i requisiti.
    \item \textbf{Classificazione e organizzazione}: i requisiti correlati sono raggruppati e organizzati.
    \item \textbf{Prioritizzazione e negoziazione}: risolve i confliti dei requisiti che sorgono dai conflitti delle necessità dei vari stakeholder.
    \item \textbf{Documentazione}: tiene traccia dei requisiti scoperti per la prossima iterazione del processo di elicitazione.
\end{enumerate}
\section{Importanza della Conoscenza degli Utenti}
La comprensione degli utenti è fondamentale per:
\begin{itemize}
    \item Scoprire i requisiti
    \item Progettare interfacce utente efficaci
\end{itemize}

\subsection{Personas}
\begin{itemize}
    \item \textbf{Definizione}: Archetipi ipotetici che rappresentano utenti reali
    \item \textbf{Scopo}: Promuovere l'empatia e comprendere meglio gli utenti
    \item \textbf{Caratteristiche}:
    \begin{itemize}
        \item Personalizzazione (nome, età, biografia)
        \item Informazioni lavorative
        \item Formazione scolastica
        \item Obiettivi
        \item Punti critici/Frustrazioni
    \end{itemize}
    \item Non sono persone reali ma vengono definiti con rigore
    \item Emergono dal processo di elicitatione dei requisiti
\end{itemize}

\subsection{Casi d'Uso delle Personas}
Utili quando:
\begin{itemize}
    \item Sviluppiamo software per il pubblico generico
    \item Gli end-user includono il pubblico generico
    \item Non abbiamo stakeholder specifici da intervistare
\end{itemize}

\subsection{Storie Utente}
\begin{itemize}
    \item Descrizioni narrative di come il sistema può essere utilizzato
    \item Presentano:
    \begin{itemize}
        \item Azioni degli utenti
        \item Informazioni utilizzate
        \item Output richiesti
    \end{itemize}
    \item Efficaci per comunicare obiettivi generali
    \item Gli stakeholder spesso le descrivono meglio dei requisiti tecnici
\end{itemize}

\section{Mockup a bassa fedeltà}
Questi mockup (o wireframe) sono bozze semplificate delle interfacce del sistema lato utente. Sono modi veloci e affetti dal costo per visualizzare la struttura e lo scorrimento base del sistema. L'interesse principale è la funzionalità e il flusso, non l'estetica.
\newline
I suoi benefici sono:
\begin{itemize}
    \item \textbf{Comprensione facilitata}: aiuta gli stakeholder a capire le funzionalità base del sistema. Dando un'idea chiara dell'idea iniziale.
    \item \textbf{Aiuta a scoprire i nuovi requisiti}: è più semplice ragionare su interfacce concrete che di proposte astratte.
    \item \textbf{Feedback interessannte e interattivo}: promuove l'attenzione da parte degli stakeholder e permette una rapida iterazione e feedback senza un design delineato.
    \item \textbf{Fondamenta per lo sviluppo}: provedde una solida fondamenta per spostarsi verso design ad alta fedeltà e a un'eventuale sviluppo.
\end{itemize}